\documentclass[addpoints,10pt]{exam}

\usepackage{amsmath,amsthm,enumitem,wrapfig,amsfonts,mathtools}
\usepackage[mathscr]{euscript}
\usepackage[super]{nth}

\usepackage{geometry}
\usepackage[T1]{fontenc} % Use 8-bit encoding that has 256 glyphs
\renewcommand{\rmdefault}{ptm} %Change the Front Family from the default(cmr) to ptm(Times)
\usepackage{amsmath,amsfonts,amsthm,amssymb} % Math packages
\usepackage{bm}
\usepackage{mathptmx}
\usepackage{graphicx}
\usepackage{sectsty} % Allows customizing section commands
% \allsectionsfont{\centering} % Make all sections centered, the default font and small caps
\usepackage{pgfplots}
\pgfplotsset{compat=1.18}
\usetikzlibrary{arrows.meta}
\usepackage{xcolor}
\definecolor{darkpastelgreen}{rgb}{0.01, 0.75, 0.24}
\definecolor{blue-violet}{rgb}{0.54, 0.17, 0.89}
%\usepackage{polylongdiv} %polynomial long division
% Custom problem environment
\usepackage{polynom}
\newcounter{cprob}
\newenvironment{cprob}[1]{%
    \setcounter{cprob}{#1}%
    \noindent\textbf{Problem \thecprob.}%
}{%
    \par\bigskip%
}

\theoremstyle{plain}
\newtheorem{thm}{\protect\theoremname}
  \theoremstyle{definition}
  \newtheorem{prob}[thm]{Problem}
  \newtheorem*{problem*}{Open Problem}
  \theoremstyle{plain}
  \newtheorem{conjecture}[thm]{Conjecture}
  \theoremstyle{plain}
  \newtheorem{lem}[thm]{Lemma}
  \newtheorem*{lem*}{Lemma}
  \newtheorem{obs}[thm]{Observation}
  \newtheorem{cor}[thm]{Corollary}
  \theoremstyle{definition}
\newtheorem{definition}[thm]{Definition}


% Patch prob environment to be single spaced
\let\oldprob\prob
\let\endoldprob\endprob
\renewenvironment{prob}
  {\begin{singlespace}\oldprob}
  {\endoldprob\end{singlespace}}

% start problem one line below like for enumerated problems with multiple parts
\newcommand{\belowtitle}{\leavevmode\newline}
%\Observe command
\newcommand{\Observe}{\text{Observe.}}
%(=>)
\newcommand{\IF}{\mathbf{(\Rightarrow)}}
%(<=)
\newcommand{\FI}{\mathbf{(\Leftarrow)}}
%equivalence classes; \class[S]{ *content in square brackets* }
\newcommand{\class}[2][]{\ensuremath{\left[\,#2\,\right]_{#1}}}

\newcommand{\horrule}[1]{\rule{\linewidth}{#1}}
\newcommand{\kk}{\ensuremath{\Bbbk}} 
\newcommand{\CC}{\ensuremath{\mathbb{C}}}
\newcommand{\FF}{\ensuremath{\mathbb{F}}}
\newcommand{\KK}{\ensuremath{\mathbb{K}}}
\newcommand{\NN}{\ensuremath{\mathbb{N}}}
\newcommand{\QQ}{\ensuremath{\mathbb{Q}}} 
\newcommand{\RR}{\ensuremath{\mathbb{R}}} 
\newcommand{\ZZ}{\ensuremath{\mathbb{Z}}}
\newcommand{\MM}{\ensuremath{\mathcal{M}}}
\newcommand{\TT}{\ensuremath{\mathcal{T}}}
\newcommand{\BB}{\ensuremath{\mathcal{B}}}
\newcommand{\VV}{\ensuremath{\mathcal{V}}}
\newcommand{\WW}{\ensuremath{\mathcal{W}}}
\newcommand{\UU}{\ensuremath{\mathcal{U}}}
\newcommand{\PP}{\ensuremath{\mathcal{P}}}
\newcommand{\LL}{\ensuremath{\mathcal{L}}}

\newcommand{\sm}{\char`\\}
%vector stuff
\DeclarePairedDelimiter{\ip}{\langle}{\rangle} %inner product/generate
\DeclarePairedDelimiter{\norm}{\lVert}{\rVert} %norm
\DeclarePairedDelimiter{\sqb}{\lbrack}{\rbrack} %corrd
\newcommand{\floor}[1]{\left\lfloor #1 \right\rfloor}
\newcommand{\ceil}[1]{\left\lceil #1 \right\rceil}
\newcommand{\mbf}[1]{\ensuremath{\mathbf{#1}}}
\newcommand{\tbf}[1]{\textbf{ #1 }}
\newcommand{\Span}{\ensuremath{\mathrm{Span}}}


\makeatletter
\renewcommand*\env@matrix[1][*\c@MaxMatrixCols c]{%
  \hskip -\arraycolsep
  \let\@ifnextchar\new@ifnextchar
  \array{#1}}
\makeatother

\def\env@matrix{\hskip -\arraycolsep
  \let\@ifnextchar\new@ifnextchar
  \array{*\c@MaxMatrixCols c}}

  \newcommand{\proj}[2]{\text{proj}_{#1}(#2)}
  

%%% Formatting: Page Header
\newcommand{\StudentName}{Danny Banegas}
\newcommand{\AssignmentName}{Homework 3}
\newcommand{\CourseName}{MATH 721 - Algebra II}


\pagestyle{headandfoot}
\runningheadrule
\firstpageheadrule
\firstpageheader{\CourseName}{\StudentName}{\AssignmentName}
\runningheader{\CourseName}{\StudentName}{\AssignmentName}
\firstpagefooter{}{\thepage}{}
\runningfooter{}{\thepage}{}

\printanswers

\DeclareMathAlphabet{\mathcal}{OMS}{cmsy}{m}{n}

\usepackage{parskip}
\usepackage{setspace}
\doublespacing
% % % % % % % % % % % % % % % % % % % % % % % % % % % % % % % % % % % % % % % % % % % % % % % % % % % % % % % % % % % % % % 
\begin{document}
%%%%%%%%%%%%%%%%%%%%%%%%%%%%%%%%%%% 40 %%%%%%%%%%%%%%%%%%%%%%%%%%%%%%%%%%%%%%
\setcounter{thm}{39}   % next prob is 40
\begin{prob}
Prove that an abelian group has a composition series if and only if it is finite.
\end{prob}

\begin{proof}
  $(\impliedby)\;$ If $G$ is finite, it must have order $n=\prod_{i=1}^{m}p_{i}^{a_{i}}$ for some distinct primes $p_{1},\hdots, p_{m}$ and $a_{0},\hdots, a_{m}\in \ZZ^{+}.$ Every subgroup of $G$ is normal since it's abelian, so each each Sylow $p_{i}-$subgroup $P_{i}<G$ of order $p_{i}^{a_{i}}$ is normal. So by $\textbf{Problem 36}$, $G$ is the internal direct product $G=P_{1}\cdots P_{m}\cong P_{1}\times \cdots \times P_{m}$ of its Sylow subgroups. Now consider any $a,b\in P_{1}\cdots P_{k}\setminus P_{1}\cdots P_{k-1}$ for some $1< k\leq m$. 
  \begin{align*}
    &[a]=[b]\in P_{1}\cdots P_{k}/ P_{1}\cdots P_{k-1} \implies b^{-1}a\in P_{1}\cdots P_{k-1}\implies |b^{-1}a|\text{ divides }p_{1}^{a_{1}}\cdots p_{k-1}^{a_{k-1}}.\\
    &a,b\in P_{1}\cdots P_{k}\setminus P_{1}\cdots P_{k-1}=P_{k}\setminus\{e\}\text{ since }P_{1}\cdots P_{k-1}\cap P_{k}=\{e\}.
  \end{align*}
  $\text{So }|a|,|b|\in \{p_{k}^{i}\mid 1\leq i\leq a_{k}\} \text{ and without loss of generality },|a|=p_{k}^\alpha,\,|b|=p_{k}^{\beta}\text{ for some }0 \leq \alpha\leq \beta\leq a_{k}.$ So then since $G$ is Abelian, $|b^{-1}a|$ divides $\mathrm{lcm}(|a|,|b|)=p_{k}^{\beta}$. So the $|b^{-1}a|$ divides $p_{k}^{\beta}$ and $p_{1}^{a_{1}}\cdots p_{k-1}^{a_{k-1}}$, and since $\mathrm{gcd}(p_{1}^{a_{1}}\cdots p_{k-1}^{a_{k-1}},p_{k}^{\beta})=1$, $|b^{-1}a|$ must in fact be $1$. So $b^{-1}a=e\implies a=b$. On the other hand, $a=b\implies [a]=[b]$ by definition. Therefore, for any $a,b\in P_{1}\cdots P_{k}\setminus P_{1}\cdots P_{k-1}$:
    $$[a]=[b]\in P_{1}\cdots P_{k}/ P_{1}\cdots P_{k-1}\iff a=b.$$
  Well, any $g\in P_{1}\cdots P_{k}$ is either in $P_{1}\cdots P_{k}\setminus P_{1}\cdots P_{k-1}$ or it isn't, so pick some $q\in P_{1}\cdots P_{k}\setminus P_{1}\cdots P_{k-1}$.
  $$[g]=\begin{cases}[q],\,\text{ if }g\in P_{1}\cdots P_{k}\setminus P_{1}\cdots P_{k-1}\\ [e],\text{ if }g\in P_{1}\cdots P_{k-1}\\ \end{cases}$$
  Therefore, $P_{1}\cdots P_{k}/ P_{1}\cdots P_{k-1}=\{[e],[q]\}\cong \ZZ_{2}$ is simple for each $1< k\leq m$, and by the same sort of argument $P_{1}/\{e\}$ is simple since $[a]=[b]\iff b^{-1}a\in \{e\}\iff a=b\implies P_{1}/\{e\}=\{[e],[g]\}$ for any $g\in P_{1}\setminus\{e\}$. So $\{e\} \triangleleft P_{1}\triangleleft P_{1}P_{2} \triangleleft \cdots \triangleleft P_{1}P_{2} \cdots P_{m-1}\triangleleft P_{1}P_{2}\cdots P_{m}=G$ is a composition series. We prove the other direction on the following page.

  \newpage $(\implies)\;$ If an abelian group $G$ has a composition series 
    $$\{e\}=H_{0}\triangleleft H_{1} \triangleleft \cdots \triangleleft H_{n-1}\triangleleft H_{n}=G$$
  for some $n\in \ZZ^{+}$, then for each $1\leq k\leq n$, $H_{k}/H_{k-1}$ is simple and abelian. So then for any $g\in H_{k}$, $\langle [g]\rangle=\{e\}$ or $H_{k}/H_{k-1}.$ If $\langle g\rangle=\{e\},\;\forall g\in H_{k}/H_{k-1}$, then $H_{k}/H_{k-1}=\{[e]\}$, otherwise $\exists g_{*}\in H_{k}$ such that $\langle [g_{*}]\rangle = H_{k}/H_{k-1}$. In either case $H_{k}/H_{k-1}$ is cyclic. Suppose $H_{k}/H_{k-1}$ infinite, so $H_{k}/H_{k-1}\cong \ZZ$. But then $H_{k}/H_{k-1}$ isn't simple since $\ZZ$ isn't simple ($\{e\}\triangleleft 2\ZZ\triangleleft \ZZ$), a contradiction. So $H_{k}/H_{k-1}$ must be a simple finite cyclic group, which implies it has prime order since $H_{k}\triangleright H_{k-1}\implies |H_{k}/H_{k-1}|>1$. Observe.
  \begin{center}
    $[H_{1}:H_{0}]\in \ZZ^{+}\implies |H_{1}|=|H_{0}|[H_{1}:H_{0}]=(1)[H_{1}:H_{0}]\in \ZZ^{+}$. Suppose $|H_{k}|\in \ZZ^{+}$ for some $1\leq k\leq n$. Therefore, $|H_{k+1}|=|H_{k}|[H_{k+1}:H_{k}]\in \ZZ^{+}$. So then $|H_{m}|\in \ZZ^{+}$ for all $0\leq m\leq n$.
  \end{center}
  So $|H_{n}|=|G|\in \ZZ^{+}$.

  Thus,

  \begin{center}
  An abelian group has a composition series if and only if it is finite.
  \end{center}
\end{proof}
\newpage
%%%%%%%%%%%%%%%%%%%%%%%%%%%%%%%%%%% 41 %%%%%%%%%%%%%%%%%%%%%%%%%%%%%%%%%%%%%%
\begin{prob}
Prove that a solvable simple group is abelian.
\end{prob}

\begin{proof}
Since $G$ is simple, $Z(G)\trianglelefteq G$ is either $\{e\}$ Suppose $Z(G)=\{e\}$, and consider the commutator subgroup $G'=\langle a^{-1}b^{-1}ab\mid a,b\in G\rangle\trianglelefteq G$. $G'\neq \{e\}$, otherwise $a^{-1}b^{-1}ab=e,\forall a,b\in G\implies G$ is abelian $\implies Z(G)=G$, a contradiction. So $G'=G$ and $G^{(2)}=(G')'=(G)'=G'=G$. Now suppose $G^{(k)}=G$ for some $k\geq 2$. Then $G^{k+1}=(G^{(k)})'=(G)'=G'=G$. But then $G^{n}=G\neq \{e\}$ for all $n\in \ZZ^{+}$, and $G$ isn't solvable. So $Z(G)=G$.

Thus,

\begin{center}
  A solvable simple group is abelian.
\end{center}
\end{proof}
We now prove a lemma for \textbf{Problem 42}.
\setcounter{thm}{0}   % next prob is 40
\begin{lem}
  Any subquotient (of subgroups normal to each other) of a consecutive quotient of derived subgroups is Abelian.
\end{lem}
\begin{proof}
Let $G^{(k-1)}\trianglerighteq  H_{1} \trianglerighteq H_{2}\trianglerighteq  \cdots \trianglerighteq H_{p}\trianglerighteq \cdots \trianglerighteq  H_{q}\trianglerighteq \cdots H_{m-1}\trianglerighteq H_{m} \trianglerighteq G^{(k)}$ for some $k,m\in \ZZ^{+}$ and some $0\leq p<q \leq m$. Well, 
$$(H_{p})'=\{g\in G^{(k)}\mid g=a^{-1}b^{-1}ab\text{ for }a,b\in H_{p}\trianglelefteq G^{(k-1)}\}\trianglelefteq G^{(k)}\implies H_{p}\trianglerighteq H_{q}\trianglerighteq G^{(k)}\trianglerighteq (H_{p})'.$$
So then $\forall a,b\in H_{p},\;(ba)^{-1}ab=a^{-1}b^{-1}ab\in (H_{p})'\leq H_{q}\implies (ab)H_{q}=(ba)H_{q}.$ Therefore, $[a][b]=[ab]=[ba]=[b][a]\in H_{p}/H_{q}$. So $H_{p}/H_{q}$ is abelian.
\end{proof}

I know now that I could have used some more theorems from class to make these proofs shorter. I was in too deep here and I derived important ideas myself by doing it this way so I'm cool with it. Apologies for the lengths though.
\newpage
%%%%%%%%%%%%%%%%%%%%%%%%%%%%%%%%%%% 42 %%%%%%%%%%%%%%%%%%%%%%%%%%%%%%%%%%%%%%
\setcounter{thm}{41}   % next prob is 42
\begin{prob}
Prove that a solvable group that has a composition series is finite.
\end{prob}

\begin{proof}
 If a solvable group $G$ with a composition series is abelian, then it is finite by \textbf{Problem 40}. Suppose such a group $G$ is not abelian.  There exists a minimal $n\in \ZZ^{+}$ such that $G^{(n)}=\{e\}$ since $G$ is solvable and we have $(i)$ the derived normal series and $(ii)$ some composition series of $G$:
  $$(i)\;G=G^{(0)}\trianglerighteq G'=G^{(1)}\trianglerighteq\cdots \trianglerighteq G^{(n-1)}\trianglerighteq G^{(n)}=\{e\}\text{ and }(ii)\;G=H_{0}\triangleright H_{1}\triangleright\cdots \triangleright H_{m-1}\triangleright H_{m}=\{e\}$$
By \textbf{Schreiers's Theorem} these normal series have an equivalent refinement, that is:
  \begin{align*}
    &(1)\;G_{i,j}=G^{(i+1)}(G^{(i)}\cap H_{j})\text{ for }\begin{array}{c} \scriptstyle 0\leq j\leq n-1\\[-12pt]\scriptstyle 0\leq j\leq m-1\end{array}\text{ and }(2) \;H_{i,j}=(G^{(i)}\cap H_{j})H_{j+1}\text{ for }\begin{array}{c} \scriptstyle 0\leq j\leq n\\[-12pt]\scriptstyle 0\leq j\leq m-1\end{array}\\
    \implies &(3) \begin{array}{c} \scriptstyle G= G^{0} = G_{0,0}\trianglerighteq G_{0,1}\trianglerighteq \cdots \trianglerighteq G_{0,m}=G'=G_{1,0}\trianglerighteq G_{1,1}\trianglerighteq \cdots \trianglerighteq G_{1,m}=G^{(2)}=G_{2,0}\trianglerighteq \cdots \trianglerighteq G_{n-1,m}=G^{(n)}=G_{n,0}=\{e\}.\\[-10pt] \scriptstyle G= H_{0} = H_{0,0}\trianglerighteq H_{1,0}\trianglerighteq \cdots \trianglerighteq H_{n,0}=H_{1}=H_{0,1}\trianglerighteq H_{1,1}\trianglerighteq \cdots \trianglerighteq H_{n,1}=H_{2}= H_{2,0}\trianglerighteq \cdots \trianglerighteq H_{n,m-1}=H_{n}=H_{0,m}=\{e\}\end{array}\\
    \text{ and }&(4)\; G_{i,j}/G_{i,j+1}\cong H_{i,j}/H_{i+1,j}.
  \end{align*}
  These series are normal by the \textbf{Butterfly Lemma} as stated in the class notes. Now, consider any $0\leq k \leq n$. We have $H_{k-1}=H_{0,k-1}\trianglerighteq \cdots \trianglerighteq H_{n,k-1}=H_{k}$ and $H_{k}$ is a maximal proper normal subgroup of $H_{k-1}$, that is: $H_{k}\trianglelefteq N\trianglelefteq H_{k-1}\implies N=H_{k-1}$ or $N=H_{k}$. So then since we have a containment chain, there exists some $0\leq p\leq n$ such that $H_{k-1}=H_{0,k-1}=\cdots = H_{p,k-1}\triangleright H_{p+1,k-1}=\cdots=H_{n,k-1}=H_{k}.$ Therefore, by $(4)$:
  $$H_{k-1}/H_{k}=H_{p,k-1}/H_{p+1,k-1}\cong G_{p,k-1}/G_{p,k}$$ which is abelian by \textbf{Lemma 1} since it is a subquotient of $G^{(p-1)}/G^{(p)}$. So then $H_{k-1}/H_{k}$ is abelian and simple, and we proved earlier in \textbf{Problem 40} that an abelian simple group must be cyclic and finite of prime order and that if quotients of a composition series of $G$ are finite, that $G$ itself is finite.

  Thus,
  \begin{center}
    A solvable group that has a composition series is finite.
  \end{center}
\end{proof}
\newpage
%%%%%%%%%%%%%%%%%%%%%%%%%%%%%%%%%%% 45 %%%%%%%%%%%%%%%%%%%%%%%%%%%%%%%%%%%%%%
\setcounter{thm}{44}   % next prob is 45
\begin{prob}
If $\KK\supseteq \FF$ is a field extension, $u,v\in \FF$, $v$ is algebraic over $\KK(u)$, and $v$ is transcendental over $\KK$, then $u$ is algebraic 
over $\KK(v)$.
\end{prob}
\begin{proof}
$v$ is algebraic over $\KK(u)\implies$ there exists a non-zero degree $n\in \ZZ^{+}$ polynomial $P(x)=\sum_{i=0}^{n}p_{i}(u)x^{i}$ over $\KK(u)$ such that $P(v)=0$. Let $m=\max\{\deg(p_{i}(x))\mid 0\leq i \leq n\}$. Then for each $0\leq i \leq n$, we have that $p_{i}(x)= \sum_{j=0}^{m}a_{ij}x^{j}$ for some $a_{i0},\hdots,a_{im}\in \KK$. Note that if $\deg{p_{i}(x)}<m,\,a_{i(\deg{p_{i}(x)})}=\cdots =a_{im}=0$. Observe.
\begin{align*}
  P(v)&=\sum_{i=0}^{n}p_{i}(u)v^{i}=\sum_{i=0}^{n}(\sum_{j=0}^{m}a_{ij}u^{j})v^{i}=\sum_{i=0}^{n}(\sum_{j=0}^{m}a_{ij}v^{i}u^{j})=\sum_{i=0}^{n}(a_{i0}v^{j}+a_{i1}v^{j}u+\cdots +a_{im}v^{j}u^{m})\\
  &=\sum_{i=0}^{n}a_{i0}v^{j}+\sum_{i=0}^{n}a_{i1}v^{j}u+\cdots +\sum_{i=0}^{n}a_{im}v^{j}u^{m}=\sum_{j=0}^{m}(\sum_{i=0}^{n}a_{ij}v^{i}u^{j})=\sum_{j=0}^{m}q_{j}(v)u^{j}=Q(u)=0,\\
  \text{where }q_{j}(x)&=\sum_{i=0}^{n}a_{ij}x^{j}\text{ and }Q(x)=\sum_{j=0}^{m}q_{j}(v)x^{j}\in \KK(v)[x].
\end{align*}
Now, by definition not all $a_{ij}$'s are zero, so not all $q_{j}(x)$'s are zero. That is, there exists some $0\leq k\leq m$ such that $q_{k}(x)\neq 0\in \KK[x]$. Well, since $v$ is transcendental over $\KK$, $q_{k}(x)\neq 0\implies q_{k}(v)\neq 0$; $v$ cannot be a zero of $q_{k}(x)$ since it's non-zero over $\KK$. Therefore, $Q(x)=\sum_{j=0}^{m}q_{j}(v)x^{j}\neq 0\in \KK(v)[x]$ and since $Q(u)=0$, $u$ must be algebraic over $\KK(v)$.

Thus,
\begin{center}
if $\KK\supseteq \FF$ is a field extension, $u,v\in \FF$, $v$ is algebraic over $\KK(u)$, and $v$ is transcendental over $\KK,\newline$ then $u$ is algebraic 
over $\KK(v)$.
\end{center}
\end{proof}
\newpage
%%%%%%%%%%%%%%%%%%%%%%%%%%%%%%%%%%% 46 %%%%%%%%%%%%%%%%%%%%%%%%%%%%%%%%%%%%%%
\begin{prob}
If $\KK\supseteq \FF$ is a field extension and $u\in \FF$ is algebraic of odd degree over $\KK$, then so is $u^{2}$ and $\KK(u)=\KK(u^{2})$. 
\end{prob}
\begin{proof}
Since $u$ is algebraic of odd degree $n=2k+1$ over $\KK$ for some $k\in \ZZ^{+}$, 
  $$\KK[x]/\langle P(x)\rangle\cong \Span\{1,x,\hdots,x^{n-1}\}\cong\Span\{1,u,\hdots, u^{n-1}\}=\KK(u)$$
for some monic irreducible degree $n$ polynomial $P(x)$ over $\KK$ such that $p(u)=0$. (This isomorphism is the canonical one $[f(x)]\longleftrightarrow f(u)$ where $[f(x)]=[g(x)]\longleftrightarrow f(u)=g(u)$. Therefore, $[0]=[P(x)]\longleftrightarrow 0\implies P(u)=0\in \KK(u)$. This is also just given since the extension is defined by that relation but whatever.) Obviously, $u^{2}\in \KK(u)$, so any $q(u^{2})$ belongs to $\KK(u)$ and therefore any $\frac{f(u^{2})}{g(u^{2})}\in \KK(u^{2})$ also belongs to $\KK(u)$. So $\KK(u^{2})\subseteq \KK(u)$. Additionally, $u^{2}$ must be algebraic otherwise $\KK(u^{2})\cong K(x)\supset K[x]=\Span\{x^{m}\mid m\in \NN\}$ is infinite dimensional and so $\KK(u^{2})\subseteq \KK(u^{2})$ implies that $\KK(u)$ is infinite dimensional, a contradiction. Next, $P(x)=\sum_{i=0}^{n}a_{i}x^{i}=\sum_{i=0}^{2k+1}a_{i}x^{i}=\sum_{i=0}^{k}a_{2i+1}x^{2i+1} + \sum_{i=0}^{k}a_{2i}x^{2i}=x\sum_{i=0}^{k}a_{2i+1}x^{2i} + \sum_{i=0}^{k}a_{2i}x^{2i}$ for some $a_{0},\hdots, a_{2k+1}\in \KK$. So then $P(u)=u\sum_{i=0}^{k}a_{2i+1}u^{2i} + \sum_{i=0}^{k}a_{2i}u^{2i}=0$. Since $\sum_{i=0}^{k}a_{2i+1}u^{2i}$ has degree $2k<n$, $u$ can't be a zero of it since it is degree $n$ over $\KK$. Therefore,
$$u=\frac{-\sum_{i=0}^{k}a_{2i}u^{2i}}{\sum_{i=0}^{k}a_{2i+1}u^{2i}}=\frac{-\sum_{i=0}^{k}a_{2i}(u^{2})^{i}}{\sum_{i=0}^{k}a_{2i+1}(u^{2})^{i}}\in \left\{\frac{f(u^{2})}{g(u^{2})}\mid f(x),g(x)\in \KK[x]\text{ and }g(u^{2})\neq 0\right\}=\KK(u^{2}).$$
So then any $f(u)\in \KK(u)$ must also belong to $\KK(u^{2})$ and $\KK(u)\supseteq \KK(u^{2})$.

Thus,
\begin{center}
if $\KK\subseteq \FF$ is a field extension and $u\in \FF$ is algebraic of odd degree over $\KK$, then so is $u^{2}$ and $\KK(u)=\KK(u^{2})$.
\end{center}


\end{proof}
\newpage
%%%%%%%%%%%%%%%%%%%%%%%%%%%%%%%%%%% 47 %%%%%%%%%%%%%%%%%%%%%%%%%%%%%%%%%%%%%%
\begin{prob}
Let $\KK \supseteq \FF$ be a field extension. If $X^n - a \in \KK[X]$ is irreducible and $u \in \FF$ is a root of $X^n - a$ and $m$ divides $n$, then the degree of $u^m$ over $\KK$ is $n/m$. What is the irreducible polynomial of $u^m$ over $\KK$?.
\end{prob}
\begin{proof}
Since $u^{n}-a=0$ and $m\mid n,\, n=mk$ for some $k\in \ZZ^{+}$ and so $u^{mk}-a=0\implies (u^{m})^{k}-a=0$, so $u^{m}$ is a zero of $x^{k}-a\in \KK[x]$. Now, suppose $x^{k}-a$ is reducible over $\KK$. Then $x^{k}-a=f(x)g(x)$
for some non-constant polynomials $f(x),g(x)\in \KK[x]$ such that $\deg(f(x))=\alpha,\,\deg(g(x))=\beta$ and $\alpha+\beta=k$. So we get that $x^{n}-a=(x^{m})^{k}-a=f(x^{m})g(x^{m})$. 

(The composition $(a\circ b)(x)=a(b(x))$ over a field $\FF$ can only $(1)$ multiply $b(x)$ by itself some finite number of times and/or $(2)$ scale $b(x)$ via $\FF$ and/or $(3)$ add scalars in $\FF$ to $b(x)$ all of which preserve structure.)

So $f(x^{m})$ and $g(x^{m})$ are polynomials of degree $m\alpha>1,\text{and }m\beta>1,$ respectively. But then $x^{n}-a$ is reducible over $\KK$, a contradiction. So $x^{k}-a$ must be irreducible over $\KK$ and $u^{m}$ is a zero of it.

Thus,
\begin{center}
if $\KK \supseteq \FF$ is a field extension, $X^n - a \in \KK[X]$ is irreducible, $u \in \FF$ is a root of $X^n - a$, and $m$ divides $n$, then the degree of $u^m$ over $\KK$ is $n/m$ and $x^{k}-a$ is the irreducible polynomial of $u^{m}$ over $\KK$.
\end{center}
\end{proof}
\newpage
%%%%%%%%%%%%%%%%%%%%%%%%%%%%%%%%%%% 48 %%%%%%%%%%%%%%%%%%%%%%%%%%%%%%%%%%%%%%
\begin{prob}
Let $\KK \supseteq R \supseteq \FF$ be an extension of rings with $\KK,\FF$ fields. If $\KK \supseteq \FF$ is algebraic, prove that $R$ is a field.
\end{prob}
\begin{proof}
Since $\KK$ is algebraic over $\FF$, $\forall \alpha\in \KK$ there exists a minimal non-zero polynomial $f(\alpha)$ over $\FF$ such that $f(\alpha)=0$. Therefore, since $R\subseteq \KK$, it must also be algebraic over $\FF$. If $\KK=R=\FF=\{0\}$, they're... arguably fields but then $\KK$ can't be algebraic over $\FF$ since there are no non-zero polynomials over $\FF$. So $R\neq \{0\}$ and it contains some non-zero element $r\in R\subseteq \KK$. It has a multiplicative inverse $r^{-1}$ in  $\KK$ and there exists some minimal degree $n\in \ZZ^{+}$ polynomial $P(x)=\sum_{i=0}^{n}a_{i}x^{i}\in \FF[x]$ such that $P(r)=0$. Note that since it's irreducible and $r\neq 0$, the constant term is non-zero, otherwise it's reducible: $\sum_{i=0}^{n}a_{i}x^{i}=\sum_{i=1}^{n}a_{i}x^{i}=x(\sum_{i=0}^{n}a_{i}x^{i-1})$. Observe.
\begin{align*}
  P(r)&=\sum_{i=0}^{n}a_{i}r^{i}=0\implies r^{-1}(\sum_{i=0}^{n}a_{i}r^{i})=0\implies r^{-1}a_{0}+\sum_{i=0}^{n}a_{i}r^{i-1}=0\\
  \implies r^{-1}a_{0}&=-\sum_{i=0}^{n}a_{i}r^{i-1}\implies r^{-1}=-\frac{1}{a_{0}}\sum_{i=0}^{n}a_{i}r^{i-1}\in R.
\end{align*}
This holds because $\FF\subseteq R$. So then every $r\in R$ has a multiplicative inverse $r^{-1}$ in $R$. So then since $R$ has multiplicative inverses, it has unity by closure, and it is commutative with no zero divisors via $R\subseteq \KK$, $R$ is a field.

Thus,
\begin{center}
if $\KK \supseteq R \supseteq \FF$ is an extension of rings where $\KK\text{ and }\FF$ are fields, and $\KK \supseteq \FF$ is algebraic, then $R$ is a field.
\end{center}
\end{proof}
\newpage
%%%%%%%%%%%%%%%%%%%%%%%%%%%%%%%%%%% 49 %%%%%%%%%%%%%%%%%%%%%%%%%%%%%%%%%%%%%%
\begin{prob}
Let $f = X^3 - 6X^2 + 9X + 3 \in \mathbb{Q}[X]$.
\begin{enumerate}[label=(\alph*)]
\item Prove that $f$ is irreducible in $\mathbb{Q}[X]$.
\item Let $u$ be a real root of $f$. Consider the extension $\mathbb{Q} \subseteq \mathbb{Q}(u)$. Express each of the following elements in terms of the basis $\{1,u,u^2\}$ of the $\mathbb{Q}$-vector space $\mathbb{Q}(u)$:
\[
u^4,\quad u^5,\quad 3u^5 - u^4 + 2,\quad (u+1)^{-1},\quad (u^2 - 6u + 8)^{-1}.
\]
\end{enumerate}
\end{prob}
\begin{proof}
\textbf{(a)} $3$ is prime and divides all integer coefficients of $f(x)=x^3 - 6x^2 + 9x + 3\in \QQ[x]$ except the leading one, and $3^{2}\nmid 3$, the constant term of $f(x)$, so by \textbf{Eisenstein's Criterion} $f(x)$ is irreducible over $\QQ.\newline$

\textbf{(b)} Since $f(x)=x^3 - 6x^2 + 9x + 3$ is monic and irreducible over $\QQ$ and $u$ is a zero of it we have 
$$\QQ[x]/\langle x^3 - 6x^2 + 9x + 3\rangle \cong \QQ(u)=\Span\{1,u,u^{2}\}\text{ and }u^3 - 6u^2 + 9u + 3=0.$$
So then $u^{3}=6u^{2}-9u-3$. Observe.
\begin{align*}
  &u^{4}=u(u^{3})=u(6u^{2}-9u-3)=6u^{3}-9u^{2}-3u=6(6u^{2}-9u-3)-9u^{2}-3u=27u^{2}-57u-18.\\
  &u^{5}=u(u^{4})=u(27u^{2}-57u-18)=27u^{3}-57u^{2}-18u=27(6u^{2}-9u-3)-57u^{2}-18u.\\
  &\ \ \ \ \ \! \! \! =105u^{2}-261u-81.\\
  &3u^5 - u^4 + 2=3(105u^{2}-261u-81)-(27u^{2}-57u-18)+2=288u^{2}-726u-223.\\
\end{align*}
Next, we use long division to factor $f(u)$ into a multiple of $(u+1)$ and $(u^2 - 6u + 8)$ so we can solve for the  using the remainder. We could have solved a system but this is easier. \newpage
\polylongdiv[style=D]{x^3-6x^2+9x+3}{x+1}\newline
\;\;So $\frac{f(u)}{u+1}=u^{2}-7u+16-\frac{13}{u+1}\implies f(u)=0=(u^{2}-7u+16)(u+1)-13\implies \frac{1}{13}(u^{2}-7u+16)(u+1)=1.$ So $(u+1)^{-1}=\frac{1}{13}(u^{2}-7u+16).\newline\text{Next we solve for }(u^2-6u+8)^{-1}.\newline$ Division didn't work so we just solve a system directly. 
$(u^2-6u+8)^{-1}=au^{2}+bu+c$ for some $a,b,c\in \QQ$. So $(u^2-6u+8)(u^2-6u+8)^{-1}=(u^2-6u+8)(au^{2}+bu+c)=a u^4 + (b - 6a)u^3 + (c - 6b + 8a)u^2 + (-6c + 8b)u + 8c=(c - a)u^2 + (-3a - b - 6c)u + (-3b + 8c)=0u^{2}+0u+1.$
\[
\begin{pmatrix}
-1 & 0 & 1 & 0\\
-3 & -1 & -6 & 0\\
0 & -3 & 8 & 1
\end{pmatrix}
\;\longrightarrow\;
\begin{pmatrix}
1 & 0 & 0 & \tfrac{1}{35}\\
0 & 1 & 0 & -\tfrac{9}{35}\\
0 & 0 & 1 & \tfrac{1}{35}
\end{pmatrix}.
\]
\[
\implies a=\tfrac{1}{35},\quad b=-\tfrac{9}{35},\quad c=\tfrac{1}{35}.
\]
So, $(u^2-6u+8)^{-1}=\tfrac{1}{35}(u^2-9u+1).$

\end{proof}
\newpage
%%%%%%%%%%%%%%%%%%%%%%%%%%%%%%%%%%% 50 %%%%%%%%%%%%%%%%%%%%%%%%%%%%%%%%%%%%%%
\begin{prob}
Let $F = \mathbb{Q}(\sqrt{2}, \sqrt{3})$. Find $[F:\mathbb{Q}]$ and a basis of $\FF$ over $\mathbb{Q}$.
\end{prob}

\begin{proof}
To begin, $\sqrt{2}$ and $\sqrt{3}$ are zeros of monic irreducible polynomials $x^{2}-2$ and $x^{2}-3$, respectively, over $\QQ$. So $\QQ(\sqrt{2})\cong \QQ[x]/\langle x^{2}-2\rangle\cong (\mathrm{Span}_{\QQ}\{1,x\}\subseteq \QQ[x]) \cong \QQ[x]/\langle x^{2}-3\rangle \cong \QQ(\sqrt{3})$. So then $\QQ(\sqrt{2})=\mathrm{Span}\{1,\sqrt{2}\}$ and $\QQ(\sqrt{3})=\mathrm{Span}\{1,\sqrt{3}\}$. Observe. 
\begin{align*}
  &\sqrt{3}=a+b\sqrt{2}\text{ for some }a,b\in \QQ \implies 3=(a+b\sqrt{2})^{2}=(a^{2}+(2ab)\sqrt{2}+2b^{2})\not\in \QQ,\\
  &\sqrt{2}=a+b\sqrt{3}\text{ for some }a,b\in \QQ\implies 2=(a+b\sqrt{3})^{2}=(a^{2}+(2ab)\sqrt{3}+3b^{2})\not\in \QQ,\\
  &\sqrt{6}=a+b\sqrt{2}\text{ for some }a,b\in \QQ\implies 6=(a+b\sqrt{2})^{2}=(a^{2}+(2ab)\sqrt{2}+2b^{2})\not\in \QQ,\\
  &\sqrt{6}=a+b\sqrt{3}\text{ for some }a,b\in \QQ\implies 6=(a+b\sqrt{3})^{2}=(a^{2}+(2ab)\sqrt{3}+3b^{2})\not\in \QQ.
\end{align*}
  All of the above are contradictions. So $1,\sqrt{2},\sqrt{3},\sqrt{6}$ must be linearly independent over $\QQ$. Next, $\QQ(\sqrt{2},\sqrt{3})=\mathrm{Span}_{\QQ(\sqrt{2})}\{1,\sqrt{3}\}=\{\alpha+\beta\sqrt{3}\mid \alpha,\beta\in \QQ(\sqrt{2})\}=\{(a+b\sqrt{2})+(c+d\sqrt{2})\sqrt{3}\mid a,b,c,d\in \QQ\}=\{a+b\sqrt{2}+c\sqrt{3}+d\sqrt{6}\mid a,b,c,d\in \QQ\}$. So $\{1,\sqrt{2},\sqrt{3},\sqrt{6}\}$ spans $\QQ(\sqrt{2},\sqrt{3})$ and since it's elements are linearly independent over $\QQ$, it must be a basis for $\QQ(\sqrt{2},\sqrt{3})$ over $\QQ$.

  Thus,
  \begin{center}
    $\{1,\sqrt{2},\sqrt{3},\sqrt{6}\}$ is a basis for $\QQ(\sqrt{2},\sqrt{3})$ over $\QQ$ and $[\QQ(\sqrt{2},\sqrt{3}):\QQ]=4.$
  \end{center}
\end{proof}

\newpage
%%%%%%%%%%%%%%%%%%%%%%%%%%%%%%%%%%% 51 %%%%%%%%%%%%%%%%%%%%%%%%%%%%%%%%%%%%%%
\begin{prob}
Let $\KK$ be a field. In the field $\KK(X)$, let $u=X^{3}/(X+1)$. What is $[\KK(X):\KK(u)]?$
\end{prob}

\begin{proof}
$(\KK(u))(x)=\left\{\frac{f(x)}{g(x)}\mid f,g\in \KK(u)[t]\right\}$ and then $u=\frac{x^{3}}{x+1}\implies u(x+1)-x^{3}=ux+u-x^{3}=0\implies x^{3}-ux-u=0$. So $x$ is a zero of the polynomial $t^{3}-ut-u$ over $\KK(u)$. This means that the degree of $x$ over $K(u)$, or equivalently, $[\KK(x):\KK(u)]$ must divide $3$. Therefore, $[\KK(x):\KK(u)]\in \{1,3\}$. Suppose $[\KK(x):\KK(u)]=1$, then $\KK(x)=\KK(u)$ and $x=\frac{f(u)}{g(u)}$ for some $f(u),g(u)\neq 0$ coprime over $\KK(u)$. Observe. 
\begin{center}
  $x^{3}-ux-u=(\frac{f(u)}{g(u)})^{3}-u(\frac{f(u)}{g(u)})-u=0$ and $f(u)^{3}-uf(u)g(u)^{2}-ug(u)^{3}=0$. So then $f(u)^{3}=uf(u)g(u)^{2}+ug(u)^{3}=ug(u)^{2}(f(u)+g(u))\newline\implies 3\mathrm{deg}(f(u))=1+2\mathrm{deg}(g(u))+\mathrm{max}\{\mathrm{deg}(f(u)),\mathrm{deg}(f(u))\}.$
\end{center}
Let $a=\mathrm{deg}(f(u)),b=\mathrm{deg}(g(u))$ and note that both belong to $\ZZ^{+}$. We get the following cases:
\begin{center}
  $\begin{cases} 3a=1+2b+a\\ \text{or} \\ 3a=1+2b+b\end{cases}\implies \begin{cases} 2a=1+2b\\ \text{or} \\ 3a=1+3b\end{cases}\implies \begin{cases} 2(a+b)=1\\ \text{or} \\ 3(a+b)=1\end{cases}\implies \begin{cases} (a+b)=\frac{1}{2}\\ \text{or} \\ (a+b)=\frac{1}{3}\end{cases}.$
\end{center}
Both of the above are contradictions. So $[\KK(x):\KK(u)]=3$.

\end{proof}
\newpage
%%%%%%%%%%%%%%%%%%%%%%%%%%%%%%%%%%% 52 %%%%%%%%%%%%%%%%%%%%%%%%%%%%%%%%%%%%%%
\begin{prob}
Let $\KK\subseteq \FF$ be a field extension. If $u,v\in \FF$ are algebraic over $\KK$ of degrees $m$ and $n$, respectively, then $[\KK(u,v): \KK]\leq mn.$ If $m$ and $n$ are relatively prime, then $[\KK(u,v):\KK]=mn.$
\end{prob}

\begin{proof}
$\KK(u)$ and $\KK(v)$ have bases $\BB_{u}=\{1,\hdots,u^{m-1}\}$ and $\BB_{v}=\{1,\hdots, v^{n-1}\}$, respectively, over $\KK$. Also, $\KK(u,v)=\Span_{\KK_{u}}\BB_{v}=\{\sum_{i=0}^{n-1}a_{i}u^{i}\mid a_{0},\hdots,a_{n-1}\in \KK(u)\}=\Span_{\KK}\,\BB_{u}\BB_{v}$. So $\BB_{u}\BB_{v}$ span $\KK(u,v)$ over $\KK$. Therefore, $[\KK(u,v):\KK]=|\BB_{m}\BB_{n}|\leq |\BB_{u}||\BB_{v}|=mn.$

Suppose $\mathrm{gcd}(m,n)=1$. Since $\KK(u,v)\supseteq \KK(u)\supseteq \KK$, by the Tower Law we have:
  $$[\KK(u,v):\KK]=[\KK(u,v):\KK(u)][\KK(u):\KK]=[\KK(u,v):\KK(v)][\KK(v):\KK].$$
Therefore, $[\KK(u):\KK]=m$ and $[\KK(v):\KK]=n$ both divide $[\KK(u,v):\KK]$, which means it is a multiple of both $m$ and $n$. Well, since $\mathrm{lcm}(m,n)=\frac{mn}{gcd(m,n)}=mn$ and $[\KK(u,v):\KK]\leq mn$, it must be the case that in fact $[\KK(u,v):\KK]=mn$.

\end{proof}
\newpage


\end{document}