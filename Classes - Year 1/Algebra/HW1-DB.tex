\documentclass[addpoints,10pt]{exam}

\usepackage{amsmath,amsthm,enumitem,wrapfig,amsfonts,mathtools}
\usepackage[mathscr]{euscript}
\usepackage[super]{nth}

\usepackage{geometry}
\usepackage[T1]{fontenc} % Use 8-bit encoding that has 256 glyphs
\renewcommand{\rmdefault}{ptm} %Change the Front Family from the default(cmr) to ptm(Times)
\usepackage{amsmath,amsfonts,amsthm,amssymb} % Math packages
\usepackage{bm}
\usepackage{mathptmx}
\usepackage{graphicx}
\usepackage{sectsty} % Allows customizing section commands
% \allsectionsfont{\centering} % Make all sections centered, the default font and small caps
\usepackage{pgfplots}
\pgfplotsset{compat=1.18}
\usetikzlibrary{arrows.meta}
\usepackage{xcolor}
\definecolor{darkpastelgreen}{rgb}{0.01, 0.75, 0.24}
\definecolor{blue-violet}{rgb}{0.54, 0.17, 0.89}

% Custom problem environment
\newcounter{cprob}
\newenvironment{cprob}[1]{%
    \setcounter{cprob}{#1}%
    \noindent\textbf{Problem \thecprob.}%
}{%
    \par\bigskip%
}

\theoremstyle{plain}
\newtheorem{thm}{\protect\theoremname}
  \theoremstyle{definition}
  \newtheorem{prob}[thm]{Problem}
  \newtheorem*{problem*}{Open Problem}
  \theoremstyle{plain}
  \newtheorem{conjecture}[thm]{Conjecture}
  \theoremstyle{plain}
  \newtheorem{lem}[thm]{Lemma}
  \newtheorem*{lem*}{Lemma}
  \newtheorem{obs}[thm]{Observation}
  \newtheorem{cor}[thm]{Corollary}
  \theoremstyle{definition}
\newtheorem{definition}[thm]{Definition}


% Patch prob environment to be single spaced
\let\oldprob\prob
\let\endoldprob\endprob
\renewenvironment{prob}
  {\begin{singlespace}\oldprob}
  {\endoldprob\end{singlespace}}

% start problem one line below like for enumerated problems with multiple parts
\newcommand{\belowtitle}{\leavevmode\newline}
%\Observe command
\newcommand{\Observe}{\text{Observe.}}
%(=>)
\newcommand{\IF}{\mathbf{(\Rightarrow)}}
%(<=)
\newcommand{\FI}{\mathbf{(\Leftarrow)}}

\newcommand{\horrule}[1]{\rule{\linewidth}{#1}}
\newcommand{\kk}{\ensuremath{\Bbbk}} 
\newcommand{\CC}{\ensuremath{\mathbb{C}}}
\newcommand{\FF}{\ensuremath{\mathbb{F}}}
\newcommand{\NN}{\ensuremath{\mathbb{N}}}
\newcommand{\QQ}{\ensuremath{\mathbb{Q}}} 
\newcommand{\RR}{\ensuremath{\mathbb{R}}} 
\newcommand{\ZZ}{\ensuremath{\mathbb{Z}}}
\newcommand{\MM}{\ensuremath{\mathcal{M}}}
\newcommand{\TT}{\ensuremath{\mathcal{T}}}
\newcommand{\BB}{\ensuremath{\mathcal{B}}}
\newcommand{\VV}{\ensuremath{\mathcal{V}}}
\newcommand{\WW}{\ensuremath{\mathcal{W}}}
\newcommand{\UU}{\ensuremath{\mathcal{U}}}
\newcommand{\PP}{\ensuremath{\mathcal{P}}}
\newcommand{\LL}{\ensuremath{\mathcal{L}}}

\newcommand{\sm}{\char`\\}
%vector stuff
\DeclarePairedDelimiter{\ip}{\langle}{\rangle} %inner product/generate
\DeclarePairedDelimiter{\norm}{\lVert}{\rVert} %norm
\DeclarePairedDelimiter{\sqb}{\lbrack}{\rbrack} %corrd
\newcommand{\floor}[1]{\left\lfloor #1 \right\rfloor}
\newcommand{\ceil}[1]{\left\lceil #1 \right\rceil}
\newcommand{\mbf}[1]{\ensuremath{\mathbf{#1}}}
\newcommand{\tbf}[1]{\textbf{ #1 }}


\makeatletter
\renewcommand*\env@matrix[1][*\c@MaxMatrixCols c]{%
  \hskip -\arraycolsep
  \let\@ifnextchar\new@ifnextchar
  \array{#1}}
\makeatother

\def\env@matrix{\hskip -\arraycolsep
  \let\@ifnextchar\new@ifnextchar
  \array{*\c@MaxMatrixCols c}}

  \newcommand{\proj}[2]{\text{proj}_{#1}(#2)}
  

%%% Formatting: Page Header
\newcommand{\StudentName}{Danny Banegas}
\newcommand{\AssignmentName}{Homework 1}
\newcommand{\CourseName}{MATH 721 - Algebra II}


\pagestyle{headandfoot}
\runningheadrule
\firstpageheadrule
\firstpageheader{\CourseName}{\StudentName}{\AssignmentName}
\runningheader{\CourseName}{\StudentName}{\AssignmentName}
\firstpagefooter{}{\thepage}{}
\runningfooter{}{\thepage}{}

\printanswers

\DeclareMathAlphabet{\mathcal}{OMS}{cmsy}{m}{n}

\usepackage{parskip}
\usepackage{setspace}
\doublespacing
% % % % % % % % % % % % % % % % % % % % % % % % % % % % % % % % % % % % % % % % % % % % % % % % % % % % % % % % % % % % % % 
\begin{document}

%%%%%%%%%%%%%%%%%%%%%%%%%%%%%%%%%%%%%%%%%%%%%%%%%%%%%%%%%% Problem 2 %%%%%%%%%%%%%%%%%%%%%%%%%%%%%%%%%%%%%%%%%%%%%%%%%%%%%%%%%
\setcounter{thm}{1}   % next prob is 2

\begin{prob}\belowtitle
  \begin{enumerate}[label=(\alph*)]
    \item Prove that the relation given by $a \sim b \iff a - b \in \mathbb{Z}$ is an equivalence relation on the additive group $\mathbb{Q}$.
    \item Prove that $\mathbb{Q}/\mathbb{Z}$ is an infinite abelian group.
  \end{enumerate}
\end{prob}

\begin{proof}\belowtitle
    (a)\quad For any $a,b,c\in (\QQ,+),$
    \begin{align*}
        &\mathbf{[a\sim a]:}\quad a-a=0\in \ZZ \implies a\sim a.\\
        &\mathbf{[a\sim b\implies b\sim a]:}\quad a\sim b\implies a-b\in \ZZ \implies -(a-b)=b-a\in \ZZ\implies  b\sim a.\\
        &\mathbf{[a\sim b,\,b\sim c\implies a\sim c]:}\quad a\sim b,\,b\sim c\implies c\sim b\implies (a-b)-(c-b)=a-c\in \ZZ\implies a\sim c.
    \end{align*}

    So $\sim$ is an equivalence relation on $(\QQ,+)$.

    (b)\quad $\QQ/\ZZ=\{[\frac{a}{b}]=\frac{a}{b}+\ZZ \mid a,b\in \ZZ\text{ and }b\nmid a\}$. Consider any $q_{1},q_{2}\in (0,1)$. If $[q_{1}]=[q_{2}],$ then $[q_{1}]-[q_{2}]=\ZZ$ and so $q_{1}-q_{1}\in \ZZ$. Well, $q_{1},q_{2}\in (0,1)$, so $q_{1}-q_{2}\in (-1,1)$ and therefore $q_{1}-q_{2}=0$. So $[q_{1}]=[q_{2}]\implies q_{1}=q_{2}$. On the other hand, $q_{1}=q_{2}\implies [q_{1}]=[q_{2}]$ by definition. So then
    $$q_{1}=q_{2}\iff [q_{1}]=[q_{2}],\;\forall q_{1},q_{2}\in (0,1).$$
    Since the rationals are dense in $\RR$, there are infinitely many distinct rationals in $(0,1)$ and infinitely many distinct cosets of the form $[q]$ where $q\in (0,1)$. Therefore, $\QQ/\ZZ$ is infinite. Lastly, since $(\QQ,+)$ is Abelian, so is $\QQ/\ZZ$ since $[q_{1}]+[q_{2}]=[q_{1}+q_{2}]=[q_{2}+q_{1}]=[q_{2}]+[q_{1}]$.

    Thus,
    \begin{center}
    $\QQ/\ZZ$ is an infinite Abelian group.
    \end{center}
\end{proof}
\newpage
%%%%%%%%%%%%%%%%%%%%%%%%%%%%%%%%%%%%%%%%%%%%%%%%%%%%%%%%%% Problem 3 %%%%%%%%%%%%%%%%%%%%%%%%%%%%%%%%%%%%%%%%%%%%%%%%%%%%%%%%%

\begin{prob}
  Let $p$ be a prime number and let $Z(p^\infty)$ be the following subset of the group $\mathbb{Q}/\mathbb{Z}$:
  \[
  \ZZ(p^\infty)=\left\{\, \frac{a}{b} \in \mathbb{Q}/\mathbb{Z} \;\middle|\; a,b\in\mathbb{Z},\ b=p^i \text{ for some } i \ge 0 \right\}.
  \]
  Prove that $\ZZ(p^\infty)$ is an infinite subgroup of $\mathbb{Q}/\mathbb{Z}$.
\end{prob}

\begin{proof}
    Clearly, $\ZZ(p^{\infty})\subset \QQ/\ZZ$. Consider some integers $i,j\geq 0$ and $a_{i},a_{j}\in \ZZ$. 
    \begin{align*}
      &\textbf{[Closure]:}\quad[\frac{a_{i}}{p^{i}}]+[\frac{a_{j}}{p^{i}}]=[\frac{p^{j}(a_{i})+p^{i}(a_{j})}{p^{i+j}}]\in \ZZ(p^{\infty}).\\
      &\textbf{[Inverses]:}\quad[\frac{-a_{i}}{p^{i}}]+[\frac{a_{i}}{p^{i}}]=[0]\implies -[\frac{a_{i}}{p^{i}}]=[\frac{-a_{i}}{p^{i}}].
    \end{align*}

    So $\ZZ(p^{\infty})\leq \QQ/\ZZ$. Now once more consider some integers $i,j\in \ZZ^{+}$ but set $a=1$. Notice that $\frac{1}{p^{i}},\frac{1}{p^{j}}\in (0,1)$. \underline{\textbf{Observe.}}

    This result essentially follows from \textbf{Problem 2}. $[\frac{1}{p^{i}}]=[\frac{1}{p^{j}}]\implies [\frac{1}{p^{i}}]-[\frac{1}{p^{j}}]=\ZZ\implies \frac{1}{p^{i}}-\frac{1}{p^{j}}\in \ZZ$. Well, $\frac{1}{p^{i}},\frac{1}{p^{j}}\in (0,1)\implies \frac{1}{p^{i}}-\frac{1}{p^{j}}\in (-1,1)\implies \frac{1}{p^{i}}-\frac{1}{p^{j}}=0\implies \frac{1}{p^{i}}=\frac{1}{p^{i}}\implies i=j$. On the other hand, $i=j\implies \frac{1}{p^{i}}=\frac{1}{p^{j}}\implies [\frac{1}{p^{i}}]=[\frac{1}{p^{j}}]$ by definition. So then,
    $$i=j\iff [\frac{1}{p^{i}}]=[\frac{1}{p^{j}}],\;\forall i,j\in \ZZ^{+}.$$
    There are infinitely many distinct positive integers so there must be infinitely many distinct cosets in $\ZZ(p^{\infty})$.

    Thus,
    \begin{center}
    $\ZZ(p^{\infty})$ is an infinite subgroup of $\QQ/\ZZ$.
    \end{center}
\end{proof}
\newpage
%%%%%%%%%%%%%%%%%%%%%%%%%%%%%%%%%%%%%%%%%%%%%%%%%%%%%%%%%% Problem 5 %%%%%%%%%%%%%%%%%%%%%%%%%%%%%%%%%%%%%%%%%%%%%%%%%%%%%%%%%
\setcounter{thm}{4}   % next prob is 5

\begin{prob}
  Let $Q_8$ be the multiplicative group generated by the complex matrices
  \[
  A=\begin{pmatrix}0 & 1 \\ -1 & 0\end{pmatrix},
  \qquad
  B=\begin{pmatrix}0 & i \\ i & 0\end{pmatrix}.
  \]
  Observe that $A^4=B^4=I_2$ and $BA=AB^3$. Prove that $Q_8$ is a group of order $8$.
\end{prob}

\begin{proof}
Well, 
\end{proof}
\newpage
%%%%%%%%%%%%%%%%%%%%%%%%%%%%%%%%%%%%%%%%%%%%%%%%%%%%%%%%%% Problem 6 %%%%%%%%%%%%%%%%%%%%%%%%%%%%%%%%%%%%%%%%%%%%%%%%%%%%%%%%%

\begin{prob}
  Let $G$ be a group and let $\operatorname{Aut}(G)$ denote the set of all automorphisms of $G$.
  \begin{enumerate}[label=(\alph*)]
  \item Prove that $\operatorname{Aut}(G)$ is a group with composition of functions as the binary operation.
  \item Prove that $\operatorname{Aut}(\mathbb{Z}) \cong \mathbb{Z}_2,\ 
  \operatorname{Aut}(\mathbb{Z}_6)\cong \mathbb{Z}_2,\
  \operatorname{Aut}(\mathbb{Z}_8)\cong \mathbb{Z}_2 \times \mathbb{Z}_2,\
  \operatorname{Aut}(\mathbb{Z}_p)\cong \mathbb{Z}_{p-1}\ (p\ \text{prime})$.
\end{enumerate}
\end{prob}

\newpage
%%%%%%%%%%%%%%%%%%%%%%%%%%%%%%%%%%%%%%%%%%%%%%%%%%%%%%%%%% Problem 8 %%%%%%%%%%%%%%%%%%%%%%%%%%%%%%%%%%%%%%%%%%%%%%%%%%%%%%%%%
\setcounter{thm}{7}   % next prob is 8

\begin{prob}
  Let $G$ be the multiplicative group of $2\times 2$ invertible matrices with rational entries. Show that
  \[
  A=\begin{pmatrix}0 & -1 \\ 1 & 0\end{pmatrix},
  \qquad
  B=\begin{pmatrix}0 & 1 \\ -1 & -1\end{pmatrix}
  \]
  have finite orders but $AB$ has infinite order.
\end{prob}

\newpage
%%%%%%%%%%%%%%%%%%%%%%%%%%%%%%%%%%%%%%%%%%%%%%%%%%%%%%%%%% Problem 10 %%%%%%%%%%%%%%%%%%%%%%%%%%%%%%%%%%%%%%%%%%%%%%%%%%%%%%%%
\setcounter{thm}{9}   % next prob is 10

\begin{prob}
  Let $H,K$ be subgroups of a group $G$. Prove that $HK$ is a subgroup of $G$ if and only if $HK=KH$.
\end{prob}

\begin{proof}\belowtitle
  $\IF\;HK\leq G\implies\text{For all }hk\in HK$, $(hk)^{-1}=k^{-1}h^{-1}\in HK$. Therefore, $HK=\{hk\mid h\in H, k\in K\}=\{k^{-1}h^{-1}\mid k\in K,h\in H\}=KH$.

  $\FI$ Note $HK=KH\implies \forall hk\in HK,\,\exists (h_{k_{1}},k_{h_{1}})\in H\times K, \text{ such that }hk=k_{h_{1}}h_{k_{1}}\in KH=HK$. The same logic holds for 'flipped' elements$\;\;kh\in KH=HK$. \Observe
  \begin{align*}
    &\textbf{[Closure]: } (h_{1}k_{1})(h_{2}k_{2})=(h_{1}k_{1})(k_{h_{2}}h_{k_{2}})=h_{1}(k_{1}k_{h_{2}})h_{k_{2}}=(k_{1}k_{h_{2}})_{h_{1}}h_{k_{1}k_{h_{2}}}h_{k_{2}}\in KH=HK.\\
    &\textbf{[Inverses]: }\text{For any }hk\in HK,\,(hk)^{-1}=k^{-1}h^{-1}\in KH=HK.
  \end{align*}
  So $HK\leq G$.

  Thus,
  $$HK\leq G\iff HK=KH.$$
\end{proof}
\newpage
%%%%%%%%%%%%%%%%%%%%%%%%%%%%%%%%%%%%%%%%%%%%%%%%%%%%%%%%%% Problem 11 %%%%%%%%%%%%%%%%%%%%%%%%%%%%%%%%%%%%%%%%%%%%%%%%%%%%%%%%

\begin{prob}
  Let $H,K$ be subgroups of finite index of a group $G$ such that $[G:H]$ and $[G:K]$ are relatively prime. Prove that $G=HK$.
\end{prob}

\begin{proof}
  We begin by proving $(H\cap K)\leq H,K\leq G$.\belowtitle
  $\textbf{[1-Step]:\;}\forall a,b\in (H\cap K)\implies ab^{-1}\in H\text{ and }ab^{-1}\in K\implies ab^{-1}\in (H\cap K)\implies (H\cap K)\leq H,K\leq G$.

  Since $(H\cap K)\leq H,K \leq G$, by the Tower Law for groups,
  \begin{center}
    $[G:(H\cap K)]=[G:H][H:H\cap K]=[G:K][K:H\cap K]\implies [K:H\cap K]=\frac{[G:H][H:H\cap K]}{[G:K]}$\newline
    and $\gcd([G:H],[G:K])=1\implies [G:K]\mid [H:H\cap K]$.
  \end{center}

  Now consider $H_{K}=\{hK\mid h\in H\}\subseteq G/K$. $h_{1}K=h_{2}K\implies h_{2}^{-1}h_{1}\in K\implies h_{2}^{-1}h_{1}\in (H\cap K).$ Well, $h_{1}(H\cap K)=h_{2}(H\cap K)\implies h_{2}^{-1}h_{1}\in (H\cap K)$. So then we see that $hK\in [h_{1}]_{K}\iff h(H\cap K)\in [h_{1}]_{(H\cap K)},\,\forall h\in H.$ Therefore, $[h]_{K}\leftrightarrow [h]_{(H\cap K)}$ is clearly a bijection from $H_{K}$ to $H/(H\cap K)$. \Observe

  $(H_{K}\subseteq G/K)\iff (|H_{k}|\leq [G:K])$ and then $(|H_{k}|\leq [G:K])$ with $([G:K]\mid [H:H\cap K]=|H_{K}|)\implies |H_{K}|=[G:K]$ and so $H_{k}\not\subset G/K\implies H_{K}=\{hK\mid h\in H\}=G/K.$ Therefore, $\forall g\in G,\exists h_{g}\in H$ such that $gK=h_{g}K$. Finally, $h_{g}^{-1}g\in K\implies \exists k_{g}\in K$ such that $h_{g}^{-1}g=k_{g}\implies g=h_{g}k_{g}$. So we see that $\forall g\in G,\,\exists (h_{g},k_{g})\in H\times K$ such that $g=h_{g}k_{g}$.

  Thus,

  $$H,K\leq G\text{ and }\gcd([G:H],[G:K])=1\implies G=HK.$$

\end{proof}
\newpage
%%%%%%%%%%%%%%%%%%%%%%%%%%%%%%%%%%%%%%%%%%%%%%%%%%%%%%%%%% Problem 12 %%%%%%%%%%%%%%%%%%%%%%%%%%%%%%%%%%%%%%%%%%%%%%%%%%%%%%%%

\begin{prob}
  Let $H,K,N$ be subgroups of $G$ such that $H\subseteq N$. Prove that $HK\cap N = H(K\cap N)$.
\end{prob}

\begin{proof}
  Notice that since $H\subseteq N,\, HN=N$. We show $H(K\cap N)=HK\cap HN=HK\cap N.$\belowtitle
  $\mathbf{[\subseteq]:\;} \forall a\in H(K\cap N),\,a=hg$ where $h\in H\text{ and }g\in (K\cap N).$ Well, $g\in K\implies a=hg\in HK.$ Similarly, $g\in N \implies a=hg\in HN.$ Therefore, $a\in HK\cap HN\implies H(K\cap N)\subseteq (HK\cap HN)=(HK\cap N)$.

  $\mathbf{[\supseteq]:\;} \forall a\in HK\cap HN,\;a=hg$ where $hg\in HK$ and $hg\in HN.$ So then $g\in K$ and $g\in N$ and we have $a=hg$ where $h\in H$ and $g\in K\cap N$. Therefore, $a\in H(K\cap N)\implies H(K\cap N)\subseteq HK\cap HN=HK\cap N$.

  Thus,
  $$H,K,N\leq G\text{ and }H\subseteq N\implies HK\cap N=HK\cap HN=H(K\cap N).$$

\end{proof}
\newpage
%%%%%%%%%%%%%%%%%%%%%%%%%%%%%%%%%%%%%%%%%%%%%%%%%%%%%%%% Problem 13 %%%%%%%%%%%%%%%%%%%%%%%%%%%%%%%%%%%%%%%%%%%%%%%%%%%%%%%%%%

\begin{prob}
  Let $H,K,N$ be subgroups of $G$ such that $H\subseteq K,\ H\cap N=K\cap N,\ HN=KN$. Prove that $H=K$.
\end{prob}

\begin{proof}
  $H\subseteq K$ is given. We show $K\subseteq H$ to prove the statement.\belowtitle
  $\mathbf{[\supseteq]:\;} \forall k\in K,\,\exists h_{k}\in H$ such that $kN=h_{k}N$ and so $h_{k}^{-1}k\in N.$ Well, $h_{k}^{-1}\in H\subseteq K$ and so by closure $h_{k}^{-1}k\in K\implies h_{k}^{-1}k\in (K\cap N)=(H\cap N).$ Finally, $h_{k}^{-1}k\in H$ and so $\exists h_{*}\in H$ such that $h_{k}k=h_{*}\implies k=h_{k}h_{*}\in H.$ Therefore, $K\subseteq H.$

  Thus,
  $$H,K,N\leq G\text{ and }H\subseteq K,\,H\cap N=K\cap N,\, HN=KN\implies H=K.$$
\end{proof}
\newpage
%%%%%%%%%%%%%%%%%%%%%%%%%%%%%%%%%%%%%%%%%%%%%%%%%%% lemma %%%%%%%%%%%%%%%%%%%%%%%%%%%%%%%%%%%%%%%%%%%%%%%%%%%%%%%%%%%%%%%%%%%%
We prove the following lemma to be used for \textbf{Problem 16.}

\begin{lem*}
  Any subgroup $H$ of a cyclic group $G$ is cyclic, and if $G$ has order $N\in \ZZ^{+}$ there exists exactly one subgroup $H_{d}\leq G$ of order $d$ for each divisor $d$ of $|G|=N$.
\end{lem*}

\begin{proof}
  If $H=\{e\}$ it is cyclic. If $H$ is non-trivial, then it contains some $h\neq e.$ Well, since $h\in H\leq G$, $h=g^{k}$ for some $k\in\ZZ^{+}$. So then there exists some minimal non-trivial power $n=\min \{i\in \ZZ^{+}\mid g^{i}\in H\setminus \{e\}\}$ of $g$ present in $H\setminus\{e\}$. Observe.

  By the division algorithm, $\forall m\in \{i\in \ZZ^{+}\mid g^{m}=H\setminus\{e\}\}$, there exist unique integers $q,r$ with $0\leq r<n$ such that
  $$m=nq+r\implies g^{m}=g^{nq+r}=g^{nq}g^{r}\implies g^{m-nq}=g^{r}\in H.$$
  But since $n$ is the minimal power of $g$ in $H$, $r=0$ otherwise we get a contradiction via $0<r<n$. So then for any $m\in \ZZ^{+},\text{ such that }g^{m}\in H,\;g^{m}=g^{nq_{m}}=(g^{n})^{q_{m}}$ for some $q_{m}\in \NN$. Therefore, $H=\langle g^{n}\rangle$, a cyclic group.

  Next, if $G$ is finite and of order $N$, consider any divisor $d$ of $|G|=N$. Since $G=\langle g\rangle$, $|g|=N$. Well, since $d|N,\,\exists!q\in \ZZ^{+}\text{ such that }dq=N$. So we see $g^{dq}=g^{N}\implies (g^{q})^{d}=e$. Such a $d$ is necessarily a minimal power that gives identity here since $0<q,d$ and otherwise $N=d'q<dq=N$, which is nonsense. So $|g^{q}|=d.$ So then there is only one power $q$ of $g$ that has order $|g^{q}|=d$ (otherwise the existence of $q'\neq q\text{ such that }|g^{q'}|=d\implies N=q'd\neq qd=N$... nonsense.) Since any $d-$ordered subgroup $H_{d}$ of $G$ is cyclic, it must be generated by some power of $G$, of which there is only one and so $H_{d}=\langle g^{d}\rangle$ is the only subgroup of order $d|N.$
   
\end{proof}
Now we present the solution to $16$ on the following page.
\newpage
%%%%%%%%%%%%%%%%%%%%%%%%%%%%%%%%%%%%%%%%%%%%%%%%%%%%%%%% Problem 16 %%%%%%%%%%%%%%%%%%%%%%%%%%%%%%%%%%%%%%%%%%%%%%%%%%%%%%%%%%
\setcounter{thm}{15}   % next prob is 16

\begin{prob}
  If $H$ is a cyclic normal subgroup of a group $G$, then every subgroup of $H$ is normal in $G$.
\end{prob}
\begin{proof}
  Suppose $|H|=n$. Since $K\leq H=\langle h\rangle$ where $|h|=n$, $K$ is cyclic by our lemma and there exists some minimal positive power $d \in \ZZ^{+}$ of $h$ such that $K=\langle h^{d}\rangle.$ So any $k\in K$ is of the form $k=(h^{d})^{q}$ for some minimal power $q\in \ZZ^{+}$. Since $H\trianglelefteq G$, 
  $$\forall g\in G,\,gHg^{-1}=H\iff \forall (g,h^{q})\in G\times H,\,\exists h^{p}\in H,\text{ such that }gh^{q}g^{-1}=h^{p}.\text{ for any powers }p,q\in \ZZ^{+}$$
  Observe.
    $$(gh^{q}g^{-1})^{m}=\overbrace{(gh^{q}g^{-1})(gh^{q}g^{-1})\cdots (gh^{q}g^{-1})}^{m}=\overbrace{g(h^{2q}g^{-1})(gh^{q}g^{-1})\cdots (gh^{q}g^{-1})}^{m-1}=\cdots=gh^{mq}g^{-1}=h^{mp}.$$
  So then for any $k\in K=\langle h^{d}\rangle,\text{ where }k=(h^{d})^{q}=(h^{q})^{d}$ and any $g\in G$, $\exists h^{p}\in H$ such that
    $$gh^{q}g^{-1}=h^{p}\implies gkg^{-1}=g(h^{q})^{d}g^{-1}=(g(h^{q})g^{-1})^{d}=(h^{p})^{d}=(h^{d})^{p}\in \langle h^{d}\rangle=K.$$
  Note that since $gkg^{-1}=h^{dp}\text{ implies }gk=h^{dp}g$, there is only one power $(h^{d})^{p}\in K$ for which the equality holds otherwise we get a contradiction. So for each $k_{l}\in K,\,\exists! k_{r}\in K$ such that $gk_{l}g^{-1}=k_{r}$. To avoid further nightmare indexing, note that we are taking the union of all conjugates $gk_{l}g^{-1}\in gKg$ on the left side and showing that since each conjugate is paired with some unique $k_{r}\in K$ on the right side. The union of all conjugates $gk_{l}g^{-1}$ is equal to the union of all their unique partners $k_{r}$ and since there are $|K|$ conjugates and $|K|$ unique partners, of course the right side must be all of $K$.
  $$\bigcup_{k_{l}\in K}gk_{l}g^{-1}=gKg^{-1}=\bigcup_{gk_{l}g^{-1}=\,k_{r}\in K} k_{r}=K.$$
  Thus,
  \begin{center}
    $K\leq H=\langle h\rangle\trianglelefteq G\implies K\trianglelefteq G.$
  \end{center}

\end{proof}
\newpage
%%%%%%%%%%%%%%%%%%%%%%%%%%%%%%%%%%%%%% Choose 2 %%%%%%%%%%%%%%%%%%%%%%%%%%%%%%%%%%%
\setcounter{thm}{20}   % next prob is 20

\begin{prob}
If $G$ is a finite group and $H,K$ are subgroups of $G$, then
\[
[G:H\cap K] \le [G:H][G:K].
\]
\end{prob}
\begin{proof}
Since $G$ is finite and $H,K\leq G$, we have the following

\begin{align}
&|HK|=\frac{|H||K|}{|H\cap K|}\leq |G|\\
&[G:H]=\frac{|G|}{|H|}\\
&[G:K]=\frac{|G|}{|K|}\\
&[G:H\cap K]=\frac{|G|}{|H\cap K|}
\end{align}
Observe.
\begin{center}
$|HK|=\frac{|H||K|}{|H|}\leq |G|\implies (|G|)\frac{|H||K|}{|H\cap K|}\leq |G|^{2}\implies (\frac{|G|}{|H||K|})\frac{|H||K|}{|H\cap K|}=\frac{|G|}{|H\cap K|}=[G:K]\leq \frac{|G|^{2}}{|H||K|}=[G:H][G:K]$
\end{center}
\end{proof}
\begin{prob}
If $H,K,L$ are subgroups of a finite group $G$ such that $H\subseteq K$, then
\[
[K:H] \ge [L\cap K : L\cap H].
\]
\end{prob}
\newpage
%%%%%%%%%%%%%%%%%%%%%%%%%%%%%%%%%%%%%%%%%%%%%%%%%%%%%%%%%% Michael Jordan %%%%%%%%%%%%%%%%%%%%%%%%%%%%%%%%%%%%%%%%%%%%%%%%%%%%
\begin{prob}
  Let $H,K$ be subgroups of a group $G$. Assume that $H\cup K$ is a subgroup of $G$. Prove that either $H\subseteq K$ or $K\subseteq H$.
\end{prob}

\begin{proof}
  $H\cup K\leq G\implies \forall (h,k)\in H\times K,\text{ we have }hk\in H\cup K$ by closure. Therefore,
  \begin{center} 
    $H\cup K=\{g\mid g\in H\text{ or }g\in K\}$ so for each product $hk\in H\cup K$ either $hk=g\in H$ or $hk=g\in K$ or both.
  \end{center}
  So in fact the only certainty here is that $H\cup K\neq H\sqcup K$ otherwise $hk\notin H\cup K$ which is a subgroup of $G$. Therefore, necessarily $K\subset H$ or $H\subset K$ or $H=K$.

  Thus,
  $$H,K,H\cup K\leq G\implies H\subseteq K\text{ or }K\subseteq H.$$
\end{proof}
%%%%%%%%%%%%%%%%%%%%%%%%%%%%%%%%%%%%%%%%%%%%%%%%%%%%%%%%%%%%%%%%%%%%%%%%%%%%%%%%%%%%%%%%%%%%%%%%%%%%%%%%%%%%%%%%%%%%%%%%%%%%%%
\begin{prob}
Let $G$ be an abelian group, $H$ a subgroup of $G$ such that $G/H$ is an infinite cyclic group. Prove that $G \cong H \times G/H$.
\end{prob}

\end{document}