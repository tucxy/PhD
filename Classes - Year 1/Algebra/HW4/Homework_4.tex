\documentclass[addpoints,10pt]{exam}

\usepackage{amsmath,amsthm,enumitem,wrapfig,amsfonts,mathtools}
\usepackage[mathscr]{euscript}
\usepackage[super]{nth}
\usepackage{bbm}
\usepackage{dsfont}

\usepackage{geometry}
\usepackage[T1]{fontenc} % Use 8-bit encoding that has 256 glyphs
\renewcommand{\rmdefault}{ptm} %Change the Front Family from the default(cmr) to ptm(Times)
\usepackage{amsmath,amsfonts,amsthm,amssymb} % Math packages
\usepackage{bm}
\usepackage{mathptmx}
\usepackage{graphicx}
\usepackage{sectsty} % Allows customizing section commands
% \allsectionsfont{\centering} % Make all sections centered, the default font and small caps
\usepackage{pgfplots}
\pgfplotsset{compat=1.18}
\usetikzlibrary{arrows.meta}
\usepackage{xcolor}
\definecolor{darkpastelgreen}{rgb}{0.01, 0.75, 0.24}
\definecolor{blue-violet}{rgb}{0.54, 0.17, 0.89}
%\usepackage{polylongdiv} %polynomial long division
% Custom problem environment
\usepackage{polynom}
\newcounter{cprob}
\newenvironment{cprob}[1]{%
    \setcounter{cprob}{#1}%
    \noindent\textbf{Problem \thecprob.}%
}{%
    \par\bigskip%
}

\theoremstyle{plain}
\newcommand{\theoremname}{Theorem}
\newtheorem{thm}{\protect\theoremname}
  \theoremstyle{definition}
  \newtheorem{prob}[thm]{Problem}
  \newtheorem*{problem*}{Open Problem}
  \theoremstyle{plain}
  \newtheorem{conjecture}[thm]{Conjecture}
  \theoremstyle{plain}
  \newtheorem{lem}[thm]{Lemma}
  \newtheorem*{lem*}{Lemma}
  \newtheorem{obs}[thm]{Observation}
  \newtheorem{cor}[thm]{Corollary}
  \theoremstyle{definition}
\newtheorem{definition}[thm]{Definition}


% Patch prob environment to be single spaced
\let\oldprob\prob
\let\endoldprob\endprob
\renewenvironment{prob}
  {\begin{singlespace}\oldprob}
  {\endoldprob\end{singlespace}}

% start problem one line below like for enumerated problems with multiple parts
\newcommand{\belowtitle}{\leavevmode\newline}
%\Observe command
\newcommand{\Observe}{\text{Observe.}}
%(=>)
\newcommand{\IF}{\mathbf{(\Rightarrow)}}
%(<=)
\newcommand{\FI}{\mathbf{(\Leftarrow)}}
%equivalence classes; \class[S]{ *content in square brackets* }
\newcommand{\class}[2][]{\ensuremath{\left[\,#2\,\right]_{#1}}}

\newcommand{\horrule}[1]{\rule{\linewidth}{#1}}
\newcommand{\kkk}{\ensuremath{\Bbbk}} 
\newcommand{\CC}{\ensuremath{\mathbb{C}}}
\newcommand{\FF}{\ensuremath{\mathbb{F}}}
\newcommand{\KK}{\ensuremath{\mathbb{K}}}
\newcommand{\NN}{\ensuremath{\mathbb{N}}}
\newcommand{\QQ}{\ensuremath{\mathbb{Q}}} 
\newcommand{\RR}{\ensuremath{\mathbb{R}}} 
\newcommand{\ZZ}{\ensuremath{\mathbb{Z}}}
\newcommand{\MM}{\ensuremath{\mathcal{M}}}
\newcommand{\TT}{\ensuremath{\mathcal{T}}}
\newcommand{\BB}{\ensuremath{\mathcal{B}}}
\newcommand{\VV}{\ensuremath{\mathcal{V}}}
\newcommand{\WW}{\ensuremath{\mathcal{W}}}
\newcommand{\UU}{\ensuremath{\mathcal{U}}}
\newcommand{\PP}{\ensuremath{\mathcal{P}}}
\newcommand{\LL}{\ensuremath{\mathcal{L}}}
\newcommand{\kk}{\ensuremath{\mathds{k}}}
\newcommand{\EE}{\ensuremath{\mathbb{E}}}

\newcommand{\sm}{\char`\\}
%vector stuff
\DeclarePairedDelimiter{\ip}{\langle}{\rangle} %inner product/generate
\DeclarePairedDelimiter{\norm}{\lVert}{\rVert} %norm
\DeclarePairedDelimiter{\sqb}{\lbrack}{\rbrack} %corrd
\newcommand{\floor}[1]{\left\lfloor #1 \right\rfloor}
\newcommand{\ceil}[1]{\left\lceil #1 \right\rceil}
\newcommand{\mbf}[1]{\ensuremath{\mathbf{#1}}}
\newcommand{\tbf}[1]{\textbf{ #1 }}
\newcommand{\Span}{\ensuremath{\mathrm{Span}}}
\newcommand{\Char}[1]{\mathrm{Char}\; #1}
\DeclareMathOperator{\lcm}{lcm}


\makeatletter
\renewcommand*\env@matrix[1][*\c@MaxMatrixCols c]{%
  \hskip -\arraycolsep
  \let\@ifnextchar\new@ifnextchar
  \array{#1}}
\makeatother

\def\env@matrix{\hskip -\arraycolsep
  \let\@ifnextchar\new@ifnextchar
  \array{*\c@MaxMatrixCols c}}

  \newcommand{\proj}[2]{\text{proj}_{#1}(#2)}
  

%%% Formatting: Page Header
\newcommand{\StudentName}{Danny Banegas}
\newcommand{\AssignmentName}{Homework 4}
\newcommand{\CourseName}{MATH 721 - Algebra II}


\pagestyle{headandfoot}
\runningheadrule
\firstpageheadrule
\firstpageheader{\CourseName}{\StudentName}{\AssignmentName}
\runningheader{\CourseName}{\StudentName}{\AssignmentName}
\firstpagefooter{}{\thepage}{}
\runningfooter{}{\thepage}{}

\printanswers

\DeclareMathAlphabet{\mathcal}{OMS}{cmsy}{m}{n}

\usepackage{parskip}
\usepackage{setspace}
\doublespacing
% % % % % % % % % % % % % % % % % % % % % % % % % % % % % % % % % % % % % % % % % % % % % % % % % % % % % % % % % % % % % % 
\begin{document}
Turn in problems 54, 55, 57, 58, 59, 60, 61, 62, 63, 64, 65.
%%%%%%%%%%%%%%%%%%%%%%%%%%%%%%%%%%% 40 %%%%%%%%%%%%%%%%%%%%%%%%%%%%%%%%%%%%%%

%%%%%%%%%%%%%%%%%%%%%%%%%%%%%%%%%%% 54–65 %%%%%%%%%%%%%%%%%%%%%%%%%%%%%%%%%%%%%%
\setcounter{thm}{53} % next prob is 54

\begin{prob}
If $f\in \KK[X]$ (with $\KK$ field) has degree $n$ and $\FF$ is a splitting field of $f$ over $\KK$, prove that $[\FF:\KK]\mid n!$.
\end{prob}

\begin{proof}
  If $f$ has a degree $1$ over $\KK$, then it has only one zero $a$ whose minimal polynomial must have degree $1=[\KK(a):\KK]=[\FF:\KK]\mid 1!.$ If $f$ has degree $2$ over $\KK$, then it has at most two distinct zeros. Suppose $f$ is redicuble. Then it splits into two linear factors over $\KK$ and so $\FF\cong \KK\implies [\FF:\KK]=1\mid 2!$. Otherwise $f$ is irreducible, and so the minimal polynomial for a zero $a_{1}$ of $f$ must be of the form $\frac{f(x)}{\ell}$ for some $\ell\in \KK$, and therefore both zeros $a_{1}$ and $a_{2}$ share the same minimal polynomial $x^{2}+bx+c=(x-a_{1})(x-a_{2})\in \KK[x]$. So then $x^{2}-(a_{1}+a_{2})x+a_{1}a_{2}=x^{2}+bx+c\implies a_{2}=-b-a_{1}\in \KK(a)$ and so $\FF\cong \KK(a)\implies [\FF:\KK]\in \{1,2\}$ both of which divide $2!.$ Suppose $[\FF:\KK]\mid d!$ if $\FF$ is the splitting field of any degree $d$ polynomial $f$ over $\KK$ for all $1\leq d<m$ for some $m\geq 2$. Consider the statement for a degree $m$ polynomial $f$ over $\KK.\newline$

  If $f$ is reducible, then $f(x)=P(x)Q(x)$ for some non-constant degree $p$ and $(m-p)$ polynomials $P$ and $Q$ over $\KK$. Let $\FF_{P}$ be the splitting field of $P$ over $\KK$ and $\FF_{Q}$ be the splitting field of $Q$ over $\FF_{P}$. Since $\deg_{\KK}(P(x))=p,\,deg_{\FF_{P}}(Q(x))=deg_{\KK}(Q(x))=m-p<m$, we have that $[\FF_{Q}:\FF_{P}]\mid (m-p)!$ an $[\FF_{P}:\KK]\mid p!$. Well, $\FF_{Q}=(\FF_{P})(\alpha\mid Q(\alpha)=0)\cong (\KK(a\mid P(a)=0))(b\mid Q(b)=0)=\KK(\alpha\mid P(\alpha)=0\text{ or }Q(\alpha)=0)\cong \FF$. $\text{So finally, }\FF_{Q}\supseteq \FF_{P}\supseteq \KK\implies [\FF:\KK]=[\FF_{Q}:\KK]=[\FF_{Q}:\FF_{P}][\FF_{P}:\KK] \mid p!(m-p)!\mid m!\;(\text{via }\binom{p}{m}=\frac{m!}{p!(m-p)!}).$ So $[\FF:\KK]\mid m!$.

  If $f$ is irreducible, then for any zero $a$ of $f$, $[\KK(a):\KK]=m$ and by the division algorithm we have $f(x)=(x-a)Q(x)$ over $\KK(a)$ where $Q$ has degree $m-1$. Since $Q$ has degree less than $m$, the splitting field $\FF_{Q}$ of $Q$ over $\KK(a)$ must be such that $[\FF_{Q}:\KK(a)]\mid (m-1)!$ and since $\FF_{Q}=(\KK(a))(\alpha \mid Q(\alpha)=0)=\KK(\alpha \mid x-\alpha =0 \text{ or }Q(\alpha)=0)\cong \FF$ we have that $[\FF:\KK]=[\FF_{Q}:\KK]=[\FF_{Q}:\FF_{P}][\FF_{P}:\KK]$ divides $m(m-1)!=m!$.

  Thus, by induction,
  \begin{center}
  If $\FF\supseteq \KK$ is the splitting field of a degree $n\in \ZZ^{+}$ polynomial over $\KK$, then $[\FF:\KK]\mid n!$.
  \end{center}
\end{proof}

\begin{prob}
If $\KK\subseteq \FF$ is a field extension, $\FF$ is algebraically closed, and $\EE$ is the set of all elements of $\FF$ that are algebraic over $\KK$, prove that $\EE$ is an algebraic closure of $\KK$.
\end{prob}

\begin{proof}
All elements of $\EE=\{\alpha\in \FF\mid \alpha \text{ is algebraic over }\KK\}$ are algebraic over $\KK$. Now, consider some polynomial $f(x)$ over $\KK$. $\FF\supseteq \KK\implies \FF[x]\supseteq \KK[x]\implies f(x)\in \FF$ and since $\FF$ is algebraically closed, any zero $a$ of $f(x)$ must belong to $\FF$. So any algebraic $a$ over $\KK$ belongs to $\{\alpha\in \FF\mid \alpha \text{ is algebraic over }\KK\}=\EE$.

Next, we prove that $\EE$ is a field. For any $\alpha,\beta\in \EE\subseteq \FF$ with $\beta\neq 0$,
  \begin{align*}
    &\alpha\beta^{-1},\alpha-\beta\in \KK(\alpha,\beta)\implies \KK(\alpha,\beta,\alpha\beta^{-1})=\KK(\alpha,\beta,\alpha-\beta)=\KK(\alpha,\beta)\\ \implies &\ZZ^{+}\ni[\KK(\alpha,\beta):\KK]=[\KK(\alpha,\beta,\alpha\beta^{-1}):\KK(\alpha\beta^{-1})][\KK(\alpha\beta^{-1}):\KK]\\
     \implies &\ZZ^{+}\ni[\KK(\alpha,\beta):\KK]=[\KK(\alpha,\beta,\alpha-\beta):\KK(\alpha-\beta)][\KK(\alpha-\beta):\KK]\\
     \implies &[\KK(\alpha\beta^{-1}):\KK],[\KK(\alpha-\beta):\KK]\in \ZZ^{+}
  \end{align*}
Therefore, since adjoining $\alpha\beta^{-1}$ or $\alpha-\beta$ to $\KK$ gives a finite extension of $K$, they must algebraic over $\KK$, and so they both belong to $\EE$, which is then a subfield of $\FF.$ Additionally, since every element of $\EE$ is algebraic over $\KK$, $\EE$ is an algebraic extension of $\KK$. Lastly, consider any $g(x)=\sum_{i=0}^{n}\alpha_{i}x^{i}\in \EE[x]$. Since $\EE$ is algebraic over $\KK$, $\alpha_{0},\hdots, \alpha_{n}$ are all algebraic over $\KK$. So then any zero $\beta$ of $g(x)$ is algebraic over $\KK(\alpha_{0},\hdots,\alpha_{n})$, which must be a finite extension of $\KK$. Observe.
\begin{align*}
&\KK(\alpha_{0},\hdots,\alpha_{n},\beta)\supseteq \KK(\beta)\supseteq \KK\text{ and }[\KK(\alpha_{0},\hdots,\alpha_{n},\beta):\KK(\alpha_{0},\hdots,\alpha_{n})]\in \ZZ^{+}\\
\implies & [\KK(\alpha_{0},\hdots,\alpha_{n},\beta):\KK]=[\KK(\alpha_{0},\hdots,\alpha_{n},\beta):\KK(\alpha_{0},\hdots,\alpha_{n})][\KK(\alpha_{0},\hdots,\alpha_{n}):\KK]\in \ZZ^{+}\\
\implies &\ZZ^{+}\ni[\KK(\alpha_{0},\hdots,\alpha_{n},\beta):\KK]=[\KK(\alpha_{0},\hdots,\alpha_{n},\beta):\KK(\beta)][\KK(\beta):\KK]
\end{align*}
So $[\KK(\beta):\KK]\in \ZZ^{+}\implies \beta\text{ is algebraic over $\KK$}.$ Therefore, $\beta\in \EE \text{ and } \EE$ is an algebraically closed, algebraic extension of $\KK$.
Thus,
\begin{center}
$\EE$ is an algebraic closure of $\KK$.
\end{center}
\end{proof}

\setcounter{thm}{56} % skip 56

\begin{prob}
If $[F:K]=2$, then $K\subseteq F$ is a normal extension.
\end{prob}

\begin{prob}
If $d$ is a nonnegative rational number, then $\operatorname{Aut}_{\mathbb{Q}}\!\big(\mathbb{Q}(\sqrt{d})\big)$ is the identity or is isomorphic to $\mathbb{Z}_{2}$.
\end{prob}

\begin{prob}
What is the Galois group of $\mathbb{Q}\!\left(\sqrt{2},\sqrt{3},\sqrt{5}\right)$ over $\mathbb{Q}$?
\end{prob}

\begin{prob}
Assume $K$ is a field of characteristic $0$. Let $G$ be the subgroup of $\operatorname{Aut}_{K}\!\big(K(X)\big)$ generated by the $K$-automorphism induced by $X\mapsto X+1$. Prove that $G$ is an infinite cyclic group. What is the fixed field $E$ of $G$? What is $[K(X):E]$?
\end{prob}
\newpage
We prepare for the next problem by proving a bunch of useful stuff.
\setcounter{thm}{0} % next prob is 54
\begin{singlespace}
\begin{lem}
If $\KK$ is a field and $f(x)=\in \KK[x]$, $f(x)$ has some repeated zero $\alpha$ if and only if $\alpha$ is also a zero of $f^{\prime}(x)$, the formal derivative of $f(x)$ over $\KK$.
\end{lem}
\begin{proof}
  $(\implies)$ If $\alpha$ is a repeated zero of $f(x)$, then $f(x)=(x-\alpha)^{2}g(x)\in \overline{\KK}[x]$ for some $g(x)$ over an algebraic closure $\overline{\KK}$ of $\KK$. So then $f^{\prime}(x)=2(x-\alpha)g(x)+(x-\alpha)^{2}g^{\prime}(x)\implies f^{\prime}(\alpha)=0$. $(\impliedby)$ On the other hand if $f(\alpha)=f^{\prime}(\alpha)=0$, then $f(x)=(x-\alpha)q(x)\in \overline{\KK}[x]$ for some $q(x)$ over $\overline{\KK}$. Therefore, $f^{\prime}(x)=q(x)+(x-\alpha)q^{\prime}(x)$ and so $f^{\prime}(\alpha)=q(\alpha)=0\implies q(\alpha)=0$ since $q(x)=0$ implies that $f(x)=0$, a contradiction. So then $q(\alpha)=0\implies (x-\alpha)\mid q(x)\in \overline{\KK}[x]\setminus\overline{\KK}\implies f(x)=(x-\alpha)^{2}h(x)$ for some $h(x)\in \overline{\KK}[x]$ where $q(x)=(x-\alpha)h(x)$. So $\alpha$ is a repeated zero of $f(x)$.

\end{proof}
\setcounter{thm}{1} % next prob is 5
\begin{thm}
$\KK$ is a finite field
  \begin{align}
    &\implies \Char{\KK}=p>0\text{ for some prime $p$}\\
    &\iff \KK\text{ is some $n$-dimensional $\kk_{p}$-vector space where }n\in \ZZ^{+}\text{ and }\kk_{p}\cong \ZZ_{p}\\
    &\iff \KK\text{ is a splitting field of }f_{p,n}(x)=x^{(p^{n})}-x\text{ over }\kk_{p}\text{ where } n\in \ZZ^{+}\text{ and }\kk_{p}\cong \ZZ_{p}
  \end{align}
\end{thm}
  \begin{proof}
    (1) We can't have $\Char{\KK}=0$ otherwise $\KK$ would be infinite, so $\Char{\KK}=p>0$. Suppose $p$ is not prime. (We typically just denote an $n$-sum of $1$'s $n\coloneq\sum_{i=1}^{n}$ in the context of our fields.) So $p=ab=0$ for some $1\leq a,b<p$. But then $p=ab=0$ for some $a,b\neq 0$ and then $\KK$ has zero divisors, a contradiction. So $\Char{\KK}=p>0$ for some prime $p.\newline$

    (2) $(\implies)$ Since $(1)\implies \Char{\KK}=p>0$ for some prime $p$, we have that $\kk_{p}=\langle 1_{\KK}\rangle_{+}\cong \langle 1_{\ZZ_{p}}\rangle_{+}=\ZZ_{p}$, a finite field because $p$ is prime, via $1_{\KK}\leftrightarrow 1_{\ZZ_{p}}$. So then $\kk_{p}\subseteq\KK$ is a subfield, and immediately by the field axioms we have that $\KK$ is a $\kk_{p}$-vector space. Additionally, since $\KK$ is finite, it must also be finite dimensional over $\kk_{p}$. Therefore, $\KK$ is an $n$-dimensional $\kk_{p}$-vector space for some $n\in \ZZ^{+}$ and $\kk_{p}\cong \ZZ_{p}$. $(\impliedby)$ If $\KK$ is an $n$-dimensional $\kk_{p}$-vector space where $n\in \ZZ^{+}$ and $\kk_{p}\cong \ZZ_{p}$, then $|\KK|=p^{n}$ and so $\KK$ is finite.

    (3) $(\implies)$ By $(2)$ we have that $|\KK|=p^{n}$ for some $n\in \ZZ^{+}$ and then for any $a\in \KK^{*}=\KK\setminus \{0\}$, the multiplicative group of $\KK$, we have that $|a|_{\boldsymbol{\cdot}}$ divides $|\KK^{*}|=p^{n}-1$. Therefore, $a^{p^{n}-1}=1\implies a^{p^{n}}=a$. Therefore, every $a\in \KK$ is a zero of $f_{p,n}(x)=x^{(p^{n})}-x\in \kk_{p}[x]$. Now, $f^{\prime}_{p,n}(x)=p^{n}x^{p^{n-1}}-1=-1$ since $\Char{\KK}=p$ and so by \textbf{Lemma 1.} $f^{\prime}_{p,n}(\alpha)=-1\neq 0 \text{ for all zeros } \alpha\text{ of }f_{p,n}(x)\implies f_{p,n}(x)$ has no repeated zeros. So then since $f_{p,n}(x)$ has at most $p^{n}$ zeros which are all distinct, in fact $\KK$ must be exactly all distinct zeros of $f_{p,n}(x)$. Suppose $f_{p,n}(x)$ splits completely in a smaller field $\mathbb{M}$ where $|\mathbb{M}|<|\KK|$. But then $\KK=\{\alpha\in \overline{\KK}\mid f_{p,n}(\alpha)=0\}\subseteq \mathbb{M}\implies |\mathbb{M}|\geq |\KK|$ for some algebraic closure $\overline{\KK}$ of $\KK$, a contradiction. So then $\KK$ is a splitting field of $f_{p,n}(x)$ over $\kk_{p}$. $(\impliedby)$ If $\KK$ is a splitting field of $f_{p,n}(x)=x^{(p^{n})}-x\text{ over }\kk_{p}\cong \ZZ_{p}$, then it is generated by finitely many algebraic elements over $\kk_{p}$ which means it is algebraic over $\kk_{p}$ and therefore a finite extension of the finite field $\kk_{p}$ which means it is finite field itself.

  \end{proof}
Now, we prepare some Corollaries.
\end{singlespace}
\newpage
\setcounter{equation}{0}
\begin{singlespace}
\begin{cor}
If $\KK$ is a finite field of characteristic $p>0$, then inside a fixed algebraic closure $\overline{\KK}$
\begin{align}
  &\kk_{p}=\langle 1 \rangle_{+}\cong \ZZ_{p}\text{ is the prime subfield of }\KK\\
  &\text{There exists a unique extension }\KK_{m}\supseteq \KK\text{ with }[\KK_{m}:\KK]=m\in \ZZ^{+}\\
  &\text{There exists a unique subfield }\kk_{p,d}\subseteq\KK\text{ with }[\kk_{p,d}:\kk_{p}]=d\text{ for each divisor }d\text{ of }[\KK:\kk_{p}]
\end{align}
\end{cor}
\begin{proof}
  (1) Suppose there exists some subfield $\mathbb{M}\subseteq \kk_{p}$ smaller than $\kk_{p}$, so $|\mathbb{M}|<|\kk_{p}|$. But then $0,1\in \mathbb{M}\implies \kk_{p}=\langle 1\rangle_{+}\subseteq \mathbb{M}\implies |\mathbb{M}|\geq |\kk_{p}|$, a contradiction. So $\kk_{p}$ must be the prime subfield of $\KK.\newline$

  (2) By \textbf{Theorem 2.}, $\KK$ has dimension $n$ over $\kk_{p}$ for some $n\in \ZZ^{+}$. Now, for some $m\in \ZZ^{+}$ let $f_{p,n,m}(x)=x^{(p^{n})^{m}}-x\in \KK_{p}$ and $\KK_{m}=\{\alpha\in \overline{\KK}\mid f_{p,n,m}(\alpha)=0\}$. Observe.$\newline$

  For any $\alpha, \beta\in \KK_{m}$ with $\beta\neq 0$, $f_{p,n,m}(\alpha-\beta)=(\alpha-\beta)^{p^{nm}}-(\alpha-\beta)=(\sum_{i=0}^{p^{mn}}\binom{p^{nm}}{i}\alpha^{p^{mn}-i}(-\beta)^{i})-(\alpha-\beta)=(\alpha^{p^{mn}}+(-\beta)^{p^{mn}})-(\alpha-\beta)$ since $\binom{p^{mn}}{0}=\binom{p^{mn}}{p^{mn}}=1$ and $p\mid \binom{p^{nm}}{i}$ for all $0<i<p^{mn}$ (I am uninterested in proving this part.) and since $p=0$ we get that in fact $f_{p,n,m}(\alpha-\beta)=(\alpha^{p^{nm}}-\beta^{p^{nm}})-(\alpha-\beta)=\alpha-\beta - (\alpha - \beta)=0$. Note that this holds for $p=2$ since $-\alpha=\alpha$ in a field with characteristic $2$. Therefore, $\alpha-\beta\in \KK_{m}$. Next, $f_{p,n,m}(\alpha\beta^{-1})=(\alpha\beta)^{p^{mn}}=\alpha^{p^{nm}}\beta^{-p^{nm}}-(\alpha\beta^{-1})=\alpha(\beta^{p^{mn}})^{-1}-\alpha\beta^{-1}=\alpha(\beta)^{-1}-\alpha\beta^{-1}=0\implies \alpha\beta^{-1}\in \KK_{m}.\newline$

  So $\KK_{m}$ is a subfield of $\overline{\KK}$ in which $f_{p,n,m}(x)$ splits completely. Suppose there is a smaller such field $\mathbb{M}$ that $f_{p,n,m}(x)$ splits completely over, so $|\mathbb{M}|<|\KK_{m}|$. But then $\KK_{m}=\{\alpha\in \overline{\KK}\mid f_{p,n,m}(\alpha)=0\}\subseteq \mathbb{M}\implies |\mathbb{M}|\geq |\KK_{m}|$, a contradiction. So $\KK_{m}$ must be a splitting field of $f_{p,n,m}(x)$ over $\kk_{p}$. Finally, $a^{(p^{n})}=a$ for all $a\in \KK$. Suppose that for any $a\in \KK$, $a^{(p^{n})^{c}}=a$ for some $c\geq 1$. Then, $a^{(p^{n})^{c+1}}=(a^{(p^{n})^{c}})^{p^{n}}=(a)^{p^{n}}=a.$ Therefore, by induction $a^{(p^{n})^{c}}=a$ for all $c\geq 1$. So then every element $a\in \KK$ is a zero of $f_{p,n,m}(x)=x^{(p^{n})^{m}}-x$ and $\KK_{m}\supseteq \KK \supseteq \kk_{p}$. So then we have $[\KK_{m}:\kk_{p}]=[\KK_{m}:\KK][\KK:\kk_{p}]=nm=[\KK_{m}:\KK]n\implies [\KK_{m}:\KK]=m$. Suppose some other distinct extension $\EE_{m}\supseteq \KK$ of degree $m$ over $\KK$ exists. Well, it has order $p^{nm}$ and all of it's elements must be zeros of $f_{p,n,m}(x)$ via $|\EE_{m}^{*}|=p^{n}-1$, and then since $\KK_{m}-\EE_{m}\neq \emptyset$ we have that $|\KK_{m}\cup \EE_{m}|>p^{nm}$ and so $f_{n,p,m}(x)$ has more than $p^{nm}$ distinct zeros, a contradiction. So $\KK_{m}$ is the unique extension with $[\KK_{m}:\KK]=m$ with respect to the closure $\overline{\KK}.\newline$

  (3) By (2), we immediately get that there exists a unique extension $\kk_{p,d}\supseteq \kk_{p}$ with $[\kk_{p,d}:\kk_{p}]=d$ for each divisior $d|n$ which is a splitting field for $f_{p,m,d}(x)=x^{p^{d}}-x$ over $\KK_{p}$. For any such divisor $d|n$, $a^{p^{d}}=a$ for all $a\in \kk_{p,d}$. So then since $d|n, n=dq$ for some $q\in \ZZ^{+}$ and therefore by the induction earlier replacing $\KK$ with $\kk_{p,d}$ we get that $a^{p^{n}}=a^{p^{dq}}=a$ for all $a\in \kk_{p,d}$. Therefore every element in $\kk_{p,d}$ is a zero of $f_{p,n}(x)=x^{p^{n}}-a$ over $\kk_{p}$, and since $\KK$ is a splitting field for $f_{p,n}(x)$ in fact we have that $\KK\supseteq \kk_{p,d}\supseteq \kk_{p}$. So then for each divisor $d\mid n$, $\kk_{p,d}$, the splitting field of $f_{p,d}$ over $\kk_{p}$, is a unique subfield of $\KK$ with $[\kk_{p,d}:\kk_{p}]=d$.

\end{proof}
\end{singlespace}

Alright now let's do the problem. I just wanted to prove this all myself instead of looking over the notes.
\newpage
\setcounter{thm}{61} % next prob is 61
\newpage
\begin{prob}
Let $k$ be a finite field of characteristic $p>0$.
\begin{enumerate}[label=(\alph*)]
\item Prove that for every $n>0$ there exists an irreducible polynomial $f\in k[X]$ of degree $n$.
\item Prove that for every irreducible polynomial $P\in k[X]$ there exists $n\ge 0$ such that $P$ divides $X^{p^{\,n}}-X$.
\end{enumerate}
\end{prob}
\begin{proof}
  Fix some algebraic closure $\overline{\KK}\supseteq \KK$. Every object that follows is contained in $\overline{\KK}$. By our theorems, if $\KK$ is a finite field of characteristic $p>0$, $p$ is prime and $\KK$ is an $n$-dimensional $\langle 1\rangle_{+}=\kk_{p}$-vector space for some $n\in \ZZ^{+}$. Also, there exists a unique extension $\KK_{m}$ of $\KK$ with $[\KK_{m}:\KK]=m$ for each $m\in \ZZ^{+}$. Finally, both $\KK$ and $\KK_{m}$ are unique splitting fields of $f_{p,n}=x^{p^{n}}-x$ and $f_{p,n,m}=x^{p^{nm}}$, respectively, over $\kk_{p}$ with respect to $\overline{\KK}$. Suppose that for all $\alpha\in \KK_{m}$, the degree of $\alpha$ over $\KK$ is strictly less than $[\KK_{m}:\KK]=m$. Any $\alpha$ must belong to $\KK(\alpha)\subseteq \KK_{m}$ and $[\KK(\alpha):\KK]=d$ for some divisor $0<d<nm$ of $[\KK_{m}:\kk_{p}]=nm$. Well, by our theorems this must be the unique subfield $\kk_{p,d}$ of order $p^{d}$. Therefore,
  $$\KK_{m}=\bigcup_{\substack{d\mid nm\\ 0<d<nm}} \kk_{p,d}\implies |\KK_{m}|=p^{nm}=|\bigcup_{\substack{d\mid nm\\ 0<d<nm}} \kk_{p,d}|\leq \sum_{\substack{d\mid nm\\ 0<d<nm}}p^{d}<\sum_{i=0}^{nm-1}p^{i}=\frac{p^{nm}-1}{1-p}<p^{nm},$$
  a contradiction. (Another contradiction is just the fact that then every $\alpha\in \KK_{m}$ belongs to $\kk_{p,d}$ for the largest $d<nm$ that divides $nm$ and is less than $m$, but $|\KK_{m}|$ is strictly less than $p^{nm}$. We did not prove directly that all these subfields are nested, so I didn't do that.) Therefore, there exists an element $\alpha_{m}\in \KK_{m}$ with degree $m$ over $\KK$, and so there exists a monic irreducible polynomial of $\alpha_{m}$ over $\KK$ with degree $m$, that is $\KK_{m}=\KK(\alpha_{m}).\newline$

  Now, consider any irreducible polynomial $P(x)\in \KK[x].$ It must have some degree $q\in \ZZ^{+}$, and some zero $\alpha$ with the minimal polynomial $P_{\alpha}(x)=\frac{P(x)}{a}\in \KK[x]$ where $a\in \KK$ is the leading coefficient of $P(x)$. So then $\alpha$ has degree $q$ over $\KK$ and $[\KK(\alpha):\KK]=q$. Therefore, by our theorems, $\KK(\alpha)=\KK_{q}\subseteq \overline{\KK}$, the splitting field of $f_{p,n,q}(x)=x^{p^{nq}}-x$ over $\kk_{p}$. Well, $\alpha\in \KK_{q}\implies |\alpha|_{_{\boldsymbol{\cdot}}}$ divides $|\KK_{q}^{*}|=p^{nq}-1$ and so $\alpha^{p^{nq}-1}=1\implies \alpha^{p^{nq}}=\alpha\implies f_{p,n,q}(\alpha)=(\alpha)^{p^{nq}}-\alpha=0$. So $\alpha$ is a zero of $f_{p,n,q}(x)=x^{p^{nq}}-x$ over $\kk_{p}$, which is also a polynomial over $\KK$.

\end{proof}
\newpage
\begin{prob}
Let $p$ be a prime and $\mathbb{F}_{q}$ (with $q=p^{s}$) be the finite field with $q$ elements. Let $f\in \mathbb{F}_{q}[X]$ be an irreducible polynomial. Prove that $f$ is irreducible in $\mathbb{F}_{q^{m}}[X]$ if and only if $m$ and $\deg f$ are relatively prime.
\end{prob}

\begin{proof}
  Let $f(x)$ be an irreducible polynomial over $\FF_{p^{n}}$ with degree $\deg{f(x)}=d$ for some prime $p$ and some $n\geq 1$. Also, let $f_{p,N}(x)=x^{p^{N}}-x$ over $\FF_{p}$ for any $N\in \ZZ^{+}$ and recall that every element of $\FF_{p^{nN}}$, the splitting field of $f_{p,N}(x)$, is a zero of $f_{p,N}(x)$.

  Now, since $f(x)$ is irreducible over $\FF_{p^{n}}$, any zero $\alpha$ of $f(x)$ has the minimal polynomial $p_{\alpha}(x)=\frac{f(x)}{c}$ over $\FF_{p^{n}}$ where $c$ is the leading coefficient of $f(x)$ and so it has degree $d$ over $\FF_{p^{n}}$. That is, $\FF_{p^{n}}(\alpha)=\FF_{p^{nd}}$. So then $\alpha$ is a zero of $f_{p,nd}(x)$, and we also have that $d$ is smallest integer such that $\alpha^{p^{nd}}=\alpha$. Furthermore, $\alpha^{nk}=\alpha\iff d\mid k$. 

  $(\implies)$ If $f(x)$ is irreducible over $\FF_{p^{nm}}$, then $\alpha$ has degree $d$ over both $\FF_{p^{n}}$ and $\FF_{p^{nm}}$. So $\alpha=k_{1}=k_{2}$ is the smallest integer such that $\alpha^{p^{nk_{1}}}=\alpha^{p^{nmk_{2}}}=\alpha$. Additionally, the previous equalities hold for any multiples $k_{1},k_{2}\geq 1$ of $d$. Suppose $g=\gcd(d,m)>1$, then $\frac{d}{g}=\ell_{d}< d,\frac{m}{g}=\ell_{m}< m$. Observe. 
    $$ m\ell_{d}=m\frac{d}{g}=\frac{m}{g}d=d\ell_{m}\implies \alpha^{p^{nm\ell_{d}}}=\alpha^{p^{nm(\frac{d}{g})}}=\alpha^{p^{nd(\frac{m}{g})}}=\alpha^{p^{nd\ell_{m}}}=\alpha.$$
  But then there is a smaller positive integer $k_{2}=\ell_{d}<d$ such that $\alpha^{nmk_{2}}=\alpha$, a contradiction. So we must have that $\gcd(d,m)=1$.

  $(\impliedby)$ On the other hand, if $\gcd(d,m)=1$, recall that that $d$ is the smallest positive integer $d=k_{1}$ such that $\alpha^{p^{nk_{1}}}=\alpha$ for any zero $\alpha$ of $f(x)$. So for any $k_{2}\geq 1$ such that $\alpha^{p^{nmk_{2}}}=\alpha$, we must have that $d\mid mk_{2}$. Suppose we have such a $k_{2}$ less than $d$. But then $\gcd(d,m)=1\text{ and }d\mid mk_{2}\implies d\mid k_{2}$ and $k_{2}<d$, which is impossible. So the smallest such $k_{2}=d$, which is also the degree of $\alpha$ over $\FF_{p^{nm}}$. Since degree of any zero $\alpha$ of $f(x)$ is $d$ over $\FF_{p^{nm}}$, $f(x)$ must be irreducible over $\FF_{p^{nm}}$. (Otherwise we have some minimal polynomial of degree less than d which can be pulled out of $f(x)$ over $\FF_{p^{nm}}$).

\end{proof}
\newpage
\begin{prob}
Prove that $E=\mathbb{F}_{2}[X]/(X^{4}+X^{3}+1)$ is a field with $16$ elements. What are the roots of $X^{4}+X^{3}+1$ in $E$?
\end{prob}

\begin{proof}
  Let $p(x)=x^{4}+x^{3}+1\in \FF_{2}$. $p(0)=p(1)=1\neq 0$, so $P(x)$ has no zeros in $\FF_{2}$ and therefore no linear factors over $\FF_{2}$, and so it can't factor into a linear and cubic. Suppose $P(x)$ is reducible. Then must split into two irreducible quadratics over $\FF_{2}$. Well, $x^{2}+1=(x+1)(x-1)$ and $x^{2}+x=x(x+1)$, and $x^{2}=x(x)$ over $\FF_{2}$. Since $x^{2}+x+1$ is the only irreducible quadratic over $\FF_{2}$, we must have that $P(x)=(x^{2}+x+1)^{2}$. Recall that $\Char \FF_{2}=2$ and so $(f(x)+g(x))^{2}=(f(x))^{2}+(g(x))^{2}$ over $\FF_{2}$. But then
    $$P(x)=x^{4}+x^{3}+1=((x^{2})+(x+1))^{2}=x^{4}+(x+1)^{2}=x^{4}+x^{2}+1,\text{ a contradiction.}$$
  So then $P(x)=x^{4}+x^{3}+1$ is irreducible of degree $4$ over $\FF_{2}$ and for any zero $\alpha$ of $P(x)$
   $$\FF_{2}[x]/\langle x^{4}+x^{3}+1\rangle=\Span_{\FF_{2}}\{[1],[x],[x^{2}],[x^{3}]\}\cong \Span_{\FF_{2}}\{1,\alpha,\alpha^{2},\alpha^{3}\}=\FF_{2}(\alpha).$$
   So $\EE=\FF_{2}[x]/\langle x^{4}+x^{3}+1\rangle$ is a $4$-dimensional $\FF_{2}$-vector space and we must have that $|\EE|=2^{4}=16$. For the remainder of this proof we will exclusively work in $\EE$ and so we refer to cosets $[f(x)]_{\EE}$ by any of their representatives $f(x)$ and let modulo $x^{4}+x^{3}+1$ be implied.

  So $\EE=\Span\{1,x,x^{2},x^{3}\}$ and since $x^{4}+x^{3}+1=0$ in this field, $x^{4}=-x^{3}-1=x^{3}+1$, and obviously $x$ is a root of $p(x)=x^{4}+x^{3}+1$ over $\EE$. Then, recall that since $\EE\cong \FF_{2^{4}}$, is a finite field extension of $\FF_{2}$ with characteristic $2$, we must have that the Frobenius mapping $\varphi_{2}\coloneq x\mapsto x^{2}$ is an endomorphism of $\EE$ (In fact it is a isomorphism in $\mathrm{Aut}_{\FF_{2}}(\FF_{2^{4}})$) and so it permutes roots of polynomials in $\EE[x]$ to each other. Well, $x^{4}=x^{3}+1\implies x^{5}=x(x^{4})=x(x^{3}+1)=x^{4}+x=x^{3}+x+1\implies x^{6}=x(x^{5})=x(x^{3}+x+1)=x^{4}+x^{2}+x=(x^{3}+1)+x^{2}+x=x^{3}+x^{2}+x+1.$
  
  Therefore, the orbit $\mathrm{Orb}_{\varphi_{2}}(\EE)=\{x\}\cup \{x^{2}\}\cup \{x^{4}=x^{3}+1\} \cup \{x^{8}=(x^{4})^{2}=(x^{3}+1)^{2}=x^{6}+1=(x^{3}+x^{2}+x+1)+1=x^{3}+x^{2}+x\}\cup \cdots = \{x,x^{2},x^{3}+1,x^{3}+x^{2}+x\}$ must be all roots of $P(x)=x^{4}+x^{3}+1$ over $\EE$, since they are four distinct elements and $P(x)$ has at most four distinct roots.

\end{proof}
\newpage
\begin{prob}
Prove that an algebraic extension of a perfect field is a perfect field.
\end{prob}

\begin{proof}
Let $\KK$ be a perfect field. If $\KK$ has characteristic $0$, then so does any extension of it since they share $1$, and so any algebraic extension of $\KK$ must be perfect.

If $\KK$ has characteristic $p>0$. Since $\KK$ is perfect, every irreducible polynomial $f(x)$ over $\KK$ has no repeated roots in some splitting field $\FF_{f(x)}$ of $f(x)$. That is, the minimal polynomial of any algebraic element has no repeated linear factors in $\FF_{P(x)}$.

Therefore, if $\EE$ is some algebraic extension of $\KK$, then any $\alpha\in \EE$ is algebraic over $\KK$. It's minimal polynomial $P_{\alpha,\KK}(x)$ over $\KK$ has no repeated roots some splitting field $\FF_{P(x)}$ of $P(x)$. Now, since $\alpha$ is a zero of $P_{\alpha,\KK}(x)$ which also belongs to $\EE[x]$, the minimal polynomial $P_{\alpha,\EE}(x)$ of $\alpha$ over $\EE$ must divide $P_{\alpha,\KK}(x)$ over $\EE$. Therefore, $P_{\alpha,\EE}(x)$ must split completely over $\FF_{P(x)}$ into distinct linear factors as well, since otherwise $P_{\alpha,\KK}(x)$ is divisible by some repeated linear factor of $P_{\alpha,\EE}(x)\mid P_{\alpha,\KK}(x)\in \FF_{P(x)}[x]$, a contradiction. So every minimal polynomial over $\EE$ is seperable. Since every irreducible polynomial over $\EE$ is simply some minimal polynomial of one of it's zeros, which we proved is seperable, scaled by some $c\in \EE$, every irreducible over $\EE$ is also seperable. Therefore, $\EE$ is perfect.

\end{proof}
\newpage
\begin{prob}
Show that the extension $\mathbb{Q}\subseteq \mathbb{Q}\!\left(\sqrt[4]{2},\, i\right)$ is Galois. Find its Galois group.
\end{prob}




\end{document}