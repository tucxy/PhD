\documentclass[addpoints,10pt]{exam}

\usepackage{amsmath,amsthm,enumitem,wrapfig,amsfonts,mathtools}
\usepackage[mathscr]{euscript}
\usepackage[super]{nth}
\usepackage{dsfont}
\usepackage{xparse}

\usepackage{geometry}
\usepackage[T1]{fontenc} % Use 8-bit encoding that has 256 glyphs
\renewcommand{\rmdefault}{ptm} %Change the Front Family from the default(cmr) to ptm(Times)
\usepackage{amsmath,amsfonts,amsthm,amssymb} % Math packages
\usepackage{bm}
\usepackage{mathptmx}
\usepackage{graphicx}
\usepackage{sectsty} % Allows customizing section commands
% \allsectionsfont{\centering} % Make all sections centered, the default font and small caps
\usepackage{pgfplots}
\pgfplotsset{compat=1.18}
\usetikzlibrary{arrows.meta}
\usepackage{xcolor}
\definecolor{darkpastelgreen}{rgb}{0.01, 0.75, 0.24}
\definecolor{blue-violet}{rgb}{0.54, 0.17, 0.89}
%\usepackage{polylongdiv} %polynomial long division
% Custom problem environment
\usepackage{polynom}
\newcounter{cprob}
\newenvironment{cprob}[1]{%
    \setcounter{cprob}{#1}%
    \noindent\textbf{Problem \thecprob.}%
}{%
    \par\bigskip%
}

\theoremstyle{plain}
\newcommand{\theoremname}{Theorem}
\newtheorem{thm}{\protect\theoremname}
  \theoremstyle{definition}
  \newtheorem{prob}[thm]{Problem}
  \newtheorem*{problem*}{Open Problem}
  \theoremstyle{plain}
  \newtheorem{conjecture}[thm]{Conjecture}
  \theoremstyle{plain}
  \newtheorem{lem}[thm]{Lemma}
  \newtheorem*{lem*}{Lemma}
  \newtheorem{obs}[thm]{Observation}
  \newtheorem{cor}[thm]{Corollary}
  \theoremstyle{definition}
\newtheorem{definition}[thm]{Definition}


% Patch prob environment to be single spaced
\let\oldprob\prob
\let\endoldprob\endprob
\renewenvironment{prob}
  {\begin{singlespace}\oldprob}
  {\endoldprob\end{singlespace}}

% start problem one line below like for enumerated problems with multiple parts
\newcommand{\belowtitle}{\leavevmode\newline}
%\Observe command
\newcommand{\Observe}{\text{Observe.}}
%(=>)
\newcommand{\IF}{\mathbf{(\Rightarrow)}}
%(<=)
\newcommand{\FI}{\mathbf{(\Leftarrow)}}
%equivalence classes; \class[S]{ *content in square brackets* }
\newcommand{\class}[2][]{\ensuremath{\left[\,#2\,\right]_{#1}}}

\newcommand{\horrule}[1]{\rule{\linewidth}{#1}}
\newcommand{\kkk}{\ensuremath{\Bbbk}} 
\newcommand{\CC}{\ensuremath{\mathbb{C}}}
\newcommand{\FF}{\ensuremath{\mathbb{F}}}
\newcommand{\KK}{\ensuremath{\mathbb{K}}}
\newcommand{\NN}{\ensuremath{\mathbb{N}}}
\newcommand{\QQ}{\ensuremath{\mathbb{Q}}} 
\newcommand{\RR}{\ensuremath{\mathbb{R}}} 
\newcommand{\ZZ}{\ensuremath{\mathbb{Z}}}
\newcommand{\MM}{\ensuremath{\mathcal{M}}}
\newcommand{\TT}{\ensuremath{\mathcal{T}}}
\newcommand{\BB}{\ensuremath{\mathcal{B}}}
\newcommand{\VV}{\ensuremath{\mathcal{V}}}
\newcommand{\WW}{\ensuremath{\mathcal{W}}}
\newcommand{\UU}{\ensuremath{\mathcal{U}}}
\newcommand{\PP}{\ensuremath{\mathcal{P}}}
\newcommand{\LL}{\ensuremath{\mathcal{L}}}
\newcommand{\kk}{\ensuremath{\mathds{k}}}
\newcommand{\EE}{\ensuremath{\mathbb{E}}}


\newcommand{\sm}{\char`\\}
%vector stuff
\DeclarePairedDelimiter{\ip}{\langle}{\rangle} %inner product/generate
\DeclarePairedDelimiter{\norm}{\lVert}{\rVert} %norm
\DeclarePairedDelimiter{\sqb}{\lbrack}{\rbrack} %corrd

\newcommand{\floor}[1]{\left\lfloor #1 \right\rfloor}
\newcommand{\ceil}[1]{\left\lceil #1 \right\rceil}
\newcommand{\mbf}[1]{\ensuremath{\mathbf{#1}}}
\newcommand{\tbf}[1]{\textbf{ #1 }}
\newcommand{\Span}{\ensuremath{\mathrm{Span}}}
\newcommand{\Char}[1]{\mathrm{Char}\; #1}
\DeclareMathOperator{\lcm}{lcm}
\newcommand{\id}{\ensuremath{\mathrm{id}}}
\newcommand{\Gal}[2]{\mathrm{Gal}(#1/#2)}
\newcommand{\Aut}{\ensuremath{\mathrm{Aut}}}
\newcommand{\Fix}[2]{\mathrm{Fix}_{#1}(#2)}
\newcommand{\nil}{\ensuremath{\mathrm{nil}}}
\newcommand{\Hom}{\ensuremath{\mathrm{Hom}}}





\makeatletter
\renewcommand*\env@matrix[1][*\c@MaxMatrixCols c]{%
  \hskip -\arraycolsep
  \let\@ifnextchar\new@ifnextchar
  \array{#1}}
\makeatother

\def\env@matrix{\hskip -\arraycolsep
  \let\@ifnextchar\new@ifnextchar
  \array{*\c@MaxMatrixCols c}}

  \newcommand{\proj}[2]{\text{proj}_{#1}(#2)}
  

%%% Formatting: Page Header
\newcommand{\StudentName}{Danny Banegas}
\newcommand{\AssignmentName}{Homework 4}
\newcommand{\CourseName}{MATH 721 - Algebra II}


\pagestyle{headandfoot}
\runningheadrule
\firstpageheadrule
\firstpageheader{\CourseName}{\StudentName}{\AssignmentName}
\runningheader{\CourseName}{\StudentName}{\AssignmentName}
\firstpagefooter{}{\thepage}{}
\runningfooter{}{\thepage}{}

\printanswers

\DeclareMathAlphabet{\mathcal}{OMS}{cmsy}{m}{n}

\usepackage{parskip}
\usepackage{setspace}
\doublespacing
% % % % % % % % % % % % % % % % % % % % % % % % % % % % % % % % % % % % % % % % % % % % % % % % % % % % % % % % % % % % % % 
\begin{document}

%%%%%%%%%%%%%%%%%%%%%%%%%%%%%%%%%%%%%%%%%%%%%%%%% 1 %%%%%%%%%%%%%%%%%%%%%%%%%%%%%%%%%%%%%%%%
\setcounter{thm}{0} % next prob is 1

\begin{prob}
Prove that if $0=1$ in a ring $R$, then $R$ is the zero ring (that is, its only element is $0$).
\end{prob}

\begin{proof}
$\forall r\in R$, $r=1\cdot r=0\cdot r=0.$ Thus, $R=\{0\}$.

\end{proof}

%%%%%%%%%%%%%%%%%%%%%%%%%%%%%%%%%%%%%%%%%%%%%%%%% 2 %%%%%%%%%%%%%%%%%%%%%%%%%%%%%%%%%%%%%%%%%%%%%%%%%

\begin{prob}
A category $C'$ is a subcategory of a category $C$ if
\begin{enumerate}[label=\arabic*)]
  \item $\mathrm{Obj}(C') \subseteq \mathrm{Obj}(C)$, and
  \item for all $A,B \in \mathrm{Obj}(C')$, $\mathrm{Hom}_{C'}(A,B) \subseteq \mathrm{Hom}_{C}(A,B)$.
\end{enumerate}
It is a full subcategory if in addition $\mathrm{Hom}_{C'}(A,B)=\mathrm{Hom}_{C}(A,B)$ for all $A,B \in \mathrm{Obj}(C')$.

Ring is a subcategory of Rng (you do not need to prove this). Show that it is not a full subcategory.
\end{prob}

\begin{proof}
$\varphi:\ZZ_{6}\mapsto \ZZ_{6}$ defined via $\varphi(a)=2a$ belongs to $\mathrm{Hom}_{\mathrm{Rng}}(\ZZ_{6},\ZZ_{6})$ but not $\mathrm{Hom}_{\mathrm{Ring}}(\ZZ_{6},\ZZ_{6})$ since $\varphi(1)=2\neq 1$, and a ring homomorphism in $\mathrm{Hom}_{\mathrm{Ring}}(\ZZ_{6},\ZZ_{6})$ must preserve $1$. Thus, $\mathrm{Hom}_{\mathrm{Ring}}(\ZZ_{6},\ZZ_{6})\subset \mathrm{Hom}_{\mathrm{Rng}}(\ZZ_{6},\ZZ_{6})\implies$ Ring is not a full subcategory of Rng.

\end{proof}

%%%%%%%%%%%%%%%%%%%%%%%%%%%%%%%%%%%%%%%%%%%%%%%%% 3 %%%%%%%%%%%%%%%%%%%%%%%%%%%%%%%%%%%%%%%%%%%%%%%%%

\begin{prob}
Let $a$ and $b$ be zero-divisors in a ring $R$. Either prove $a+b$ is always a zero-divisor or provide a specific counterexample.
\end{prob}

\begin{proof}
Consider $[2],[3]\in \ZZ_{6}$. $[2]\cdot [3]=[0]$ and both are non-zero, so they are both zero-divisors. Yet $[2]+[3]=[5]$, which isn't a zero divisor since $([5][m])_{m=0}^{5}=([0],[5],[4],[3],[2],[1])$. Thus, the sum of two zero-divisors is not always a zero-divisor.

\end{proof}
%%%%%%%%%%%%%%%%%%%%%%%%%%%%%%%%%%%%%%%%%%%%%%%%% 4 %%%%%%%%%%%%%%%%%%%%%%%%%%%%%%%%%%%%%%%%%%%%%%%%%
\newpage
\begin{prob}
The center of a ring $R$ is $Z(R)=\{z \in R \mid rz=zr \text{ for all } r \in R\}$. Prove that the center of a ring $R$ is a subring of $R$.
\end{prob}

\begin{proof}
$\forall a,b \in Z(R)$, and $\forall r\in R$,
\begin{align*}
    &\mathbf{[1]:} \; 1r=r1=r\implies 1\in Z(R)\\
    &\mathbf{[\leq_{+}]:} \; (a-b)r=ar-br=(ra)-(rb)=r(a-b)\implies a-b\in Z(R)\\
    &\mathbf{[\cdot]:} \; (ab)r=a(br)=a(rb)=(ar)b=(ra)b=r(ab)\implies ab\in Z(R). 
\end{align*}
Thus, $Z(R)$ is a subring of $R$.

\end{proof}

%%%%%%%%%%%%%%%%%%%%%%%%%%%%%%%%%%%%%%%%%%%%%%%%% 5 %%%%%%%%%%%%%%%%%%%%%%%%%%%%%%%%%%%%%%%%%%%%%%%%%

\begin{prob}
An element $x$ in a ring $R$ is called nilpotent if $x^m=0$ for some $m \in \ZZ^{+}$ (here $x^m$ denotes $x\cdot x \cdots x$ ($m$ times)).

Prove that the nilpotent elements of a commutative ring $R$ form an ideal (this is called the nilradical of $R$).
\end{prob}

\begin{proof}
Let $R$ be a commutative ring with unity, and let $\nil(R)$ be the set of all nilpotent elements of $R$; $\nil(R)=\{x\in R\mid x^{m}=0 \text{ for some }m\in \ZZ^{+}\}$. Let $x,y \in \nil(R)$. Then $x^{m}=0,y^{n}=0$ for some $m,n\in \ZZ^{+}$. Observe.
$$(x-y)^{mn}=\sum_{i=0}^{m}\binom{m}{i}x^i(-y)^{mn-i}=$$
\end{proof}

%%%%%%%%%%%%%%%%%%%%%%%%%%%%%%%%%%%%%%%%%%%%%%%%% 6 %%%%%%%%%%%%%%%%%%%%%%%%%%%%%%%%%%%%%%%%%%%%%%%%%

\begin{prob}
Let $n \in \NN$.
\begin{parts}
\part Show that if $n=ab$ for some integers $a,b$, then $ab$ is a nilpotent element of $\ZZ/n\ZZ$.
\part If $a \in \ZZ$, show that the element $a \in \ZZ/n\ZZ$ is nilpotent if and only if every prime divisor of $n$ divides $a$. In particular, determine the nilpotent elements of $\ZZ/72\ZZ$.
\end{parts}
\end{prob}

\begin{proof}

\end{proof}

%%%%%%%%%%%%%%%%%%%%%%%%%%%%%%%%%%%%%%%%%%%%%%%%% 7 %%%%%%%%%%%%%%%%%%%%%%%%%%%%%%%%%%%%%%%%%%%%%%%%%

\begin{prob}
Let $R$ and $S$ be rings.
\begin{parts}
\part Prove that the direct product $R \times S=\{(r,s)\mid r \in R,\ s \in S\}$ forms a ring under componentwise addition and multiplication.
\part Prove that $R \times S$ is commutative if and only if both $R$ and $S$ are commutative.
\end{parts}
\end{prob}

\begin{proof}

\end{proof}

%%%%%%%%%%%%%%%%%%%%%%%%%%%%%%%%%%%%%%%%%%%%%%%%% 8 %%%%%%%%%%%%%%%%%%%%%%%%%%%%%%%%%%%%%%%%%%%%%%%%%

\begin{prob}
Let $R$ be a commutative ring. Define the ring $R[[x]]$ of formal power series by
$$
R[[x]]=\left\{\sum_{n=0}^{\infty} a_n x^n \ \middle|\ a_i \in R \right\}.
$$
\begin{parts}
\part Prove that $R[[x]]$ is a commutative ring, and be sure to explain how to add and multiply elements.
\part Show that $1-x$ is a unit with inverse $1+x+x^2+\cdots=\sum_{n=0}^{\infty} x^n$.
\part (Optional Challenge) Prove that $\sum_{n=0}^{\infty} a_n x^n$ is a unit in $R[[x]]$ if and only if $a_0$ is a unit in $R$.
\end{parts}
\end{prob}

\begin{proof}

\end{proof}

%%%%%%%%%%%%%%%%%%%%%%%%%%%%%%%%%%%%%%%%%%%%%%%%% 9 %%%%%%%%%%%%%%%%%%%%%%%%%%%%%%%%%%%%%%%%%%%%%%%%%

\begin{prob}
Decide which of the following are ring homomorphisms from $M_2(\ZZ)$ to $\ZZ$:
\begin{parts}
\part $\begin{pmatrix} a & b \\ c & d \end{pmatrix} \mapsto a$
\part $\begin{pmatrix} a & b \\ c & d \end{pmatrix} \mapsto a+d$
\part $\begin{pmatrix} a & b \\ c & d \end{pmatrix} \mapsto ad-bc$
\end{parts}
\end{prob}

\begin{proof}

\end{proof}

%%%%%%%%%%%%%%%%%%%%%%%%%%%%%%%%%%%%%%%%%%%%%%%%% 10 %%%%%%%%%%%%%%%%%%%%%%%%%%%%%%%%%%%%%%%%%%%%%%%%

\begin{prob}
Let
$$
R=\left\{
\begin{pmatrix}
a & b \\
0 & d
\end{pmatrix}
\;\middle|\;
a,b,d \in \ZZ
\right\}.
$$
Prove that the map
$$
\varphi : R \to \ZZ \times \ZZ,\qquad
\varphi\!\left(
\begin{pmatrix}
a & b \\
0 & d
\end{pmatrix}
\right)=(a,d)
$$
is a surjective ring homomorphism and describe its kernel.
\end{prob}

\begin{proof}

\end{proof}

%%%%%%%%%%%%%%%%%%%%%%%%%%%%%%%%%%%%%%%%%%%%%%%%% 11 %%%%%%%%%%%%%%%%%%%%%%%%%%%%%%%%%%%%%%%%%%%%%%%%

\begin{prob}
Decide which of the following are ideals of $\ZZ \times \ZZ$:
\begin{parts}
\part $\{(a,a)\mid a \in \ZZ\}$
\part $\{(2a,2b)\mid a,b \in \ZZ\}$
\part $\{(2a,0)\mid a \in \ZZ\}$
\part $\{(a,-a)\mid a \in \ZZ\}$
\end{parts}
\end{prob}

\begin{proof}

\end{proof}

%%%%%%%%%%%%%%%%%%%%%%%%%%%%%%%%%%%%%%%%%%%%%%%%% 12 %%%%%%%%%%%%%%%%%%%%%%%%%%%%%%%%%%%%%%%%%%%%%%%%

\begin{prob}
The characteristic of a ring $R$ (denoted $\mathrm{char}\,R$) is the smallest $n \in \NN$ such that
$$
\underbrace{1_R+\cdots+1_R}_{n\ \text{times}}=0,
$$
and if there is no such $n$ we say the characteristic of $R$ is $0$.

Prove that an integral domain has characteristic $0$ or a prime.
\end{prob}

\begin{proof}

\end{proof}
%%%%%%%%%%%%%%%%%%%%%%%%%%%%%%%%%%%%%%%%%%%%%%%%%%%%%%%%%%%%%%%%%%%%%%%%%%%%%%%%%%%%%%%%%%%%%%%%%%%%%



\end{document}