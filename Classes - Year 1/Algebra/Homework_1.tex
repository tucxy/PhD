\documentclass[addpoints,10pt]{exam}

\usepackage{amsmath,amsthm,enumitem,wrapfig,amsfonts,mathtools}
\usepackage[mathscr]{euscript}
\usepackage[super]{nth}

\usepackage{geometry}
\usepackage[T1]{fontenc} % Use 8-bit encoding that has 256 glyphs
\renewcommand{\rmdefault}{ptm} %Change the Front Family from the default(cmr) to ptm(Times)
\usepackage{amsmath,amsfonts,amsthm,amssymb} % Math packages
\usepackage{bm}
\usepackage{mathptmx}
\usepackage{graphicx}
\usepackage{sectsty} % Allows customizing section commands
% \allsectionsfont{\centering} % Make all sections centered, the default font and small caps
\usepackage{pgfplots}
\pgfplotsset{compat=1.18}
\usetikzlibrary{arrows.meta}
\usepackage{xcolor}
\definecolor{darkpastelgreen}{rgb}{0.01, 0.75, 0.24}
\definecolor{blue-violet}{rgb}{0.54, 0.17, 0.89}

% Custom problem environment
\newcounter{cprob}
\newenvironment{cprob}[1]{%
    \setcounter{cprob}{#1}%
    \noindent\textbf{Problem \thecprob.}%
}{%
    \par\bigskip%
}

\theoremstyle{plain}
\newtheorem{thm}{\protect\theoremname}
  \theoremstyle{definition}
  \newtheorem{prob}[thm]{Problem}
  \newtheorem*{problem*}{Open Problem}
  \theoremstyle{plain}
  \newtheorem{conjecture}[thm]{Conjecture}
  \theoremstyle{plain}
  \newtheorem{lem}[thm]{Lemma}
  \newtheorem{obs}[thm]{Observation}
  \newtheorem{cor}[thm]{Corollary}
  \theoremstyle{definition}
\newtheorem{definition}[thm]{Definition}


% Patch prob environment to be single spaced
\let\oldprob\prob
\let\endoldprob\endprob
\renewenvironment{prob}
  {\begin{singlespace}\oldprob}
  {\endoldprob\end{singlespace}}

% start problem one line below like for enumerated problems with multiple parts
\newcommand{\belowtitle}{\leavevmode\newline}
%\Observe command
\newcommand{\Observe}{\underline{\textbf{Observe.}}}
%(=>)
\newcommand{\IF}{\mathbf{(\Rightarrow)}}
%(<=)
\newcommand{\FI}{\mathbf{(\Leftarrow)}}

\newcommand{\horrule}[1]{\rule{\linewidth}{#1}}
\newcommand{\kk}{\ensuremath{\Bbbk}} 
\newcommand{\CC}{\ensuremath{\mathbb{C}}}
\newcommand{\FF}{\ensuremath{\mathbb{F}}}
\newcommand{\NN}{\ensuremath{\mathbb{N}}}
\newcommand{\QQ}{\ensuremath{\mathbb{Q}}} 
\newcommand{\RR}{\ensuremath{\mathbb{R}}} 
\newcommand{\ZZ}{\ensuremath{\mathbb{Z}}}
\newcommand{\MM}{\ensuremath{\mathcal{M}}}
\newcommand{\TT}{\ensuremath{\mathcal{T}}}
\newcommand{\BB}{\ensuremath{\mathcal{B}}}
\newcommand{\VV}{\ensuremath{\mathcal{V}}}
\newcommand{\WW}{\ensuremath{\mathcal{W}}}
\newcommand{\UU}{\ensuremath{\mathcal{U}}}
\newcommand{\PP}{\ensuremath{\mathcal{P}}}
\newcommand{\LL}{\ensuremath{\mathcal{L}}}

\newcommand{\sm}{\char`\\}
%vector stuff
\DeclarePairedDelimiter{\ip}{\langle}{\rangle} %inner product/generate
\DeclarePairedDelimiter{\norm}{\lVert}{\rVert} %norm
\DeclarePairedDelimiter{\sqb}{\lbrack}{\rbrack} %corrd
\newcommand{\floor}[1]{\left\lfloor #1 \right\rfloor}
\newcommand{\ceil}[1]{\left\lceil #1 \right\rceil}
\newcommand{\mbf}[1]{\ensuremath{\mathbf{#1}}}
\newcommand{\tbf}[1]{\textbf{ #1 }}


\makeatletter
\renewcommand*\env@matrix[1][*\c@MaxMatrixCols c]{%
  \hskip -\arraycolsep
  \let\@ifnextchar\new@ifnextchar
  \array{#1}}
\makeatother

\def\env@matrix{\hskip -\arraycolsep
  \let\@ifnextchar\new@ifnextchar
  \array{*\c@MaxMatrixCols c}}

  \newcommand{\proj}[2]{\text{proj}_{#1}(#2)}
  

%%% Formatting: Page Header
\newcommand{\StudentName}{Danny Banegas}
\newcommand{\AssignmentName}{Homework 1}
\newcommand{\CourseName}{MATH 721 - Algebra II}


\pagestyle{headandfoot}
\runningheadrule
\firstpageheadrule
\firstpageheader{\CourseName}{\StudentName}{\AssignmentName}
\runningheader{\CourseName}{\StudentName}{\AssignmentName}
\firstpagefooter{}{\thepage}{}
\runningfooter{}{\thepage}{}

\printanswers

\DeclareMathAlphabet{\mathcal}{OMS}{cmsy}{m}{n}

\usepackage{parskip}
\usepackage{setspace}
\doublespacing
% % % % % % % % % % % % % % % % % % % % % % % % % % % % % % % % % % % % % % % % % % % % % % % % % % % % % % % % % % % % % % 
\begin{document}

Submit the following from the problem list: 2, 3, 5, 6, 8, 10, 11, 12, 13, 16, 23.

Submit two of the following : 19, 20, 21, 22, 24.

\begin{prob}
If $p$ is a prime number, prove that the nonzero elements of $\mathbb{Z}_p$ form a multiplicative group of order $p-1$. Show that this statement is false if $p$ is not a prime.
\end{prob}

    \begin{proof} Consider $\ZZ_{4}\setminus\{0\}=\{1,2,3\}$. $2(2)=0\not\in \ZZ_{4}\setminus\{0\}$, so closure doesn't hold and it can't be a group under multiplication at all. Therefore, the statement is false if $p$ is not prime. Now consider the statement for a prime $p$.
        
    $\ZZ_{2}=\{0,1\}$ and so $\ZZ^{*}_{2}=\{1\}$ is clearly a group under multiplication of order $2-1=1$. Now consider any prime $p>2$, which must be odd. $p=2k+1$ for some $k\in \ZZ^{+}$. \underline{\textbf{Observe.}}

    $\langle 2\rangle^{*}_{p}=\{2,4,\hdots, 2k\}\sqcup \{2(2k),\hdots\}$. Well, since $p=2k+1$, $2(2k)=4k=2k+2k=(2k+1)+(2k-1)=p+2k-1=2k-1=p-2$. So note that the elements following $2k$ must be odd since $p$ is odd. Additionally, $2q(p-2)=-4q=p-4q$ for $q=1,\hdots, k-1$ and finally note that $2(k-1)(p-2)=2(k-1)p-2(k-1)(2)=p-2k=1$. Therefore,

    $\langle 2\rangle^{*}_{p}=\{2,4,\hdots, 2k\}\sqcup \{2(2k),\hdots\}=\{2,4,\hdots, 2k\}\sqcup \{p-2,p-4\hdots,p-2k,\hdots\}=\{2,4,\hdots,p-1\}\sqcup \{p-2,p-4,\hdots,1,2,\hdots\}.$ and continuing in this fashion loops us back around to the evens.

    So, $\langle 2\rangle^{*}_{p}=(\mathcal{E}_{p}\setminus\{0\})\sqcup (\mathcal{O}_{p})=\ZZ^{*}_{p}$ must therefore be a cyclic multiplicative group of order $p-1$.

\end{proof}
\newpage
%%%%%%%%%%%%%%%%%%%%%%%%%%%%%%%%%%%%%%%%%%%%%%%%%%%%%%%%%%%%%%%%%%%%%%%%%%%%%%%%%%%%%%%%%%%%%%%%%%%%%%%%%%%%%%%%%%%%%%%%%%%%%%
\begin{prob}\belowtitle
\begin{enumerate}[label=(\alph*)]
\item Prove that the relation given by $a \sim b \iff a - b \in \mathbb{Z}$ is an equivalence relation on the additive group $\mathbb{Q}$.
\item Prove that $\mathbb{Q}/\mathbb{Z}$ is an infinite abelian group.
\end{enumerate}
\end{prob}

\begin{proof}\belowtitle
    (a)\quad For any $a,b,c\in (\QQ,+),$
    \begin{align*}
        &\mathbf{[a\sim a]:}\quad a-a=0\in \ZZ \implies a\sim a.\\
        &\mathbf{[a\sim b\implies b\sim a]:}\quad a\sim b\implies a-b\in \ZZ \implies -(a-b)=b-a\in \ZZ\implies  b\sim a.\\
        &\mathbf{[a\sim b,\,b\sim c\implies a\sim c]:}\quad a\sim b,\,b\sim c\implies c\sim b\implies (a-b)-(c-b)=a-c\in \ZZ\implies a\sim c.
    \end{align*}

    So $\sim$ is an equivalence relation on $(\QQ,+)$.

    (b)\quad $\QQ/\ZZ=\{[\frac{a}{b}]=\frac{a}{b}+\ZZ \mid a,b\in \ZZ\text{ and }b\nmid a\}$. Consider any $q_{1},q_{2}\in (0,1)$. If $[q_{1}]=[q_{2}],$ then $[q_{1}]-[q_{2}]=\ZZ$ and so $q_{1}-q_{1}\in \ZZ$. Well, $q_{1},q_{2}\in (0,1)$, so $q_{1}-q_{2}\in (-1,1)$ and therefore $q_{1}-q_{2}=0$. So $[q_{1}]=[q_{2}]\implies q_{1}=q_{2}$. On the other hand, $q_{1}=q_{2}\implies [q_{1}]=[q_{2}]$ by definition. So then
    $$q_{1}=q_{2}\iff [q_{1}]=[q_{2}],\;\forall q_{1},q_{2}\in (0,1).$$
    Since the rationals are dense in $\RR$, there are infinitely many distinct rationals in $(0,1)$ and infinitely many distinct cosets of the form $[q]$ where $q\in (0,1)$. Therefore, $\QQ/\ZZ$ is infinite. Lastly, since $(\QQ,+)$ is Abelian, so is $\QQ/\ZZ$ since $[q_{1}]+[q_{2}]=[q_{1}+q_{2}]=[q_{2}+q_{1}]=[q_{2}]+[q_{1}]$.

    Thus,
    \begin{center}
    $\QQ/\ZZ$ is an infinite Abelian group.
    \end{center}
\end{proof}
\newpage
%%%%%%%%%%%%%%%%%%%%%%%%%%%%%%%%%%%%%%%%%%%%%%%%%%%%%%%%%%%%%%%%%%%%%%%%%%%%%%%%%%%%%%%%%%%%%%%%%%%%%%%%%%%%%%%%%%%%%%%%%%%%%%
\begin{prob}
Let $p$ be a prime number and let $Z(p^\infty)$ be the following subset of the group $\mathbb{Q}/\mathbb{Z}$:
\[
\ZZ(p^\infty)=\left\{\, \frac{a}{b} \in \mathbb{Q}/\mathbb{Z} \;\middle|\; a,b\in\mathbb{Z},\ b=p^i \text{ for some } i \ge 0 \right\}.
\]
Prove that $\ZZ(p^\infty)$ is an infinite subgroup of $\mathbb{Q}/\mathbb{Z}$.
\end{prob}

\begin{proof}
    Clearly, $\ZZ(p^{\infty})\subset \QQ/\ZZ$. Consider some integers $i,j\geq 0$ and $a_{i},a_{j}\in \ZZ$. 
    \begin{align*}
    &\textbf{[Closure]:}\quad[\frac{a_{i}}{p^{i}}]+[\frac{a_{j}}{p^{i}}]=[\frac{p^{j}(a_{i})+p^{i}(a_{j})}{p^{i+j}}]\in \ZZ(p^{\infty}).\\
    &\textbf{[Inverses]:}\quad[\frac{-a_{i}}{p^{i}}]+[\frac{a_{i}}{p^{i}}]=[0]\implies -[\frac{a_{i}}{p^{i}}]=[\frac{-a_{i}}{p^{i}}].
    \end{align*}

    So $\ZZ(p^{\infty})\leq \QQ/\ZZ$. Now once more consider some integers $i,j\in \ZZ^{+}$ but set $a=1$. Notice that $\frac{1}{p^{i}},\frac{1}{p^{j}}\in (0,1)$. \underline{\textbf{Observe.}}

    This result essentially follows from \textbf{Problem 2}. $[\frac{1}{p^{i}}]=[\frac{1}{p^{j}}]\implies [\frac{1}{p^{i}}]-[\frac{1}{p^{j}}]=\ZZ\implies \frac{1}{p^{i}}-\frac{1}{p^{j}}\in \ZZ$. Well, $\frac{1}{p^{i}},\frac{1}{p^{j}}\in (0,1)\implies \frac{1}{p^{i}}-\frac{1}{p^{j}}\in (-1,1)\implies \frac{1}{p^{i}}-\frac{1}{p^{j}}=0\implies \frac{1}{p^{i}}=\frac{1}{p^{i}}\implies i=j$. On the other hand, $i=j\implies \frac{1}{p^{i}}=\frac{1}{p^{j}}\implies [\frac{1}{p^{i}}]=[\frac{1}{p^{j}}]$ by definition. So then,
    $$i=j\iff [\frac{1}{p^{i}}]=[\frac{1}{p^{j}}],\;\forall i,j\in \ZZ^{+}.$$
    There are infinitely many distinct positive integers so there must be infinitely many distinct cosets in $\ZZ(p^{\infty})$.

    Thus,
    \begin{center}
    $\ZZ(p^{\infty})$ is an infinite subgroup of $\QQ/\ZZ$.
    \end{center}
\end{proof}
\newpage
%%%%%%%%%%%%%%%%%%%%%%%%%%%%%%%%%%%%%%%%%%%%%%%%%%%%%%%%%%%%%%%%%%%%%%%%%%%%%%%%%%%%%%%%%%%%%%%%%%%%%%%%%%%%%%%%%%%%%%%%%%%%%%

\begin{prob}
If $G$ is a finite group of even order, prove that $G$ has an element of order two. 
\end{prob}

\begin{proof}
If $G$ is a finite group of even order, then $|G|=2k$ and $|G\setminus\{e\}|=2k-1$ for some $k\in \ZZ^{+}.$ Suppose there doesn't exist an element of order $2$ in $G$. Then, $\forall g\in G\setminus{e}$, $g\neq g^{-1}$. \Observe

If all non-identity elements are not equal to their inverse, then non-identity elements come two at a time. But then $|G\setminus \{e\}|=2k-1$ is even, a contradiction.

Thus,

\begin{center}
If $G$ is a finite group of even order, then it contains an element of order $2$.
\end{center}
\end{proof}

\begin{prob}
Let $Q_8$ be the multiplicative group generated by the complex matrices
\[
A=\begin{pmatrix}0 & 1 \\ -1 & 0\end{pmatrix},
\qquad
B=\begin{pmatrix}0 & i \\ i & 0\end{pmatrix}.
\]
Observe that $A^4=B^4=I_2$ and $BA=AB^3$. Prove that $Q_8$ is a group of order $8$.
\end{prob}

\begin{proof}
Well, 
\end{proof}

\begin{prob}
Let $G$ be a group and let $\operatorname{Aut}(G)$ denote the set of all automorphisms of $G$.
\begin{enumerate}[label=(\alph*)]
\item Prove that $\operatorname{Aut}(G)$ is a group with composition of functions as the binary operation.
\item Prove that $\operatorname{Aut}(\mathbb{Z}) \cong \mathbb{Z}_2,\ 
\operatorname{Aut}(\mathbb{Z}_6)\cong \mathbb{Z}_2,\
\operatorname{Aut}(\mathbb{Z}_8)\cong \mathbb{Z}_2 \times \mathbb{Z}_2,\
\operatorname{Aut}(\mathbb{Z}_p)\cong \mathbb{Z}_{p-1}\ (p\ \text{prime})$.
\end{enumerate}
\end{prob}

\begin{prob}
Let $G$ be an infinite group that is isomorphic to each of its proper subgroups. Prove that $G \cong \mathbb{Z}$.
\end{prob}

\begin{prob}
Let $G$ be the multiplicative group of $2\times 2$ invertible matrices with rational entries. Show that
\[
A=\begin{pmatrix}0 & -1 \\ 1 & 0\end{pmatrix},
\qquad
B=\begin{pmatrix}0 & 1 \\ -1 & -1\end{pmatrix}
\]
have finite orders but $AB$ has infinite order.
\end{prob}

\begin{prob}
Let $G$ be an abelian group containing elements $a$ and $b$ of orders $m$ and $n$, respectively. Prove that $G$ contains an element of order $\operatorname{lcm}(m,n)$.
\end{prob}

\begin{prob}
Let $H,K$ be subgroups of a group $G$. Prove that $HK$ is a subgroup of $G$ if and only if $HK=KH$.
\end{prob}

\begin{proof}\belowtitle
  $\IF\;HK\leq G\implies\text{For all }hk\in HK$, $(hk)^{-1}=k^{-1}h^{-1}\in HK$. Therefore, $HK=\{hk\mid h\in H, k\in K\}=\{k^{-1}h^{-1}\mid k\in K,h\in H\}=KH$.

  $\FI$ Note $HK=KH\implies \forall hk\in HK,\,\exists (h_{k_{1}},k_{h_{1}})\in H\times K, \text{ such that }hk=k_{h_{1}}h_{k_{1}}\in KH=HK$. The same logic holds for 'flipped' elements$\;\;kh\in KH=HK$. \Observe
  \begin{align*}
    &\textbf{[Closure]: } (h_{1}k_{1})(h_{2}k_{2})=(h_{1}k_{1})(k_{h_{2}}h_{k_{2}})=h_{1}(k_{1}k_{h_{2}})h_{k_{2}}=(k_{1}k_{h_{2}})_{h_{1}}h_{k_{1}k_{h_{2}}}h_{k_{2}}\in KH=HK.\\
    &\textbf{[Inverses]: }\text{For any }hk\in HK,\,(hk)^{-1}=k^{-1}h^{-1}\in KH=HK.
  \end{align*}
  So $HK\leq G$.

  Thus,
  $$HK\leq G\iff HK=KH.$$
\end{proof}
\newpage
%%%%%%%%%%%%%%%%%%%%%%%%%%%%%%%%%%%%%%%%%%%%%%%%%%%%%%%%%%%%%%%%%%%%%%%%%%%%%%%%%%%%%%%%%%%%%%%%%%%%%%%%%%%%%%%%%%%%%%%%%%%%%%
\begin{prob}
Let $H,K$ be subgroups of finite index of a group $G$ such that $[G:H]$ and $[G:K]$ are relatively prime. Prove that $G=HK$.
\end{prob}

\begin{proof}
$[G:H]=n,\,[G:K]=m$ for some coprime integers $m,n\in \ZZ^{+}$. If $G$ is finite, then without loss of generality $|K|<|H|$ and so $\frac{[G:K]}{[G:H]}=\frac{\frac{|G|}{|K|}}{\frac{|G|}{|H|}}=\frac{|H|}{|K|}$. Therefore, we must have that $[G:K]=|G/K|=|H|$ and $[G:H]=|G/H|=|K|$. So $|G|=[G:K]|K|=|H||K|$, and since $[G:H]=|K|$ and $[G:K]=|H|$ are coprime, we have that $H\cap K=\{e\}$. Otherwise a shared element $g\in H\cap K$ would have order $1<|g|$ which divides $|H|$ and $|G|$ and then $1<|g|\leq \gcd(|H|,|G|)$, a contradiction. Therefore, $|HK|=\frac{|H||K|}{|H\cap K|}=|H||K|=|G|$ and necessarily $HK=G$.

Now, if $G$ is infinite
\end{proof}

\begin{prob}
Let $H,K,N$ be subgroups of $G$ such that $H\subseteq N$. Prove that $HK\cap N = H(K\cap N)$.
\end{prob}

\begin{prob}
Let $H,K,N$ be subgroups of $G$ such that $H\subseteq K,\ H\cap N=K\cap N,\ HN=KN$. Prove that $H=K$.
\end{prob}

\begin{prob}
Let $H$ be a subgroup of $G$. For $a\in G$, prove that $aHa^{-1}$ is a subgroup of $G$ that is isomorphic to $H$.
\end{prob}

\begin{prob}
Let $G$ be a finite group and $H$ a subgroup of $G$ of order $n$. If $H$ is the only subgroup of $G$ of order $n$, prove that $H$ is normal in $G$.
\end{prob}

\begin{prob}
If $H$ is a cyclic normal subgroup of a group $G$, then every subgroup of $H$ is normal in $G$.
\end{prob}

\begin{prob}
What is $Z(S_n)$ for $n \ge 2$?
\end{prob}

\begin{prob}
If $H$ is a normal subgroup of $G$ such that $H$ and $G/H$ are finitely generated, then $G$ is finitely generated.
\end{prob}

\begin{prob}
If $N$ is a normal subgroup of $G$, $[G:N]$ is finite, $H$ is a subgroup of $G$, $|H|$ is finite, and $[G:N]$ and $|H|$ are relatively prime, then $H$ is a subgroup of $N$.
\end{prob}

\begin{prob}
If $N$ is a normal subgroup of $G$, $|N|$ is finite, $H$ is a subgroup of $G$, $[G:H]$ is finite, and $[G:H]$ and $|N|$ are relatively prime, then $N$ is a subgroup of $H$.
\end{prob}

\begin{prob}
If $G$ is a finite group and $H,K$ are subgroups of $G$, then
\[
[G:H\cap K] \le [G:H][G:H].
\]
\end{prob}

\begin{prob}
If $H,K,L$ are subgroups of a finite group $G$ such that $H\subseteq K$, then
\[
[K:H] \ge [L\cap K : L\cap H].
\]
\end{prob}

\begin{prob}
Let $H,K$ be subgroups of a group $G$. Assume that $H\cup K$ is a subgroup of $G$. Prove that either $H\subseteq K$ or $K\subseteq H$.
\end{prob}

\begin{prob}
Let $G$ be an abelian group, $H$ a subgroup of $G$ such that $G/H$ is an infinite cyclic group. Prove that $G \cong H \times G/H$.
\end{prob}

\end{document}