\documentclass[addpoints,10pt]{exam}

\usepackage{amsmath,amsthm,enumitem,wrapfig,amsfonts,mathtools}
\usepackage[mathscr]{euscript}
\usepackage[super]{nth}
\usepackage{dsfont}
\usepackage{xparse}

\usepackage{geometry}
\usepackage[T1]{fontenc} % Use 8-bit encoding that has 256 glyphs
\renewcommand{\rmdefault}{ptm} %Change the Front Family from the default(cmr) to ptm(Times)
\usepackage{amsmath,amsfonts,amsthm,amssymb} % Math packages
\usepackage{bm}
\usepackage{mathptmx}
\usepackage{graphicx}
\usepackage{sectsty} % Allows customizing section commands
% \allsectionsfont{\centering} % Make all sections centered, the default font and small caps
\usepackage{pgfplots}
\pgfplotsset{compat=1.18}
\usetikzlibrary{arrows.meta}
\usepackage{xcolor}
\definecolor{darkpastelgreen}{rgb}{0.01, 0.75, 0.24}
\definecolor{blue-violet}{rgb}{0.54, 0.17, 0.89}
%\usepackage{polylongdiv} %polynomial long division
% Custom problem environmenturther
\usepackage{polynom}
\newcounter{cprob}
\newenvironment{cprob}[1]{%
    \setcounter{cprob}{#1}%
    \noindent\textbf{Problem \thecprob.}%
}{%
    \par\bigskip%
}

\theoremstyle{plain}
\newcommand{\theoremname}{Theorem}
\newtheorem{thm}{\protect\theoremname}
  \theoremstyle{definition}
  \newtheorem{prob}[thm]{Problem}
  \newtheorem*{problem*}{Open Problem}
  \theoremstyle{plain}
  \newtheorem{conjecture}[thm]{Conjecture}
  \theoremstyle{plain}
  \newtheorem{lem}[thm]{Lemma}
  \newtheorem*{lem*}{Lemma}
  \newtheorem{obs}[thm]{Observation}
  \newtheorem{cor}[thm]{Corollary}
  \theoremstyle{definition}
\newtheorem{definition}[thm]{Definition}


% Patch prob environment to be single spaced
\let\oldprob\prob
\let\endoldprob\endprob
\renewenvironment{prob}
  {\begin{singlespace}\oldprob}
  {\endoldprob\end{singlespace}}

% start problem one line below like for enumerated problems with multiple parts
\newcommand{\belowtitle}{\leavevmode\newline}
%\Observe command
\newcommand{\Observe}{\text{Observe.}}
%(=>)
\newcommand{\IF}{\mathbf{(\Rightarrow)}}
%(<=)
\newcommand{\FI}{\mathbf{(\Leftarrow)}}
%equivalence classes; \class[S]{ *content in square brackets* }
\newcommand{\class}[2][]{\ensuremath{\left[\,#2\,\right]_{#1}}}

\newcommand{\horrule}[1]{\rule{\linewidth}{#1}}
\newcommand{\kkk}{\ensuremath{\Bbbk}} 
\newcommand{\CC}{\ensuremath{\mathbb{C}}}
\newcommand{\FF}{\ensuremath{\mathbb{F}}}
\newcommand{\KK}{\ensuremath{\mathbb{K}}}
\newcommand{\NN}{\ensuremath{\mathbb{N}}}
\newcommand{\QQ}{\ensuremath{\mathbb{Q}}} 
\newcommand{\RR}{\ensuremath{\mathbb{R}}} 
\newcommand{\ZZ}{\ensuremath{\mathbb{Z}}}
\newcommand{\MM}{\ensuremath{\mathcal{M}}}
\newcommand{\TT}{\ensuremath{\mathcal{T}}}
\newcommand{\BB}{\ensuremath{\mathcal{B}}}
\newcommand{\VV}{\ensuremath{\mathcal{V}}}
\newcommand{\WW}{\ensuremath{\mathcal{W}}}
\newcommand{\UU}{\ensuremath{\mathcal{U}}}
\newcommand{\PP}{\ensuremath{\mathcal{P}}}
\newcommand{\LL}{\ensuremath{\mathcal{L}}}
\newcommand{\kk}{\ensuremath{\mathds{k}}}
\newcommand{\EE}{\ensuremath{\mathbb{E}}}


\newcommand{\sm}{\char`\\}
%vector stuff
\DeclarePairedDelimiter{\ip}{\langle}{\rangle} %inner product/generate
\DeclarePairedDelimiter{\norm}{\lVert}{\rVert} %norm
\DeclarePairedDelimiter{\sqb}{\lbrack}{\rbrack} %corrd

\newcommand{\floor}[1]{\left\lfloor #1 \right\rfloor}
\newcommand{\ceil}[1]{\left\lceil #1 \right\rceil}
\newcommand{\mbf}[1]{\ensuremath{\mathbf{#1}}}
\newcommand{\tbf}[1]{\textbf{ #1 }}
\newcommand{\Span}{\ensuremath{\mathrm{Span}}}
\newcommand{\Char}[1]{\mathrm{Char}\; #1}
\DeclareMathOperator{\lcm}{lcm}
\newcommand{\id}{\ensuremath{\mathrm{id}}}
\newcommand{\Gal}[2]{\mathrm{Gal}(#1/#2)}
\newcommand{\Aut}{\ensuremath{\mathrm{Aut}}}
\newcommand{\Fix}[2]{\mathrm{Fix}_{#1}(#2)}
\newcommand{\nil}{\ensuremath{\mathrm{nil}}}
\newcommand{\Hom}{\ensuremath{\mathrm{Hom}}}
\newcommand{\Obj}{\ensuremath{\mathrm{Obj}}}

\newcommand{\Rng}{\ensuremath{\mathrm{Rng}}}
\newcommand{\Ring}{\ensuremath{\mathrm{Ring}}}




\makeatletter
\renewcommand*\env@matrix[1][*\c@MaxMatrixCols c]{%
  \hskip -\arraycolsep
  \let\@ifnextchar\new@ifnextchar
  \array{#1}}
\makeatother

\def\env@matrix{\hskip -\arraycolsep
  \let\@ifnextchar\new@ifnextchar
  \array{*\c@MaxMatrixCols c}}

  \newcommand{\proj}[2]{\text{proj}_{#1}(#2)}
  

%%% Formatting: Page Header
\newcommand{\StudentName}{Danny Banegas}
\newcommand{\AssignmentName}{Homework 2}
\newcommand{\CourseName}{MATH 720 - Algebra I}


\pagestyle{headandfoot}
\runningheadrule
\firstpageheadrule
\firstpageheader{\CourseName}{\StudentName}{\AssignmentName}
\runningheader{\CourseName}{\StudentName}{\AssignmentName}
\firstpagefooter{}{\thepage}{}
\runningfooter{}{\thepage}{}

\printanswers

\DeclareMathAlphabet{\mathcal}{OMS}{cmsy}{m}{n}

\usepackage{parskip}
\usepackage{setspace}

% % % % % % % % % % % % % % % % % % % % % % % % % % % % % % % % % % % % % % % % % % % % % % % % % % % % % % % % % % % % % % 
\begin{document}

%%%%%%%%%%%%%%%%%%%%%%%%%%%%%%%%%%%%%%%%%%%%%%%%% 0 %%%%%%%%%%%%%%%%%%%%%%%%%%%%%%%%%%%%%%%%
Problem List\footnote{%
\begin{minipage}{0.95\linewidth}
\begin{itemize}[leftmargin=*,nosep]
    \item \textbf{Sagan—Chapter 5:} 1(ac), 4, 5abc, 6a, 7bc, 8c, 9 (2 parts), 10a, 15 (1 poset not $C_n$), 17ab, 18ab, 19a (oneway), 20ab
    \item \textbf{Stanley—Chapter 3:} 7, 14a, 25, 30a, 34, 38, 33, 52, 62abc
    \item \textbf{Fulton—Chapter 1:} Exercises 1, 2 (pp.\ 15--16), compute product both ways
    \item \textbf{Fulton—Chapter 2:} Exercises 1, 2 (pp.\ 24--26)
\end{itemize}
\end{minipage}
}

\setcounter{thm}{0}
\renewcommand{\thethm}{5.\arabic{thm}}
 % next prob is 5.1

%%%%%%%%%%%%%%%%%%%%%%%%%%%%%%%%%%%%%%%% Sagan Ch. 5 %%%%%%%%%%%%%%%%%%%%%%%%%%%%%%%%%%%%%%%
\begin{center}
    \underline{\textbf{Sagan}}
\end{center}
%%%%%%%%%%%%%%%%%%%%%%%%%%%%%%%%%%%%%%%%%%%%%%%% 5.1 %%%%%%%%%%%%%%%%%%%%%%%%%%%%%%%%%%%%%%%

\begin{prob}
(\textbf{Sagan, Chapter 5, Exercise 1})
\belowtitle
This exercise refers to the list of examples just after the definition of a poset.
\begin{enumerate}[label=(\alph*)]
\item Verify that they satisfy the definition of a poset.
\item Show that the partial order in $\Pi_n$ is equivalent to defining $\rho \le \pi$ if every block of $\pi$ is a union of blocks of $\rho$.
\item Describe the cover relations in the list. For example, in $C_n$ the covers are of the form $i \prec i+1$ for $0 \le i < n$.
\end{enumerate}
\end{prob}

\begin{proof}
    \textbf{(a) Two Posets} 
    We prove that the set of all positive divisors of $n\in \ZZ$ with $\mid$:$(D_{n},\mid)$ and the set of all subspaces of a vector space $\VV$ over $\FF_{q}$ with subspace containment $\leq$: $(L(\VV), \leq)$ are posets. For any $a,b,c\in D_{n}$ and any $\UU_{1},\UU_{2},\UU_{3}\in L(\VV)$,
    \begin{align*}
        &(\text{reflexivity}) && a\mid a &&& \UU_{1}\leq \UU_{1}\\
        &(\text{antisymmetry}) && a\neq b\text{ and }a\mid b \implies a<b\implies b\nmid a &&& \UU_{1}<\UU_{2}\implies \UU_{2}\not\leq \UU_{1}\\
        &(\text{transitivity}) && a\mid b \mid c\implies \exists k,q\in \ZZ \text{ s.t. }\substack{ak=b\\ bq=c \\ \implies akq=c} \implies a\mid c &&& \UU_{1}\leq \UU_{2}\leq \UU_{3}\implies \UU_{1}\leq \UU_{3}
    \end{align*}
    \textbf{(c)}
    In $D_{n}$, if $|n|=\prod_{i} p_{i}^{a_{i}}$ is a prime decomposition, then each positive divisor is of the form $d=\prod_{i}p_{i}^{b_{i}}$ where $0 \leq b_{i}\leq a_{i}$ is some 'subdecomposition'. So then each cover relation just looks like $$\prod_{i}p_{i}^{b_{i}}\mid \prod_{i}p_{i}^{\beta_{i}}\text{ where each }\beta_{i}\geq b_{i} \text{ and }\sum_{i}(\beta_{i}-b_{i})=1$$
    In $L(\VV)$, cover relations look like
    \begin{center}
        $\WW \leq \UU$ and $\BB_{\UU}=\BB_{\WW}\sqcup \{b\}$ is a basis for $\UU$ and $b\in \UU$.
    \end{center}
    That is, $\WW \leq  \UU$ and $\dim \UU = \dim \WW+1$.
    
\end{proof}

%%%%%%%%%%%%%%%%%%%%%%%%%%%%%%%%%%%%%%%%%%%%%%%% 5.4 %%%%%%%%%%%%%%%%%%%%%%%%%%%%%%%%%%%%%%%

\setcounter{thm}{3}
\newpage
\begin{prob}
(\textbf{Sagan, Chapter 5, Exercise 4})
\belowtitle
Complete the proof of Proposition 5.1.3. To show that $K_n \cong B_{n-1}$ it may be simpler to show that $K_n \cong B_{n-1}^*$ using the map $\phi$ from Section 1.7.
\end{prob}

\begin{proof}

\end{proof}

\begin{prob}
(\textbf{Sagan, Chapter 5, Exercise 5})
\belowtitle
Let $f : P \to Q$ be an isomorphism of posets.
\begin{enumerate}[label=(\alph*)]
\item Show that $f$ is also an isomorphism of $P^*$ with $Q^*$.
\item Show that if $P$ has a $\hat{0}$, then so does $Q$.
\item Show in two ways that if $P$ has a $\hat{1}$, then so does $Q$: by mimicking the proof of part (b) and by using the result of (b) together with part (a).
\end{enumerate}
\end{prob}

\begin{prob}
(\textbf{Sagan, Chapter 5, Exercise 6})
\belowtitle
\begin{enumerate}[label=(\alph*)]
\item Show that the axioms for a partially ordered set are satisfied by $P \sqcup Q$, $P + Q$, and $P \times Q$.
\end{enumerate}
\end{prob}

\begin{prob}
(\textbf{Sagan, Chapter 5, Exercise 7})
\belowtitle
Complete the proof of Proposition 5.2.1.
\end{prob}

\begin{prob}
(\textbf{Sagan, Chapter 5, Exercise 8})
\belowtitle
\begin{enumerate}[label=(\alph*)]
\item Show that if $P$ is a ranked poset, then for any $k$ we have $R_k(P)$ is an antichain.
\item Let $P$ be a ranked poset and assume $f : P \to Q$ is an isomorphism. Show that $Q$ is also ranked and for all $x \in P$ we have $\mathrm{rk}_P x = \mathrm{rk}_Q f(x)$.
\item Show that if $P,Q$ are ranked posets, then so is $P \times Q$ with rank function
\[
\mathrm{rk}_{P\times Q}(x,y) = \mathrm{rk}_P x + \mathrm{rk}_Q y.
\]
\end{enumerate}
\end{prob}

\begin{prob}
(\textbf{Sagan, Chapter 5, Exercise 9})
\belowtitle
Prove Proposition 5.2.2.
\end{prob}

\begin{prob}
(\textbf{Sagan, Chapter 5, Exercise 10})
\belowtitle
\begin{enumerate}[label=(\alph*)]
\item Prove Proposition 5.3.1.
\end{enumerate}
\end{prob}

\begin{prob}
(\textbf{Sagan, Chapter 5, Exercise 15})
\belowtitle
Prove Proposition 5.3.4.
\end{prob}

\begin{prob}
(\textbf{Sagan, Chapter 5, Exercise 17})
\belowtitle
Let $P$ be a finite poset and let $L = J(P)$ be the corresponding distributive lattice.  
If $X \subseteq P$ is a lower-order ideal, then use the corresponding lowercase letter $x$ to denote the associated element of $L$.
\begin{enumerate}[label=(\alph*)]
\item Show that $x$ covers $y$ in $L$ if and only if $Y = X - \{m\}$ where $m$ is a maximal element of $X$.
\item Show that $x$ is join irreducible in $L$ if and only if $X$ is a principal ideal of $P$.
\end{enumerate}
\end{prob}

\begin{prob}
(\textbf{Sagan, Chapter 5, Exercise 18})
\belowtitle
Given a poset $P$, let $\mathcal{A}(P)$ be the set of antichains of $P$.  
Show that the map
\[
f : \mathcal{A}(P) \to J(P)
\]
given by $f(A) = I(A)$ (where $I(A)$ is the order ideal generated by $A$) is a bijection.
\end{prob}

\begin{prob}
(\textbf{Sagan, Chapter 5, Exercise 19})
\belowtitle
\begin{enumerate}[label=(\alph*)]
\item Rederive the formula for $\mu$ in $B_n$, equation (5.6), in two ways: by mimicking the proof of (5.7) and by constructing an $m \in P$ such that $D_m = B_n$ and then applying (5.7).
\end{enumerate}
\end{prob}

\begin{prob}
(\textbf{Sagan, Chapter 5, Exercise 20})
\belowtitle
\begin{enumerate}[label=(\alph*)]
\item Let $P$ be a locally finite poset with a $\hat{0}$. Show that if $x$ covers exactly one element of $P$, then
\[
\mu(x) =
\begin{cases}
-1 & \text{if $x$ covers $\hat{0}$},\\
0 & \text{otherwise}.
\end{cases}
\]
\item Given any $n \in \ZZ$, construct a poset containing an element $x$ with $\mu(x)=n$.
\end{enumerate}
\end{prob}

\end{document}