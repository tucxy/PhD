\documentclass[addpoints,10pt]{exam}

\usepackage{amsmath,amsthm,enumitem,wrapfig,amsfonts,mathtools}
\usepackage[mathscr]{euscript}
\usepackage[super]{nth}
\usepackage{dsfont}
\usepackage{xparse}

\usepackage{geometry}
\usepackage[T1]{fontenc} % Use 8-bit encoding that has 256 glyphs
\renewcommand{\rmdefault}{ptm} %Change the Front Family from the default(cmr) to ptm(Times)
\usepackage{amsmath,amsfonts,amsthm,amssymb} % Math packages
\usepackage{bm}
\usepackage{mathptmx}
\usepackage{graphicx}
\usepackage{sectsty} % Allows customizing section commands
% \allsectionsfont{\centering} % Make all sections centered, the default font and small caps
\usepackage{pgfplots}
\pgfplotsset{compat=1.18}
\usetikzlibrary{arrows.meta}
\usepackage{xcolor}
\definecolor{darkpastelgreen}{rgb}{0.01, 0.75, 0.24}
\definecolor{blue-violet}{rgb}{0.54, 0.17, 0.89}
%\usepackage{polylongdiv} %polynomial long division
% Custom problem environment
\usepackage{polynom}
\newcounter{cprob}
\newenvironment{cprob}[1]{%
    \setcounter{cprob}{#1}%
    \noindent\textbf{Problem \thecprob.}%
}{%
    \par\bigskip%
}

\theoremstyle{plain}
\newcommand{\theoremname}{Theorem}
\newtheorem{thm}{\protect\theoremname}
  \theoremstyle{definition}
  \newtheorem{prob}[thm]{Problem}
  \newtheorem*{problem*}{Open Problem}
  \theoremstyle{plain}
  \newtheorem{conjecture}[thm]{Conjecture}
  \theoremstyle{plain}
  \newtheorem{lem}[thm]{Lemma}
  \newtheorem*{lem*}{Lemma}
  \newtheorem{obs}[thm]{Observation}
  \newtheorem{cor}[thm]{Corollary}
  \theoremstyle{definition}
\newtheorem{definition}[thm]{Definition}


% Patch prob environment to be single spaced
\let\oldprob\prob
\let\endoldprob\endprob
\renewenvironment{prob}
  {\begin{singlespace}\oldprob}
  {\endoldprob\end{singlespace}}

% start problem one line below like for enumerated problems with multiple parts
\newcommand{\belowtitle}{\leavevmode\newline}
%\Observe command
\newcommand{\Observe}{\text{Observe.}}
%(=>)
\newcommand{\IF}{\mathbf{(\Rightarrow)}}
%(<=)
\newcommand{\FI}{\mathbf{(\Leftarrow)}}
%equivalence classes; \class[S]{ *content in square brackets* }
\newcommand{\class}[2][]{\ensuremath{\left[\,#2\,\right]_{#1}}}

\newcommand{\horrule}[1]{\rule{\linewidth}{#1}}
\newcommand{\kkk}{\ensuremath{\Bbbk}} 
\newcommand{\CC}{\ensuremath{\mathbb{C}}}
\newcommand{\FF}{\ensuremath{\mathbb{F}}}
\newcommand{\KK}{\ensuremath{\mathbb{K}}}
\newcommand{\NN}{\ensuremath{\mathbb{N}}}
\newcommand{\QQ}{\ensuremath{\mathbb{Q}}} 
\newcommand{\RR}{\ensuremath{\mathbb{R}}} 
\newcommand{\ZZ}{\ensuremath{\mathbb{Z}}}
\newcommand{\MM}{\ensuremath{\mathcal{M}}}
\newcommand{\TT}{\ensuremath{\mathcal{T}}}
\newcommand{\BB}{\ensuremath{\mathcal{B}}}
\newcommand{\VV}{\ensuremath{\mathcal{V}}}
\newcommand{\WW}{\ensuremath{\mathcal{W}}}
\newcommand{\UU}{\ensuremath{\mathcal{U}}}
\newcommand{\PP}{\ensuremath{\mathcal{P}}}
\newcommand{\LL}{\ensuremath{\mathcal{L}}}
\newcommand{\kk}{\ensuremath{\mathds{k}}}
\newcommand{\EE}{\ensuremath{\mathbb{E}}}


\newcommand{\sm}{\char`\\}
%vector stuff
\DeclarePairedDelimiter{\ip}{\langle}{\rangle} %inner product/generate
\DeclarePairedDelimiter{\norm}{\lVert}{\rVert} %norm
\DeclarePairedDelimiter{\sqb}{\lbrack}{\rbrack} %corrd

\newcommand{\floor}[1]{\left\lfloor #1 \right\rfloor}
\newcommand{\ceil}[1]{\left\lceil #1 \right\rceil}
\newcommand{\mbf}[1]{\ensuremath{\mathbf{#1}}}
\newcommand{\tbf}[1]{\textbf{ #1 }}
\newcommand{\Span}{\ensuremath{\mathrm{Span}}}
\newcommand{\Char}[1]{\mathrm{Char}\; #1}
\DeclareMathOperator{\lcm}{lcm}
\newcommand{\id}{\ensuremath{\mathrm{id}}}
\newcommand{\Gal}[2]{\mathrm{Gal}(#1/#2)}
\newcommand{\Aut}{\ensuremath{\mathrm{Aut}}}
\newcommand{\Fix}[2]{\mathrm{Fix}_{#1}(#2)}
\newcommand{\nil}{\ensuremath{\mathrm{nil}}}
\newcommand{\Hom}{\ensuremath{\mathrm{Hom}}}
\newcommand{\Obj}{\ensuremath{\mathrm{Obj}}}

\newcommand{\Rng}{\ensuremath{\mathrm{Rng}}}
\newcommand{\Ring}{\ensuremath{\mathrm{Ring}}}









\makeatletter
\renewcommand*\env@matrix[1][*\c@MaxMatrixCols c]{%
  \hskip -\arraycolsep
  \let\@ifnextchar\new@ifnextchar
  \array{#1}}
\makeatother

\def\env@matrix{\hskip -\arraycolsep
  \let\@ifnextchar\new@ifnextchar
  \array{*\c@MaxMatrixCols c}}

  \newcommand{\proj}[2]{\text{proj}_{#1}(#2)}
  

%%% Formatting: Page Header
\newcommand{\StudentName}{Danny Banegas}
\newcommand{\AssignmentName}{Homework 1}
\newcommand{\CourseName}{MATH 720 - Algebra I}


\pagestyle{headandfoot}
\runningheadrule
\firstpageheadrule
\firstpageheader{\CourseName}{\StudentName}{\AssignmentName}
\runningheader{\CourseName}{\StudentName}{\AssignmentName}
\firstpagefooter{}{\thepage}{}
\runningfooter{}{\thepage}{}

\printanswers

\DeclareMathAlphabet{\mathcal}{OMS}{cmsy}{m}{n}

\usepackage{parskip}
\usepackage{setspace}

% % % % % % % % % % % % % % % % % % % % % % % % % % % % % % % % % % % % % % % % % % % % % % % % % % % % % % % % % % % % % % 
\begin{document}
%%%%%%%%%%%%%%%%%%%%%%%%%%%%%%%%%%%%%%%%%%%%%%%%% 1 %%%%%%%%%%%%%%%%%%%%%%%%%%%%%%%%%%%%%%%%
\setcounter{thm}{0} % next prob is 1

\begin{prob}
Prove that if $0=1$ in a ring $R$, then $R$ is the zero ring (that is, its only element is $0$).
\end{prob}

\begin{proof}
$\forall r\in R$, $r=1\cdot r=0\cdot r=0\implies R=\{0\}$.

\end{proof}

%%%%%%%%%%%%%%%%%%%%%%%%%%%%%%%%%%%%%%%%%%%%%%%%% 2 %%%%%%%%%%%%%%%%%%%%%%%%%%%%%%%%%%%%%%%%%%%%%%%%%

\begin{prob}
A category $C'$ is a subcategory of a category $C$ if
\begin{enumerate}[label=(\arabic*)]
  \item $\mathrm{Obj}(C') \subseteq \mathrm{Obj}(C)$, and
  \item for all $A,B \in \mathrm{Obj}(C')$, $\mathrm{Hom}_{C'}(A,B) \subseteq \mathrm{Hom}_{C}(A,B)$.
\end{enumerate}
It is a full subcategory if in addition $\mathrm{Hom}_{C'}(A,B)=\mathrm{Hom}_{C}(A,B)$ for all $A,B \in \mathrm{Obj}(C')$.

Ring with unity is a subcategory of Ring (you do not need to prove this). Show that it is not a full subcategory.
\end{prob}

\begin{proof}
$\varphi:\ZZ_{6}\mapsto \ZZ_{6}$ defined via $\varphi(a)=[0]a$ belongs to $\mathrm{Hom}_{\mathrm{Ring}}(\ZZ_{6},\ZZ_{6})$ but not $\mathrm{Hom}_{\mathrm{Ring\;with\;unity}}(\ZZ_{6},\ZZ_{6})$ since $\varphi([1])=[0]\neq [1]$, and a ring homomorphism in $\mathrm{Hom}_{\mathrm{Ring\;with\;unity}}(\ZZ_{6},\ZZ_{6})$ must preserve unity. So $\Hom_{\mathrm{Ring\;with\;unity}}(A,B)\neq \Hom_{\mathrm{Ring}}(A,B)$ for $A=B=\ZZ_{6}\in \mathrm{Obj}(\mathrm{Ring\;with\;unity})$.

Thus, 
\begin{center}
Ring with unity is not a full subcategory of Ring.
\end{center}
\end{proof}

%%%%%%%%%%%%%%%%%%%%%%%%%%%%%%%%%%%%%%%%%%%%%%%%% 3 %%%%%%%%%%%%%%%%%%%%%%%%%%%%%%%%%%%%%%%%%%%%%%%%%

\begin{prob}
Let $a$ and $b$ be zero-divisors in a ring $R$. Either prove $a+b$ is always a zero-divisor or provide a specific counterexample.
\end{prob}

\begin{proof}
Consider $[2],[3]\in \ZZ_{6}$. $[2]\cdot [3]=[0]$ and both are non-zero, so they are both zero-divisors. Yet $[2]+[3]=[5]$, which isn't a zero divisor since $([5][m])_{m=0}^{5}=([0],[5],[4],[3],[2],[1])$. 

Thus, 
\begin{center}
the sum of two zero-divisors is not always a zero-divisor.
\end{center}
\end{proof}
%%%%%%%%%%%%%%%%%%%%%%%%%%%%%%%%%%%%%%%%%%%%%%%%% 4 %%%%%%%%%%%%%%%%%%%%%%%%%%%%%%%%%%%%%%%%%%%%%%%%%
\newpage
\begin{prob}
The center of a ring $R$ is $Z(R)=\{z \in R \mid rz=zr \text{ for all } r \in R\}$. Prove that the center of a ring $R$ is a subring of $R$.
\end{prob}

\begin{proof}
$\forall a,b \in Z(R)$, and $\forall r\in R$,
\begin{align*}
    &\mathbf{[1]:} \; 1r=r1=r\implies 1\in Z(R)\\
    &\mathbf{[\leq_{+}]:} \; (a-b)r=ar-br=(ra)-(rb)=r(a-b)\implies a-b\in Z(R)\\
    &\mathbf{[\cdot]:} \; (ab)r=a(br)=a(rb)=(ar)b=(ra)b=r(ab)\implies ab\in Z(R). 
\end{align*}
Thus, 
\begin{center}
$Z(R)$ is a subring of $R$.
\end{center}
\end{proof}

%%%%%%%%%%%%%%%%%%%%%%%%%%%%%%%%%%%%%%%%%%%%%%%%% 5 %%%%%%%%%%%%%%%%%%%%%%%%%%%%%%%%%%%%%%%%%%%%%%%%%
\newpage
\begin{prob}
An element $x$ in a ring $R$ is called nilpotent if $x^m=0$ for some $m \in \ZZ^{+}$ (here $x^m$ denotes $x\cdot x \cdots x$ ($m$ times)).

Prove that the nilpotent elements of a commutative ring $R$ form an ideal (this is called the nilradical of $R$).
\end{prob}

\begin{proof}
Let $R$ be a commutative ring with unity, and let $\nil(R)$ be the set of all nilpotent elements of $R$; $\nil(R)=\{x\in R\mid x^{m}=0 \text{ for some }m\in \ZZ^{+}\}$. Notice that in general for $r\in \nil(R)$: if $r^{N}=0$, then $r^{N+k}=r^{N}r^{k}=(0)r^{k}=0\implies r^{j}=0,\forall j\geq N$.

Now let $x,y \in \nil(R)$. Then $x^{m}=0,y^{n}=0$ for some $m,n\in \ZZ^{+}$. By the Binomial Theorem,
\begin{center}
  $(x-y)^{m+n}=\sum_{i=0}^{m+n}\binom{m+n}{i}x^i(-y)^{m+n-i}=\sum_{i=0}^{m+n}(-1)^{m+n-i}\binom{m+n}{i}x^i(y)^{m+n-i}=0$

  Since $i\geq m\implies x^{i}=0$ or $0\leq i<m\implies m+n-i\geq n\implies y^{m+n-i}=0$
\end{center}
So $x-y\in \nil(R)$, and therefore it is an additive subgroup of $R$.

Since $R$ is commutative, $(rx)^{k}=r^{k}x^{k}$ For all $k\geq 1$. So then obviously $(rx)^{m}=r^{m}x^{m}=r^{m}(0)=0\implies rx=xr\in \nil(R)$.

Thus,
$$\text{The nilradical }\nil(R)\text{ of R is a two-sided ideal of }R.$$
\end{proof}

%%%%%%%%%%%%%%%%%%%%%%%%%%%%%%%%%%%%%%%%%%%%%%%%% 6 %%%%%%%%%%%%%%%%%%%%%%%%%%%%%%%%%%%%%%%%%%%%%%%%%
\newpage
\begin{prob}
Let $n \in \NN$.
\begin{parts}
\part Show that if $n=ab$ for some integers $a,b$, then $ab$ is a nilpotent element of $\ZZ/n\ZZ$.
\part If $a \in \ZZ$, show that the element $a \in \ZZ/n\ZZ$ is nilpotent if and only if every prime divisor of $n$ divides $a$. In particular, determine the nilpotent elements of $\ZZ/72\ZZ$.
\end{parts}
\end{prob}
We quickly prove a lemma, which follows from \textbf{Euclid's Lemma}.
\setcounter{thm}{0} % next prob is 1
\begin{lem}
If $p$ is prime, $a\in \ZZ$, and $n\in \ZZ^{+}$ such that $p\mid a^{n}$, then $p\mid a$.
\end{lem}
\begin{proof}
Trivially, $p\mid a\implies p\mid a$. Next, $p\mid a^{2}=a(a)\implies p\mid a$ or $p\mid a\implies p\mid a$ by \textbf{Euclid's Lemma}. Now suppose that $p\mid a^{k}\implies p\mid a$ for some $k\geq 2$. Then, by \textbf{Euclid's Lemma}, $p\mid a^{k+1}=a(a^{k})\implies (i) p\mid a$ or $(ii) p\mid a^{k}$ and so $(i)$ $p\mid a$ or by our inductive step $(ii) p\mid a^{k}\implies p\mid a$. So $p\mid a^{k+1}\implies p\mid a$. Therefore, by induction, $p\mid a^{n}\implies p\mid a$ for all $n\geq 1$, and the statement is proven.
\end{proof}
\setcounter{thm}{6} % next prob is 7
\begin{proof}
\textbf{(a): }Since $ab=n\implies (ab)^{1}=ab=n=0\in \ZZ_{n} \implies ab\in \nil(\ZZ_{n})$.

\textbf{(b): $(\implies)$ }If $a\in \ZZ$ is nilpotent in $\ZZ_{n}$, then $a^{m}=0$ for some $m\in \ZZ^{+}$. So then $a^{m}=0\in \ZZ_{n}\implies a^{m}=qn\in \ZZ$ for some $q\in \ZZ$ and $n\mid a^{m}$. Well, for any prime divisor $p$ of $n$, we must then have that $p\mid n$ and $n\mid a^{m}\implies p\mid a^{m}$. Therefore by \textbf{Lemma 1} $p\mid a$.

$(\impliedby)$ Let $n=\prod_{i=1}^{k} p_{i}^{b_{i}}$ be the prime decomposition of $n$ where $p_{1},\hdots, p_{k}$ are all distinct prime divisors of $n$, and $b_{1},\hdots, b_{k}\in \ZZ^{+}$. If each prime divisor $p_{i}$ divides $a$, then $\exists q_{i}\in \ZZ$ such that $a=p_{i}q_{i}$ and $a^{b_{i}}=p_{i}^{b_{i}}q_{i}^{b_{i}}$for all $1\leq i\leq k$. Therefore, since $\ZZ$ is commutative,
$$a^{\sum_{i=1}^{k}b_{i}}=\prod_{i=1}^{k} a^{b_{i}}=\prod_{i=1}^{k}p_{i}^{b_{i}}q_{i}^{b_{i}}=\prod_{i=1}^{k}q_{i}(\prod_{i=1}^{k}p_{i}^{b_{i}})=(\prod_{i=1}^{k}q_{i})n\implies a^{\sum_{i=1}^{k}b_{i}}=0\in \ZZ_{n}\implies a\in \nil(\ZZ_{n}).$$
Thus, 
\begin{center}
$a\in \ZZ$ is nilpotent in $\ZZ_{n} \text{ for }n\in \NN \iff$ every prime divisor of $n$ divides $a$.
\end{center}
\end{proof}
$72=9(8)=2^{3}3^{2}$ and so the nilpotent elements of $\ZZ_{72}$ are all $a\in \ZZ_{72}$ which are divisible by both $2$ and $3$. That is, all multiples of $6$ in $\ZZ_{72}$. So $\nil(\ZZ_{72})=\langle 6\rangle=\{0,6,12,18,24,30,36,42,48,54,60,66\}$.
%%%%%%%%%%%%%%%%%%%%%%%%%%%%%%%%%%%%%%%%%%%%%%%%% 7 %%%%%%%%%%%%%%%%%%%%%%%%%%%%%%%%%%%%%%%%%%%%%%%%%
\newpage
\begin{prob}
Let $R$ and $S$ be rings.
\begin{parts}
\part Prove that the direct product $R \times S=\{(r,s)\mid r \in R,\ s \in S\}$ forms a ring under componentwise addition and multiplication.
\part Prove that $R \times S$ is commutative if and only if both $R$ and $S$ are commutative.
\end{parts}
\end{prob}
\begin{proof}
\textbf{(a):} For all $a=(r_{1},s_{1}),b=(r_{2},s_{2}),c=(r_{3},s_{3})\in R\times S,$
we verify the ring axioms.
\begin{align*}
&[+_1]\ \text{(associativity)}:\quad (a+b)+c
= (r_1+r_2,\; s_1+s_2)+(r_3,s_3) \\
&\hspace{40mm}=((r_1+r_2)+r_3,\; (s_1+s_2)+s_3)
=(r_1+(r_2+r_3),\; s_1+(s_2+s_3)) \\
&\hspace{40mm}=(r_1,s_1)+(r_2+r_3,\; s_2+s_3)=a+(b+c).
\\[2mm]
&[+_2]\ \text{(commutativity)}:\quad a+b=(r_1+r_2,\; s_1+s_2)=(r_2+r_1,\; s_2+s_1)=b+a.
\\[2mm]
&[0]\ \text{(zero)}:\quad a+(0_{R},0_{S})=(r_1+0_R, s_1+0_S)=(0_R+r_{1},0_S+s_{1})=(0_{R},0_{S})+a=a\implies 0_{R\times S}=(0_{R},0_{S}).
\\[2mm]
&[+_3]\ \text{(additive inverse)}:\quad \text{Let }-a=(-r_1,-s_1).
\ \text{Then }a+(-a)=(r_1-r_1,\; s_1-s_1)=(0_R,0_S)=0_{R\times S}.
\\[3mm]
&[\cdot_1]\ \text{(associativity)}:\quad (ab)c
=(r_1r_2,\; s_1s_2)(r_3,s_3)
=((r_1r_2)r_3,\; (s_1s_2)s_3)\\
&\hspace{40mm}=(r_1(r_2r_3),\; s_1(s_2s_3))
=(r_1,s_1)(r_2r_3,\; s_2s_3)=a(bc).
\\[2mm]
&[D_1]\ \text{(left distributivity)}:\quad a(b+c)
=(r_1,s_1)(r_2+r_3,\; s_2+s_3)
=(r_1(r_2+r_3),\; s_1(s_2+s_3))\\
&\hspace{40mm}=(r_1r_2+r_1r_3,\; s_1s_2+s_1s_3)
=(r_1r_2,\; s_1s_2)+(r_1r_3,\; s_1s_3)=ab+ac.
\\[2mm]
&[D_2]\ \text{(right distributivity)}:\quad (a+b)c
=(r_1+r_2,\; s_1+s_2)(r_3,s_3)
=((r_1+r_2)r_3,\; (s_1+s_2)s_3)\\
&\hspace{40mm}=(r_1r_3+r_2r_3,\; s_1s_3+s_2s_3)
=(r_1r_3,\; s_1s_3)+(r_2r_3,\; s_2s_3)=ac+bc.
\\[2mm]
&[1]\ \text{(unity)}:\quad (1_{R},1_{S})a=(1_{R}\cdot r_{1},1_{S}\cdot s_{1})=(r_{1}\cdot 1_{R},s_{1}\cdot 1_{S})=(r_{1},s_{1})=a\implies 1_{R\times S}=(1_{R},1_{S}).
\end{align*}

Thus,
\begin{center}
$R\times S$ is a ring with unity.
\end{center}
\end{proof}

\begin{proof} \textbf{(b):} $(\implies)$ If $R\times S$ is commutative, then for all $r_{1},r_{2}\in R$ and all $s_{1},s_{2}\in S$, $(r_{1},s_{1})(r_{2},s_{2})=(r_{1}r_{2},s_{1}s_{2})=(r_{2},s_{2})(r_{1},s_{1})=(r_{2}r_{1},s_{2}s_{1})$. So then for all $r_{1},r_{2}\in R$, $s_{1},s_{2}\in S$, we have that $r_{1}r_{2}=r_{2}r_{1}$ and $s_{1}s_{2}=s_{2}s_{1}$, since such pairs are only equal if their components are equal. Therefore, $R$ and $S$ are both commutative.

$(\impliedby)$ If $R$ and $S$ are both commutative, then for all $r_{1},r_{2}\in R$ and all $s_{1},s_{2}\in S$, $(r_{1},s_{1})(r_{2},s_{2})=(r_{1}r_{2},s_{1}s_{2})=(r_{2}r_{1},s_{2}s_{1})=(r_{2},s_{2})(r_{1},s_{1})$. So $R\times S$ is commutative.

\end{proof}

%%%%%%%%%%%%%%%%%%%%%%%%%%%%%%%%%%%%%%%%%%%%%%%%% 8 %%%%%%%%%%%%%%%%%%%%%%%%%%%%%%%%%%%%%%%%%%%%%%%%%

\begin{prob}
Let $R$ be a commutative ring. Define the ring $R[[x]]$ of formal power series by
$$
R[[x]]=\left\{\sum_{n=0}^{\infty} a_n x^n \ \middle|\ a_i \in R \right\}.
$$
\begin{parts}
\part Prove that $R[[x]]$ is a commutative ring, and be sure to explain how to add and multiply elements.
\part Show that $1-x$ is a unit with inverse $1+x+x^2+\cdots=\sum_{n=0}^{\infty} x^n$.
\part (Optional Challenge) Prove that $\sum_{n=0}^{\infty} a_n x^n$ is a unit in $R[[x]]$ if and only if $a_0$ is a unit in $R$.
\end{parts}
\end{prob}

\begin{proof}
\textbf{(a):} For all $A(x)=\sum_{i=0}^{\infty} a_{i}x^{i}, B(x)=\sum_{j=0}^{\infty} b_{j}x^{j}\in R[[x]]$, addition is just adding coefficients of the same index (like in $R[x]$) and multiplication is termwise multiplication (Like in $R[x]$) which is called the \textit{Cauchy Product} for series. These are defined as follows:
\begin{align*}
A(x)B(x)&=(\sum_{i=0}^{\infty} a_{i}x^{i})(\sum_{j=0}^{\infty} b_{j}x^{j})=\sum_{n=0}^{\infty}(\sum_{i+j=n}a_{i}b_{j})x^{n}\\
A(x)+B(x) &= (\sum_{i=0}^{\infty} a_{i}x^{i})+(\sum_{j=0}^{\infty} b_{j}x^{j})=\sum_{n=0}^{\infty} (a_{n}+b_{n})x^{n}
\end{align*}
It is given that $R[[x]]$ is a ring, so we just show multiplication is commutative. Recall that $R$ and $\ZZ$ are  commutative. So: 
$$A(x)B(x)=\sum_{n=0}^{\infty}(\sum_{i+j=n}a_{i}b_{j})x^{n}=\sum_{n=0}^{\infty}(\sum_{j+i=n}b_{j}a_{i})x^{n}=B(x)A(x)$$
\end{proof}

\begin{proof}
\textbf{(b): } $$(1-x)\sum_{n=0}^{\infty}x^{n}=\sum_{n=0}^{\infty}x^{n}-x\sum_{n=0}^{\infty}x^{n}=\sum_{n=0}^{\infty}x^{n}(1-x)=\sum_{n=0}^{\infty}x^{n}-\sum_{n=0}^{\infty}x^{n+1}=\sum_{n=0}^{\infty}x^{n}-\sum_{n=1}x^{n}=1$$
\end{proof}
%%%%%%%%%%%%%%%%%%%%%%%%%%%%%%%%%%%%%%%%%%%%%%%%% 9 %%%%%%%%%%%%%%%%%%%%%%%%%%%%%%%%%%%%%%%%%%%%%%%%%
\newpage
\begin{prob}
Decide which of the following are ring homomorphisms from $M_2(\ZZ)$ to $\ZZ$:
\begin{parts}
\part $\begin{pmatrix} a & b \\ c & d \end{pmatrix} \mapsto a$
\part $\begin{pmatrix} a & b \\ c & d \end{pmatrix} \mapsto a+d$
\part $\begin{pmatrix} a & b \\ c & d \end{pmatrix} \mapsto ad-bc$
\end{parts}
\end{prob}
\begin{proof}
Let us refer to the mappings in $(a),(b),(c)$ via $\varphi_{a},\varphi_{b},\varphi_{c}$, respectively. 

\textbf{(a):} $\varphi_{a}(\begin{pmatrix} 1 & 1 \\ 1 & 1 \end{pmatrix})=1=\varphi_{a}(\begin{pmatrix} 1 & 1 \\ 1 & 1 \end{pmatrix})\varphi_{a}(\begin{pmatrix} 1 & 1 \\ 1 & 1 \end{pmatrix})\neq 2 = \varphi_{a}(\begin{pmatrix} 2 & 2 \\ 2 & 2 \end{pmatrix})=\varphi_{a}(\begin{pmatrix} 1 & 1 \\ 1 & 1 \end{pmatrix}\begin{pmatrix} 1 & 1 \\ 1 & 1 \end{pmatrix})$. So $\varphi_{a}$ is not a ring homomorphism.

\textbf{(b):} $\varphi_{b}(\begin{pmatrix} 1 & 1 \\ 1 & 0 \end{pmatrix})=1=\varphi_{b}(\begin{pmatrix} 1 & 1 \\ 1 & 0 \end{pmatrix})\varphi_{b}(\begin{pmatrix} 1 & 1 \\ 1 & 0 \end{pmatrix})\neq 2 = \varphi_{b}(\begin{pmatrix} 2 & 1 \\ 1 & 0 \end{pmatrix})=\varphi_{b}(\begin{pmatrix} 1 & 1 \\ 1 & 0 \end{pmatrix}\begin{pmatrix} 1 & 1 \\ 1 & 0 \end{pmatrix})$. So $\varphi_{b}$ is not a ring homomorphism.

\textbf{(c):}$\varphi_{c}(\begin{pmatrix} 1 & 0 \\ 0 & 0 \end{pmatrix})+\varphi_{c}(\begin{pmatrix} 0 & 0 \\ 0 & 1 \end{pmatrix})=0+0=0\neq 1=\varphi_{c}(\begin{pmatrix} 1 & 0 \\ 0 & 1 \end{pmatrix})=\varphi_{c}(\begin{pmatrix} 1 & 0 \\ 0 & 0 \end{pmatrix}+\begin{pmatrix} 0 & 0 \\ 0 & 1 \end{pmatrix})$ So $\varphi_{c}$ is not a ring homomorphism.

So none of them are ring homomorphisms.

\end{proof}

%%%%%%%%%%%%%%%%%%%%%%%%%%%%%%%%%%%%%%%%%%%%%%%%% 10 %%%%%%%%%%%%%%%%%%%%%%%%%%%%%%%%%%%%%%%%%%%%%%%%
\newpage
\begin{prob}
Let
$$
R=\left\{
\begin{pmatrix}
a & b \\
0 & d
\end{pmatrix}
\;\middle|\;
a,b,d \in \ZZ
\right\}.
$$
Prove that the map
$$
\varphi : R \to \ZZ \times \ZZ,\qquad
\varphi\!\left(
\begin{pmatrix}
a & b \\
0 & d
\end{pmatrix}
\right)=(a,d)
$$
is a surjective ring homomorphism and describe its kernel.
\end{prob}

\begin{proof}
For all $\alpha=(a_{ij})=\begin{pmatrix} a_{11} & a_{12} \\ a_{21}=0 & a_{22} \end{pmatrix}, \beta=(b_{ij})=\begin{pmatrix} b_{11} & b_{12} \\ b_{21}=0 & b_{22} \end{pmatrix}\in R$,
\begin{align*}
&[1]: \varphi(I_{2})= (1,1)=1_{\ZZ\times \ZZ}\\
&[+]:\varphi(\alpha)+\varphi(\beta)=(a_{11},a_{22})+(b_{11},b_{22})=(a_{11}+b_{11},a_{22}+b_{22})=\varphi((a_{ij}+b_{ij}))=\varphi(\alpha+\beta)\\
&[\cdot]:\varphi(\alpha)\varphi(\beta)=(a_{11},a_{22})(b_{11},b_{22})=(a_{11}b_{11},a_{22}b_{22})=\varphi(\begin{pmatrix} a_{11}b_{11} & a_{11}b_{12}+a_{12}b_{22} \\ 0 & a_{22}b_{22} \end{pmatrix})=\varphi(\alpha\beta)\\
&[\text{Onto}]:\forall (m,n)\in \ZZ\times \ZZ,\;\varphi(\begin{pmatrix}m & 0 \\ 0 & n \end{pmatrix})=(m,n)
\end{align*}
$\varphi(\alpha)=(0,0)\implies a_{11}=a_{22}=0\implies \alpha\in \{\begin{pmatrix} 0 & b \\ 0 & 0 \end{pmatrix}\mid b\in \ZZ\}$ and so $\ker\varphi\subseteq \{\begin{pmatrix} 0 & b \\ 0 & 0 \end{pmatrix}\mid b\in \ZZ\}$ and obviously $\varphi(\begin{pmatrix} 0 & b \\ 0 & 0 \end{pmatrix})=(0,0)$ for all $b\in \ZZ$ so $\{\begin{pmatrix} 0 & b \\ 0 & 0 \end{pmatrix}\mid b\in \ZZ\}\subseteq \ker \varphi$

Thus,

$$\varphi \text{ is a surjective ring homomorphism with} \ker \varphi = \{\begin{pmatrix} 0 & b \\ 0 & 0 \end{pmatrix}\mid b\in \ZZ\}$$
\end{proof}

%%%%%%%%%%%%%%%%%%%%%%%%%%%%%%%%%%%%%%%%%%%%%%%%% 11 %%%%%%%%%%%%%%%%%%%%%%%%%%%%%%%%%%%%%%%%%%%%%%%%
\newpage
\begin{prob}
Decide which of the following are ideals of $\ZZ \times \ZZ$:
\begin{parts}
\part $\{(a,a)\mid a \in \ZZ\}$
\part $\{(2a,2b)\mid a,b \in \ZZ\}$
\part $\{(2a,0)\mid a \in \ZZ\}$
\part $\{(a,-a)\mid a \in \ZZ\}$
\end{parts}
\end{prob}

\begin{proof} Call the subsets of $\ZZ\times \ZZ$ specified in $(a),(b),(c),(d)$: $A,B,C,D$, respectively.

\textbf{(a):} $(1,2)\in \ZZ\times \ZZ$ and $(1,1)\in A$ but $(1,2)(1,1)=(1,1)(1,2)=(1,2)\notin A$, so this is not an ideal.

\textbf{(b):} For all $(m,n)\in \ZZ\times \ZZ$ and all $(2a,2b),(2\alpha,2\beta)\in B$, $((2a,2b)-(2\alpha,2\beta))=(2(a-\alpha),2(b-\beta))\in B$ and $(m,n)(2a,2b)=(m2a,n2b)=(2(ma),2(nb))=(2(am),2(bn))=(2a,2b)(m,n)\in B=2\ZZ\times 2\ZZ$. This is a two-sided ideal.

\textbf{(c):} For all $(m,n)\in \ZZ\times \ZZ$ and all $(2a,0),(2\alpha,0)\in C$, $((2a,0)-(2\alpha,0))=(2(a-\alpha),0)\in C$ and then $(m,n)(2a,0)=(m2a,0)=(2(ma),0)=(2(am),0)=(2a,0)(m,n)\in C=2\ZZ\times \{0\}$. This is a two-sided ideal.

\textbf{(d):} $(1,-1)\in D$ but $(1,-1)(1,-1)=(1,1)\notin D$ so it's not closed under multiplication and therefore not an ideal.




\end{proof}

%%%%%%%%%%%%%%%%%%%%%%%%%%%%%%%%%%%%%%%%%%%%%%%%% 12 %%%%%%%%%%%%%%%%%%%%%%%%%%%%%%%%%%%%%%%%%%%%%%%%

\begin{prob}
The characteristic of a ring $R$ (denoted $\mathrm{char}\,R$) is the smallest $n \in \NN$ such that
$$
\underbrace{1_R+\cdots+1_R}_{n\ \text{times}}=0,
$$
and if there is no such $n$ we say the characteristic of $R$ is $0$.

Prove that an integral domain has characteristic $0$ or a prime.
\end{prob}

\begin{proof}
Let $R$ be an integral domain and suppose $\Char{R}=p>0$ where $p$ is not prime. If a ring has characteristic $1$, then $1=0$ and the ring is $\{0\}$, which isn't an integral domain by definition. So, $1<p=ab$ for some non-trivial divisors $1<a,b<p$. But then $(\sum_{i=1}^{a} 1)(\sum_{j=1}^{b} 1)=\sum_{n=1}^{ab}1=\sum_{n=1}^{p}1=0\in R$ and $a=\sum_{i=1}^{a} 1, b=\sum_{j=1}^{b} 1$ are both non-zero zero-divisors in $R$ since $a,b<\Char{R}=p\implies (\sum_{i=1}^{a} 1),(\sum_{j=1}^{b} 1)\neq 0\in R$, a contradiction since $R$ is an integral domain. Therefore, if $\Char{R}$ is non-zero, it is prime. Otherwise $\Char{R}=0.$

Thus,

\begin{center}
An integral domain $R$ has characteristic $0$ or a prime $p>0$.
\end{center}

\end{proof}
%%%%%%%%%%%%%%%%%%%%%%%%%%%%%%%%%%%%%%%%%%%%%%%%%%%%%%%%%%%%%%%%%%%%%%%%%%%%%%%%%%%%%%%%%%%%%%%%%%%%%



\end{document}