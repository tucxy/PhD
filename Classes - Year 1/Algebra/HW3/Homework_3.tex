\documentclass[addpoints,10pt]{exam}

\usepackage{amsmath,amsthm,enumitem,wrapfig,amsfonts,mathtools}
\usepackage[mathscr]{euscript}
\usepackage[super]{nth}

\usepackage{geometry}
\usepackage[T1]{fontenc} % Use 8-bit encoding that has 256 glyphs
\renewcommand{\rmdefault}{ptm} %Change the Front Family from the default(cmr) to ptm(Times)
\usepackage{amsmath,amsfonts,amsthm,amssymb} % Math packages
\usepackage{bm}
\usepackage{mathptmx}
\usepackage{graphicx}
\usepackage{sectsty} % Allows customizing section commands
% \allsectionsfont{\centering} % Make all sections centered, the default font and small caps
\usepackage{pgfplots}
\pgfplotsset{compat=1.18}
\usetikzlibrary{arrows.meta}
\usepackage{xcolor}
\definecolor{darkpastelgreen}{rgb}{0.01, 0.75, 0.24}
\definecolor{blue-violet}{rgb}{0.54, 0.17, 0.89}

% Custom problem environment
\newcounter{cprob}
\newenvironment{cprob}[1]{%
    \setcounter{cprob}{#1}%
    \noindent\textbf{Problem \thecprob.}%
}{%
    \par\bigskip%
}

\theoremstyle{plain}
\newtheorem{thm}{\protect\theoremname}
  \theoremstyle{definition}
  \newtheorem{prob}[thm]{Problem}
  \newtheorem*{problem*}{Open Problem}
  \theoremstyle{plain}
  \newtheorem{conjecture}[thm]{Conjecture}
  \theoremstyle{plain}
  \newtheorem{lem}[thm]{Lemma}
  \newtheorem*{lem*}{Lemma}
  \newtheorem{obs}[thm]{Observation}
  \newtheorem{cor}[thm]{Corollary}
  \theoremstyle{definition}
\newtheorem{definition}[thm]{Definition}


% Patch prob environment to be single spaced
\let\oldprob\prob
\let\endoldprob\endprob
\renewenvironment{prob}
  {\begin{singlespace}\oldprob}
  {\endoldprob\end{singlespace}}

% start problem one line below like for enumerated problems with multiple parts
\newcommand{\belowtitle}{\leavevmode\newline}
%\Observe command
\newcommand{\Observe}{\text{Observe.}}
%(=>)
\newcommand{\IF}{\mathbf{(\Rightarrow)}}
%(<=)
\newcommand{\FI}{\mathbf{(\Leftarrow)}}
%equivalence classes; \class[S]{ *content in square brackets* }
\newcommand{\class}[2][]{\ensuremath{\left[\,#2\,\right]_{#1}}}

\newcommand{\horrule}[1]{\rule{\linewidth}{#1}}
\newcommand{\kk}{\ensuremath{\Bbbk}} 
\newcommand{\CC}{\ensuremath{\mathbb{C}}}
\newcommand{\FF}{\ensuremath{\mathbb{F}}}
\newcommand{\KK}{\ensuremath{\mathbb{K}}}
\newcommand{\NN}{\ensuremath{\mathbb{N}}}
\newcommand{\QQ}{\ensuremath{\mathbb{Q}}} 
\newcommand{\RR}{\ensuremath{\mathbb{R}}} 
\newcommand{\ZZ}{\ensuremath{\mathbb{Z}}}
\newcommand{\MM}{\ensuremath{\mathcal{M}}}
\newcommand{\TT}{\ensuremath{\mathcal{T}}}
\newcommand{\BB}{\ensuremath{\mathcal{B}}}
\newcommand{\VV}{\ensuremath{\mathcal{V}}}
\newcommand{\WW}{\ensuremath{\mathcal{W}}}
\newcommand{\UU}{\ensuremath{\mathcal{U}}}
\newcommand{\PP}{\ensuremath{\mathcal{P}}}
\newcommand{\LL}{\ensuremath{\mathcal{L}}}

\newcommand{\sm}{\char`\\}
%vector stuff
\DeclarePairedDelimiter{\ip}{\langle}{\rangle} %inner product/generate
\DeclarePairedDelimiter{\norm}{\lVert}{\rVert} %norm
\DeclarePairedDelimiter{\sqb}{\lbrack}{\rbrack} %corrd
\newcommand{\floor}[1]{\left\lfloor #1 \right\rfloor}
\newcommand{\ceil}[1]{\left\lceil #1 \right\rceil}
\newcommand{\mbf}[1]{\ensuremath{\mathbf{#1}}}
\newcommand{\tbf}[1]{\textbf{ #1 }}


\makeatletter
\renewcommand*\env@matrix[1][*\c@MaxMatrixCols c]{%
  \hskip -\arraycolsep
  \let\@ifnextchar\new@ifnextchar
  \array{#1}}
\makeatother

\def\env@matrix{\hskip -\arraycolsep
  \let\@ifnextchar\new@ifnextchar
  \array{*\c@MaxMatrixCols c}}

  \newcommand{\proj}[2]{\text{proj}_{#1}(#2)}
  

%%% Formatting: Page Header
\newcommand{\StudentName}{Danny Banegas}
\newcommand{\AssignmentName}{Homework 2}
\newcommand{\CourseName}{MATH 721 - Algebra II}


\pagestyle{headandfoot}
\runningheadrule
\firstpageheadrule
\firstpageheader{\CourseName}{\StudentName}{\AssignmentName}
\runningheader{\CourseName}{\StudentName}{\AssignmentName}
\firstpagefooter{}{\thepage}{}
\runningfooter{}{\thepage}{}

\printanswers

\DeclareMathAlphabet{\mathcal}{OMS}{cmsy}{m}{n}

\usepackage{parskip}
\usepackage{setspace}
\doublespacing
% % % % % % % % % % % % % % % % % % % % % % % % % % % % % % % % % % % % % % % % % % % % % % % % % % % % % % % % % % % % % % 
\begin{document}
Submit the following from the problem list: 40, 41, 42, 45, 46, 47, 48, 49, 50, 51, 52.
%%%%%%%%%%%%%%%%%%%%%%%%%%%%%%%%%%% 40 %%%%%%%%%%%%%%%%%%%%%%%%%%%%%%%%%%%%%%
\setcounter{thm}{39}   % next prob is 40
\begin{prob}
Prove that an abelian group has a composition series if and only if it is finite.
\end{prob}
\newpage
%%%%%%%%%%%%%%%%%%%%%%%%%%%%%%%%%%% 41 %%%%%%%%%%%%%%%%%%%%%%%%%%%%%%%%%%%%%%
\begin{prob}
Prove that a solvable simple group is abelian.
\end{prob}
\newpage
%%%%%%%%%%%%%%%%%%%%%%%%%%%%%%%%%%% 42 %%%%%%%%%%%%%%%%%%%%%%%%%%%%%%%%%%%%%%
\begin{prob}
Prove that a solvable group that has a composition series is finite.
\end{prob}
\newpage
%%%%%%%%%%%%%%%%%%%%%%%%%%%%%%%%%%% 45 %%%%%%%%%%%%%%%%%%%%%%%%%%%%%%%%%%%%%%
\setcounter{thm}{44}   % next prob is 45
\begin{prob}
If $\KK\subseteq \FF$ is a field extension, $u,v\in \FF$, $v$ is algebraic over $\KK(u)$, and $v$ is transcendental over $\KK$, then $u$ is algebraic over $\KK(v)$.
\end{prob}
\newpage
%%%%%%%%%%%%%%%%%%%%%%%%%%%%%%%%%%% 46 %%%%%%%%%%%%%%%%%%%%%%%%%%%%%%%%%%%%%%
\begin{prob}
If $\KK\subseteq \FF$ is a field extension and $u\in \FF$ is algebraic of odd degree over $\KK$, then so is $u^{2}$ and $\KK(u)=\KK(u^{2})$. 
\end{prob}
\newpage
%%%%%%%%%%%%%%%%%%%%%%%%%%%%%%%%%%% 47 %%%%%%%%%%%%%%%%%%%%%%%%%%%%%%%%%%%%%%
\begin{prob}
Let $\KK \subseteq \FF$ be a field extension. If $X^n - a \in \KK[X]$ is irreducible and $u \in \FF$ is a root of $X^n - a$ and $m$ divides $n$, then the degree of $u^m$ over $\KK$ is $n/m$. What is the irreducible polynomial of $u^m$ over $\KK$?.
\end{prob}
\newpage
%%%%%%%%%%%%%%%%%%%%%%%%%%%%%%%%%%% 48 %%%%%%%%%%%%%%%%%%%%%%%%%%%%%%%%%%%%%%
\begin{prob}
Let $\KK \subseteq R \subseteq \FF$ be an extension of rings with $\KK,\FF$ fields. If $\KK \subseteq \FF$ is algebraic, prove that $R$ is a field.
\end{prob}
\newpage
%%%%%%%%%%%%%%%%%%%%%%%%%%%%%%%%%%% 49 %%%%%%%%%%%%%%%%%%%%%%%%%%%%%%%%%%%%%%
\begin{prob}
Let $f = X^3 - 6X^2 + 9X + 3 \in \mathbb{Q}[X]$.
\begin{enumerate}[label=(\alph*)]
\item Prove that $f$ is irreducible in $\mathbb{Q}[X]$.
\item Let $u$ be a real root of $f$. Consider the extension $\mathbb{Q} \subseteq \mathbb{Q}(u)$. Express each of the following elements in terms of the basis $\{1,u,u^2\}$ of the $\mathbb{Q}$-vector space $\mathbb{Q}(u)$:
\[
u^4,\quad u^5,\quad 3u^5 - u^4 + 2,\quad (u+1)^{-1},\quad (u^2 - 6u + 8)^{-1}.
\]
\end{enumerate}
\end{prob}
\newpage
%%%%%%%%%%%%%%%%%%%%%%%%%%%%%%%%%%% 50 %%%%%%%%%%%%%%%%%%%%%%%%%%%%%%%%%%%%%%
\begin{prob}
Let $F = \mathbb{Q}(\sqrt{2}, \sqrt{3})$. Find $[F:\mathbb{Q}]$ and a basis of $\FF$ over $\mathbb{Q}$.
\end{prob}
\newpage
%%%%%%%%%%%%%%%%%%%%%%%%%%%%%%%%%%% 51 %%%%%%%%%%%%%%%%%%%%%%%%%%%%%%%%%%%%%%
\begin{prob}
Let $\KK$ be a field. In the field $\KK(X)$, let $u=X^{3}/(X+1)$. What is $[\KK(X):\KK(u)]?$
\end{prob}
\newpage
%%%%%%%%%%%%%%%%%%%%%%%%%%%%%%%%%%% 52 %%%%%%%%%%%%%%%%%%%%%%%%%%%%%%%%%%%%%%
\begin{prob}
Let $\KK\subseteq \FF$ be a field extension. If $u,v\in \FF$ are algebraic over $\KK$ of degrees $m$ and $n$, respectively, then $[\KK(u,v): \KK]\leq mn.$ If $m$ and $n$ are relatively prime, then $[\KK(u,v):\KK]=mn.$
\end{prob}
\newpage


\end{document}