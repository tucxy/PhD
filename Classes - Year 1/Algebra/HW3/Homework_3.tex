\documentclass[addpoints,10pt]{exam}

\usepackage{amsmath,amsthm,enumitem,wrapfig,amsfonts,mathtools}
\usepackage[mathscr]{euscript}
\usepackage[super]{nth}

\usepackage{geometry}
\usepackage[T1]{fontenc} % Use 8-bit encoding that has 256 glyphs
\renewcommand{\rmdefault}{ptm} %Change the Front Family from the default(cmr) to ptm(Times)
\usepackage{amsmath,amsfonts,amsthm,amssymb} % Math packages
\usepackage{bm}
\usepackage{mathptmx}
\usepackage{graphicx}
\usepackage{sectsty} % Allows customizing section commands
% \allsectionsfont{\centering} % Make all sections centered, the default font and small caps
\usepackage{pgfplots}
\pgfplotsset{compat=1.18}
\usetikzlibrary{arrows.meta}
\usepackage{xcolor}
\definecolor{darkpastelgreen}{rgb}{0.01, 0.75, 0.24}
\definecolor{blue-violet}{rgb}{0.54, 0.17, 0.89}

% Custom problem environment
\newcounter{cprob}
\newenvironment{cprob}[1]{%
    \setcounter{cprob}{#1}%
    \noindent\textbf{Problem \thecprob.}%
}{%
    \par\bigskip%
}

\theoremstyle{plain}
\newtheorem{thm}{\protect\theoremname}
  \theoremstyle{definition}
  \newtheorem{prob}[thm]{Problem}
  \newtheorem*{problem*}{Open Problem}
  \theoremstyle{plain}
  \newtheorem{conjecture}[thm]{Conjecture}
  \theoremstyle{plain}
  \newtheorem{lem}[thm]{Lemma}
  \newtheorem*{lem*}{Lemma}
  \newtheorem{obs}[thm]{Observation}
  \newtheorem{cor}[thm]{Corollary}
  \theoremstyle{definition}
\newtheorem{definition}[thm]{Definition}


% Patch prob environment to be single spaced
\let\oldprob\prob
\let\endoldprob\endprob
\renewenvironment{prob}
  {\begin{singlespace}\oldprob}
  {\endoldprob\end{singlespace}}

% start problem one line below like for enumerated problems with multiple parts
\newcommand{\belowtitle}{\leavevmode\newline}
%\Observe command
\newcommand{\Observe}{\text{Observe.}}
%(=>)
\newcommand{\IF}{\mathbf{(\Rightarrow)}}
%(<=)
\newcommand{\FI}{\mathbf{(\Leftarrow)}}
%equivalence classes; \class[S]{ *content in square brackets* }
\newcommand{\class}[2][]{\ensuremath{\left[\,#2\,\right]_{#1}}}

\newcommand{\horrule}[1]{\rule{\linewidth}{#1}}
\newcommand{\kk}{\ensuremath{\Bbbk}} 
\newcommand{\CC}{\ensuremath{\mathbb{C}}}
\newcommand{\FF}{\ensuremath{\mathbb{F}}}
\newcommand{\KK}{\ensuremath{\mathbb{K}}}
\newcommand{\NN}{\ensuremath{\mathbb{N}}}
\newcommand{\QQ}{\ensuremath{\mathbb{Q}}} 
\newcommand{\RR}{\ensuremath{\mathbb{R}}} 
\newcommand{\ZZ}{\ensuremath{\mathbb{Z}}}
\newcommand{\MM}{\ensuremath{\mathcal{M}}}
\newcommand{\TT}{\ensuremath{\mathcal{T}}}
\newcommand{\BB}{\ensuremath{\mathcal{B}}}
\newcommand{\VV}{\ensuremath{\mathcal{V}}}
\newcommand{\WW}{\ensuremath{\mathcal{W}}}
\newcommand{\UU}{\ensuremath{\mathcal{U}}}
\newcommand{\PP}{\ensuremath{\mathcal{P}}}
\newcommand{\LL}{\ensuremath{\mathcal{L}}}

\newcommand{\sm}{\char`\\}
%vector stuff
\DeclarePairedDelimiter{\ip}{\langle}{\rangle} %inner product/generate
\DeclarePairedDelimiter{\norm}{\lVert}{\rVert} %norm
\DeclarePairedDelimiter{\sqb}{\lbrack}{\rbrack} %corrd
\newcommand{\floor}[1]{\left\lfloor #1 \right\rfloor}
\newcommand{\ceil}[1]{\left\lceil #1 \right\rceil}
\newcommand{\mbf}[1]{\ensuremath{\mathbf{#1}}}
\newcommand{\tbf}[1]{\textbf{ #1 }}
\newcommand{\Span}{\ensuremath{\mathrm{Span}}}

\makeatletter
\renewcommand*\env@matrix[1][*\c@MaxMatrixCols c]{%
  \hskip -\arraycolsep
  \let\@ifnextchar\new@ifnextchar
  \array{#1}}
\makeatother

\def\env@matrix{\hskip -\arraycolsep
  \let\@ifnextchar\new@ifnextchar
  \array{*\c@MaxMatrixCols c}}

  \newcommand{\proj}[2]{\text{proj}_{#1}(#2)}
  

%%% Formatting: Page Header
\newcommand{\StudentName}{Danny Banegas}
\newcommand{\AssignmentName}{Homework 3}
\newcommand{\CourseName}{MATH 721 - Algebra II}


\pagestyle{headandfoot}
\runningheadrule
\firstpageheadrule
\firstpageheader{\CourseName}{\StudentName}{\AssignmentName}
\runningheader{\CourseName}{\StudentName}{\AssignmentName}
\firstpagefooter{}{\thepage}{}
\runningfooter{}{\thepage}{}

\printanswers

\DeclareMathAlphabet{\mathcal}{OMS}{cmsy}{m}{n}

\usepackage{parskip}
\usepackage{setspace}
\doublespacing
% % % % % % % % % % % % % % % % % % % % % % % % % % % % % % % % % % % % % % % % % % % % % % % % % % % % % % % % % % % % % % 
\begin{document}
%%%%%%%%%%%%%%%%%%%%%%%%%%%%%%%%%%% 40 %%%%%%%%%%%%%%%%%%%%%%%%%%%%%%%%%%%%%%
\setcounter{thm}{39}   % next prob is 40
\begin{prob}
Prove that an abelian group has a composition series if and only if it is finite.
\end{prob}

\begin{proof}
hey
\end{proof}
\newpage
%%%%%%%%%%%%%%%%%%%%%%%%%%%%%%%%%%% 41 %%%%%%%%%%%%%%%%%%%%%%%%%%%%%%%%%%%%%%
\begin{prob}
Prove that a solvable simple group is abelian.
\end{prob}
\newpage
%%%%%%%%%%%%%%%%%%%%%%%%%%%%%%%%%%% 42 %%%%%%%%%%%%%%%%%%%%%%%%%%%%%%%%%%%%%%
\begin{prob}
Prove that a solvable group that has a composition series is finite.
\end{prob}
\newpage
%%%%%%%%%%%%%%%%%%%%%%%%%%%%%%%%%%% 45 %%%%%%%%%%%%%%%%%%%%%%%%%%%%%%%%%%%%%%
\setcounter{thm}{44}   % next prob is 45
\begin{prob}
If $\KK\subseteq \FF$ is a field extension, $u,v\in \FF$, $v$ is algebraic over $\KK(u)$, and $v$ is transcendental over $\KK$, then $u$ is algebraic over $\KK(v)$.
\end{prob}
\newpage
%%%%%%%%%%%%%%%%%%%%%%%%%%%%%%%%%%% 46 %%%%%%%%%%%%%%%%%%%%%%%%%%%%%%%%%%%%%%
\begin{prob}
If $\KK\subseteq \FF$ is a field extension and $u\in \FF$ is algebraic of odd degree over $\KK$, then so is $u^{2}$ and $\KK(u)=\KK(u^{2})$. 
\end{prob}
\newpage
%%%%%%%%%%%%%%%%%%%%%%%%%%%%%%%%%%% 47 %%%%%%%%%%%%%%%%%%%%%%%%%%%%%%%%%%%%%%
\begin{prob}
Let $\KK \subseteq \FF$ be a field extension. If $X^n - a \in \KK[X]$ is irreducible and $u \in \FF$ is a root of $X^n - a$ and $m$ divides $n$, then the degree of $u^m$ over $\KK$ is $n/m$. What is the irreducible polynomial of $u^m$ over $\KK$?.
\end{prob}
\newpage
%%%%%%%%%%%%%%%%%%%%%%%%%%%%%%%%%%% 48 %%%%%%%%%%%%%%%%%%%%%%%%%%%%%%%%%%%%%%
\begin{prob}
Let $\KK \subseteq R \subseteq \FF$ be an extension of rings with $\KK,\FF$ fields. If $\KK \subseteq \FF$ is algebraic, prove that $R$ is a field.
\end{prob}
\newpage
%%%%%%%%%%%%%%%%%%%%%%%%%%%%%%%%%%% 49 %%%%%%%%%%%%%%%%%%%%%%%%%%%%%%%%%%%%%%
\begin{prob}
Let $f = X^3 - 6X^2 + 9X + 3 \in \mathbb{Q}[X]$.
\begin{enumerate}[label=(\alph*)]
\item Prove that $f$ is irreducible in $\mathbb{Q}[X]$.
\item Let $u$ be a real root of $f$. Consider the extension $\mathbb{Q} \subseteq \mathbb{Q}(u)$. Express each of the following elements in terms of the basis $\{1,u,u^2\}$ of the $\mathbb{Q}$-vector space $\mathbb{Q}(u)$:
\[
u^4,\quad u^5,\quad 3u^5 - u^4 + 2,\quad (u+1)^{-1},\quad (u^2 - 6u + 8)^{-1}.
\]
\end{enumerate}
\end{prob}
\newpage
%%%%%%%%%%%%%%%%%%%%%%%%%%%%%%%%%%% 50 %%%%%%%%%%%%%%%%%%%%%%%%%%%%%%%%%%%%%%
\begin{prob}
Let $F = \mathbb{Q}(\sqrt{2}, \sqrt{3})$. Find $[F:\mathbb{Q}]$ and a basis of $\FF$ over $\mathbb{Q}$.
\end{prob}

\begin{proof}
To begin, $\sqrt{2}$ and $\sqrt{3}$ are zeros of monic irreducible polynomials $x^{2}-2$ and $x^{2}-3$, respectively, over $\QQ$. So $\QQ(\sqrt{2})\cong \QQ[x]/\langle x^{2}-2\rangle\cong (\mathrm{Span}_{\QQ}\{1,x\}\subseteq \QQ[x]) \cong \QQ[x]/\langle x^{2}-3\rangle \cong \QQ(\sqrt{3})$. So then $\QQ(\sqrt{2})=\mathrm{Span}\{1,\sqrt{2}\}$ and $\QQ(\sqrt{3})=\mathrm{Span}\{1,\sqrt{3}\}$. Observe. 
\begin{align*}
  &\sqrt{3}=a+b\sqrt{2}\text{ for some }a,b\in \QQ \implies 3=(a+b\sqrt{2})^{2}=(a^{2}+(2ab)\sqrt{2}+2b^{2})\not\in \QQ,\\
  &\sqrt{2}=a+b\sqrt{3}\text{ for some }a,b\in \QQ\implies 2=(a+b\sqrt{3})^{2}=(a^{2}+(2ab)\sqrt{3}+3b^{2})\not\in \QQ,\\
  &\sqrt{6}=a+b\sqrt{2}\text{ for some }a,b\in \QQ\implies 6=(a+b\sqrt{2})^{2}=(a^{2}+(2ab)\sqrt{2}+2b^{2})\not\in \QQ,\\
  &\sqrt{6}=a+b\sqrt{3}\text{ for some }a,b\in \QQ\implies 6=(a+b\sqrt{3})^{2}=(a^{2}+(2ab)\sqrt{3}+3b^{2})\not\in \QQ.
\end{align*}
  All of the above are contradictions. So $1,\sqrt{2},\sqrt{3},\sqrt{6}$ must be linearly independent over $\QQ$. Next, $\QQ(\sqrt{2},\sqrt{3})=\mathrm{Span}_{\QQ(\sqrt{2})}\{1,\sqrt{3}\}=\{\alpha+\beta\sqrt{3}\mid \alpha,\beta\in \QQ(\sqrt{2})\}=\{(a+b\sqrt{2})+(c+d\sqrt{2})\sqrt{3}\mid a,b,c,d\in \QQ\}=\{a+b\sqrt{2}+c\sqrt{3}+d\sqrt{6}\mid a,b,c,d\in \QQ\}$. So $\{1,\sqrt{2},\sqrt{3},\sqrt{6}\}$ spans $\QQ(\sqrt{2},\sqrt{3})$ and since it's elements are linearly independent over $\QQ$, it must be a basis for $\QQ(\sqrt{2},\sqrt{3})$ over $\QQ$.

  Thus,
  \begin{center}
    $\{1,\sqrt{2},\sqrt{3},\sqrt{6}\}$ is a basis for $\QQ(\sqrt{2},\sqrt{3})$ over $\QQ$ and $[\QQ(\sqrt{2},\sqrt{3}):\QQ]=4.$
  \end{center}
\end{proof}

\newpage
%%%%%%%%%%%%%%%%%%%%%%%%%%%%%%%%%%% 51 %%%%%%%%%%%%%%%%%%%%%%%%%%%%%%%%%%%%%%
\begin{prob}
Let $\KK$ be a field. In the field $\KK(X)$, let $u=X^{3}/(X+1)$. What is $[\KK(X):\KK(u)]?$
\end{prob}

\begin{proof}
$(\KK(u))(x)=\left\{\frac{f(x)}{g(x)}\mid f,g\in \KK(u)[t]\right\}$ and then $u=\frac{x^{3}}{x+1}\implies u(x+1)-x^{3}=ux+u-x^{3}=0\implies x^{3}-ux-u=0$. So $x$ is a zero of the polynomial $t^{3}-ut-u$ over $\KK(u)$. This means that the degree of $x$ over $K(u)$, or equivalently, $[\KK(x):\KK(u)]$ must divide $3$. Therefore, $[\KK(x):\KK(u)]\in \{1,3\}$. Suppose $[\KK(x):\KK(u)]=1$, then $\KK(x)=\KK(u)$ and $x=\frac{f(u)}{g(u)}$ for some $f(u),g(u)\neq 0$ coprime over $\KK(u)$. Observe. 
\begin{center}
  $x^{3}-ux-u=(\frac{f(u)}{g(u)})^{3}-u(\frac{f(u)}{g(u)})-u=0$ and $f(u)^{3}-uf(u)g(u)^{2}-ug(u)^{3}=0$. So then $f(u)^{3}=uf(u)g(u)^{2}+ug(u)^{3}=ug(u)^{2}(f(u)+g(u))\newline\implies 3\mathrm{deg}(f(u))=1+2\mathrm{deg}(g(u))+\mathrm{max}\{\mathrm{deg}(f(u)),\mathrm{deg}(f(u))\}.$
\end{center}
Let $a=\mathrm{deg}(f(u)),b=\mathrm{deg}(g(u))$ and note that both belong to $\ZZ^{+}$. We get the following cases:
\begin{center}
  $\begin{cases} 3a=1+2b+a\\ \text{or} \\ 3a=1+2b+b\end{cases}\implies \begin{cases} 2a=1+2b\\ \text{or} \\ 3a=1+3b\end{cases}\implies \begin{cases} 2(a+b)=1\\ \text{or} \\ 3(a+b)=1\end{cases}\implies \begin{cases} (a+b)=\frac{1}{2}\\ \text{or} \\ (a+b)=\frac{1}{3}\end{cases}.$
\end{center}
Both of the above are contradictions. So $[\KK(x):\KK(u)]=3$.

\end{proof}
\newpage
%%%%%%%%%%%%%%%%%%%%%%%%%%%%%%%%%%% 52 %%%%%%%%%%%%%%%%%%%%%%%%%%%%%%%%%%%%%%
\begin{prob}
Let $\KK\subseteq \FF$ be a field extension. If $u,v\in \FF$ are algebraic over $\KK$ of degrees $m$ and $n$, respectively, then $[\KK(u,v): \KK]\leq mn.$ If $m$ and $n$ are relatively prime, then $[\KK(u,v):\KK]=mn.$
\end{prob}

\begin{proof}
$\KK(u)$ and $\KK(v)$ have bases $\BB_{u}=\{1,\hdots,u^{m-1}\}$ and $\BB_{v}=\{1,\hdots, v^{n-1}\}$, respectively, over $\KK$. Also, $\KK(u,v)=\Span_{\KK_{u}}\BB_{v}=\{\sum_{i=0}^{n-1}a_{i}u^{i}\mid a_{0},\hdots,a_{n-1}\in \KK(u)\}=\Span_{\KK}\,\BB_{u}\BB_{v}$. So $\BB_{u}\BB_{v}$ span $\KK(u,v)$ over $\KK$. Therefore, $[\KK(u,v):\KK]=|\BB_{m}\BB_{n}|\leq |\BB_{u}||\BB_{v}|=mn.$

Suppose $\mathrm{gcd}(m,n)=1$. Since $\KK(u,v)\supseteq \KK(u)\supseteq \KK$, by the Tower Law we have:
  $$[\KK(u,v):\KK]=[\KK(u,v):\KK(u)][\KK(u):\KK]=[\KK(u,v):\KK(v)][\KK(v):\KK].$$
Therefore, $[\KK(u):\KK]=m$ and $[\KK(v):\KK]=n$ both divide $[\KK(u,v):\KK]$, which means it is a multiple of both $m$ and $n$. Well, since $\mathrm{lcm}(m,n)=\frac{mn}{gcd(m,n)}=mn$ and $[\KK(u,v):\KK]\leq mn$, it must be the case that in fact $[\KK(u,v):\KK]=mn$.

\end{proof}
\newpage


\end{document}