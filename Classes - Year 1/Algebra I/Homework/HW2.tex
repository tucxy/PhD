\documentclass[addpoints,10pt]{exam}

\usepackage{amsmath,amsthm,enumitem,wrapfig,amsfonts,mathtools}
\usepackage[mathscr]{euscript}
\usepackage[super]{nth}
\usepackage{dsfont}
\usepackage{xparse}

\usepackage{geometry}
\usepackage[T1]{fontenc} % Use 8-bit encoding that has 256 glyphs
\renewcommand{\rmdefault}{ptm} %Change the Front Family from the default(cmr) to ptm(Times)
\usepackage{amsmath,amsfonts,amsthm,amssymb} % Math packages
\usepackage{bm}
\usepackage{mathptmx}
\usepackage{graphicx}
\usepackage{sectsty} % Allows customizing section commands
% \allsectionsfont{\centering} % Make all sections centered, the default font and small caps
\usepackage{pgfplots}
\pgfplotsset{compat=1.18}
\usetikzlibrary{arrows.meta}
\usepackage{xcolor}
\definecolor{darkpastelgreen}{rgb}{0.01, 0.75, 0.24}
\definecolor{blue-violet}{rgb}{0.54, 0.17, 0.89}
%\usepackage{polylongdiv} %polynomial long division
% Custom problem environment
\usepackage{polynom}
\newcounter{cprob}
\newenvironment{cprob}[1]{%
    \setcounter{cprob}{#1}%
    \noindent\textbf{Problem \thecprob.}%
}{%
    \par\bigskip%
}

\theoremstyle{plain}
\newcommand{\theoremname}{Theorem}
\newtheorem{thm}{\protect\theoremname}
  \theoremstyle{definition}
  \newtheorem{prob}[thm]{Problem}
  \newtheorem*{problem*}{Open Problem}
  \theoremstyle{plain}
  \newtheorem{conjecture}[thm]{Conjecture}
  \theoremstyle{plain}
  \newtheorem{lem}[thm]{Lemma}
  \newtheorem*{lem*}{Lemma}
  \newtheorem{obs}[thm]{Observation}
  \newtheorem{cor}[thm]{Corollary}
  \theoremstyle{definition}
\newtheorem{definition}[thm]{Definition}


% Patch prob environment to be single spaced
\let\oldprob\prob
\let\endoldprob\endprob
\renewenvironment{prob}
  {\begin{singlespace}\oldprob}
  {\endoldprob\end{singlespace}}

% start problem one line below like for enumerated problems with multiple parts
\newcommand{\belowtitle}{\leavevmode\newline}
%\Observe command
\newcommand{\Observe}{\text{Observe.}}
%(=>)
\newcommand{\IF}{\mathbf{(\Rightarrow)}}
%(<=)
\newcommand{\FI}{\mathbf{(\Leftarrow)}}
%equivalence classes; \class[S]{ *content in square brackets* }
\newcommand{\class}[2][]{\ensuremath{\left[\,#2\,\right]_{#1}}}

\newcommand{\horrule}[1]{\rule{\linewidth}{#1}}
\newcommand{\kkk}{\ensuremath{\Bbbk}} 
\newcommand{\CC}{\ensuremath{\mathbb{C}}}
\newcommand{\FF}{\ensuremath{\mathbb{F}}}
\newcommand{\KK}{\ensuremath{\mathbb{K}}}
\newcommand{\NN}{\ensuremath{\mathbb{N}}}
\newcommand{\QQ}{\ensuremath{\mathbb{Q}}} 
\newcommand{\RR}{\ensuremath{\mathbb{R}}} 
\newcommand{\ZZ}{\ensuremath{\mathbb{Z}}}
\newcommand{\MM}{\ensuremath{\mathcal{M}}}
\newcommand{\TT}{\ensuremath{\mathcal{T}}}
\newcommand{\BB}{\ensuremath{\mathcal{B}}}
\newcommand{\VV}{\ensuremath{\mathcal{V}}}
\newcommand{\WW}{\ensuremath{\mathcal{W}}}
\newcommand{\UU}{\ensuremath{\mathcal{U}}}
\newcommand{\PP}{\ensuremath{\mathcal{P}}}
\newcommand{\LL}{\ensuremath{\mathcal{L}}}
\newcommand{\kk}{\ensuremath{\mathds{k}}}
\newcommand{\EE}{\ensuremath{\mathbb{E}}}


\newcommand{\sm}{\char`\\}
%vector stuff
\DeclarePairedDelimiter{\ip}{\langle}{\rangle} %inner product/generate
\DeclarePairedDelimiter{\norm}{\lVert}{\rVert} %norm
\DeclarePairedDelimiter{\sqb}{\lbrack}{\rbrack} %corrd

\newcommand{\floor}[1]{\left\lfloor #1 \right\rfloor}
\newcommand{\ceil}[1]{\left\lceil #1 \right\rceil}
\newcommand{\mbf}[1]{\ensuremath{\mathbf{#1}}}
\newcommand{\tbf}[1]{\textbf{ #1 }}
\newcommand{\Span}{\ensuremath{\mathrm{Span}}}
\newcommand{\Char}[1]{\mathrm{Char}\; #1}
\DeclareMathOperator{\lcm}{lcm}
\newcommand{\id}{\ensuremath{\mathrm{id}}}
\newcommand{\Gal}[2]{\mathrm{Gal}(#1/#2)}
\newcommand{\Aut}{\ensuremath{\mathrm{Aut}}}
\newcommand{\Fix}[2]{\mathrm{Fix}_{#1}(#2)}
\newcommand{\nil}{\ensuremath{\mathrm{nil}}}
\newcommand{\Hom}{\ensuremath{\mathrm{Hom}}}
\newcommand{\Obj}{\ensuremath{\mathrm{Obj}}}

\newcommand{\Rng}{\ensuremath{\mathrm{Rng}}}
\newcommand{\Ring}{\ensuremath{\mathrm{Ring}}}




\makeatletter
\renewcommand*\env@matrix[1][*\c@MaxMatrixCols c]{%
  \hskip -\arraycolsep
  \let\@ifnextchar\new@ifnextchar
  \array{#1}}
\makeatother

\def\env@matrix{\hskip -\arraycolsep
  \let\@ifnextchar\new@ifnextchar
  \array{*\c@MaxMatrixCols c}}

  \newcommand{\proj}[2]{\text{proj}_{#1}(#2)}
  

%%% Formatting: Page Header
\newcommand{\StudentName}{Danny Banegas}
\newcommand{\AssignmentName}{Homework 2}
\newcommand{\CourseName}{MATH 720 - Algebra I}


\pagestyle{headandfoot}
\runningheadrule
\firstpageheadrule
\firstpageheader{\CourseName}{\StudentName}{\AssignmentName}
\runningheader{\CourseName}{\StudentName}{\AssignmentName}
\firstpagefooter{}{\thepage}{}
\runningfooter{}{\thepage}{}

\printanswers

\DeclareMathAlphabet{\mathcal}{OMS}{cmsy}{m}{n}

\usepackage{parskip}
\usepackage{setspace}

% % % % % % % % % % % % % % % % % % % % % % % % % % % % % % % % % % % % % % % % % % % % % % % % % % % % % % % % % % % % % % 
\begin{document}
%%%%%%%%%%%%%%%%%%%%%%%%%%%%%%%%%%%%%%%%%%%%%%%%% 0 %%%%%%%%%%%%%%%%%%%%%%%%%%%%%%%%%%%%%%%%
\setcounter{thm}{0} % next prob is 1
Here are some Lemmas which are referenced throughout the problems.
\begin{lem}[Sum of Ideals is an Ideal]
  If $I,J$ are ideals of a ring $R$, then $I+J=\{i+j\mid i\in I,j\in J\}$ is also an ideal of $R$.
\end{lem}
\begin{proof}
  $\forall (i_{1}+j_{1}),(i_{2}+j_{2})\in I+J,\;(i_{1}+j_{1})-(i_{2}+j_{2})=(i_{1}-i_{2})+(j_{1}-j_{2})\in I+J\implies I+J\leq_{+}R.$ Next for all $r\in R$ and all $(i+j)\in I+J,\; r(i+j)=ri+rj,(i+j)r=ir+jr\in I+J$ since $ir,ri\in I$ and $rj,jr\in J$. Thus, 
  \begin{center}
    If $I,J$ are ideals of $R$, then $I+J$ is an ideal of $R$. 
  \end{center}
  \end{proof}

\begin{lem}[Inclusion Mapping is Ring Embedding]
  If $S\subseteq R$ is a subring of a ring $R$, then the inclusion mapping $\iota: S\hookrightarrow R$ defined via $\iota(s)=s$ is a ring embedding (an injective ring homomorphism).
\end{lem}
\begin{proof}
  For all $a,b\in S$,
  \begin{align*}
      &\iota(a+b)=a+b=\iota(a)+\iota(b)\\
      &\iota(ab)=ab=\iota(a)\iota(b)\\
      &\iota(1_{S})=1_{S}=1_{R}\\
      &\iota(a)=\iota(b)\implies a=b\text{ by definition.}
  \end{align*}
\end{proof}
\begin{lem}[Evaluation Mapping is a Surjective Ring Homomorphism]
The evaluation homomorphism $\epsilon_{a}:R[x]\to R$ defined by $\epsilon_{a}(f(x))=f(a)$ is a surjective ring homomorphism for any $a\in R$.
\end{lem}
\begin{proof}
  For all $f(x),g(x)\in R[x]$,
  \begin{align*}
      &\epsilon_{a}(f(x)+g(x))=f(a)+g(a)=\epsilon_{a}(f(x))+\epsilon_{a}(g(x))\\
      &\epsilon_{a}(f(x)g(x))=f(a)g(a)=\epsilon_{a}(f(x))\epsilon_{a}(g(x))\\
      &\epsilon_{a}(1_{R[x]})=1_{R}\\
      &\text{For any }r\in R,\;\epsilon_{a}(r)=r\text{ so }\epsilon_{a}\text{ is surjective.}
  \end{align*}
\end{proof}
\begin{lem}[Telescoping Identity]
  For any $k\in \NN$ and any polynomial $x^{k}-y^{k}$ over a commutative ring $R$ where $x,y$ may either be indeterminates or elements of $R$, we have that:
    $$x^{k}-y^{k}=(x-y)\sum_{\substack{i+j=k-1\\ i,j\in \NN}}x^{i}y^{j}.$$
\end{lem}
\begin{proof}
  For $k>0$:
  \begin{align*}
    x\sum_{\substack{i+j=k-1\\ i,j\in \NN}}x^{i}y^{i}&=x(x^{k-1}+x^{k-2}y+\cdots +xy^{k-2}+y^{k-1})=x^{k}+(x^{k-1}y+\cdots +x^{2}y^{k-2}+xy^{k-1})\\
    y\sum_{\substack{i+j=k-1\\ i,j\in \NN}}x^{i}y^{i}&= y(x^{k-1}+x^{k-2}y+\cdots +xy^{k-2}+y^{k-1})=(yx^{k-1}+x^{k-2}y^{2}+\cdots xy^{k-1})+y^{k}\\
    &=(x^{k-1}y+x^{k-2}y^{2}+\cdots +xy^{k-1})+y^{k}\\
    \implies (x-y)\sum_{\substack{i+j=k-1\\ i,j\in \NN}}x^{i}y^{i}&=x^{k}+(x^{k-1}y+\cdots +x^{2}y^{k-2}+xy^{k-1})-(x^{k-1}y+x^{k-2}y^{2}+\cdots +xy^{k-1})-y^{k}\\
    &=x^{k}-y^{k}.
  \end{align*}
\end{proof}
%%%%%%%%%%%%%%%%%%%%%%%%%%%%%%%%%%%%%%%%%%%%%%%%% 1 %%%%%%%%%%%%%%%%%%%%%%%%%%%%%%%%%%%%%%%%
\newpage
\setcounter{thm}{0} % next prob is 1
\begin{prob}
  In the following problems, we investigate the relationship between integral domains and fields.
  \begin{enumerate}[label=(\alph*)]
    \item Prove that every field is an integral domain.
    \item Prove that if $R$ is a finite integral domain, then $R$ is a field. \tbf{Hint:} Consider the set function $a:R\to R$ given by multiplication by an element $a$ (it will not be a ring homomorphism). If $a$ is a non-zero-divisor, prove this is an injective function, and so must also be surjective since $R$ has finitely many elements.
    \item Find an example of an integral domain which is not a field.
  \end{enumerate}
\end{prob}

\begin{proof}
    \textbf{(a) } Let $\KK$ be a field, and suppose it contains some non-zero zero divisor $b\in \KK\setminus\{0\}$. Then there exists some $a\in \KK\setminus\{0\}$ such that $ab=ba=0$. Since $a,b\in\KK\setminus\{0\}$, they are both units and so there exist $a^{-1},b^{-1}\in \KK$ such that $a^{-1}a=bb^{-1}=1\implies a^{-1}(ab)b^{-1}=1=a^{-1}(0)b^{-1}=0\implies 1=0$. But then $\KK=\{0\}$ which is not a field, a contradiction. Therefore, all non-zero elements are not zero divisors and so $\KK$ is a commutative ring with unity and no zero divisors, which is the definition of an integral domain.

    \textbf{(b) } Let $R$ be a finite integral domain, and let $a$ be some non-zero zero divisor $a\in R\setminus\{0\}$. We show the mapping $a:R\to R$ via $a(x)=ax,\;\forall x\in R$ is a bijection. For all $x,y\in R$,
    \begin{align*}
        &\textbf{[1-1] }a(x)=a(y)\implies ax=ay\implies ax-ay=a(x-y)=0\implies x=y\\
        &\text{otherwise } a\text{ is a zero divisor, and we get a contradiction.}\\
        &\textbf{[Onto] }a:R\hookrightarrow R \implies \text{ the domain is in bijection with its image }\implies |R|=|a(R)|\\
        &\text{and since }a(R)\subseteq R\text{ (the domain is equal to the codomain), we must have that }a(R)=R. 
    \end{align*}
    (This image is just the orbit of $a$ in the canonical multiplicative monoid action of $R$ on itself.) So then for each non-zero zero divisor $a\in R\setminus\{0\}$, $\exists x\in R$ such that $a(x)=ax=1$. That is, each non-zero zero divisor, which is just all of $R\setminus\{0\}$ is a unit. Therefore, $R$ is a commutative ring with unity and inverses for all non-zero elements, which is the definition of a field.

    \textbf{(c) }$\ZZ$ is an integral domain but not a field.

\end{proof}
%%%%%%%%%%%%%%%%%%%%%%%%%%%%%%%%%%%%%%%%%%%%%%%%% 2 %%%%%%%%%%%%%%%%%%%%%%%%%%%%%%%%%%%%%%%%
\begin{prob}
Let $I,J$ be ideals in a commutative ring $R$ such that $I+J=(1)$. Prove that $IJ=I\cap J$.
\end{prob}

\begin{proof}
    $(\subseteq)$ Recall that since $I,J$ are ideals of $R$, $ir\in I$ and $rj\in J,\;\forall i\in I,\forall j\in J,\forall r\in R$. So then for all $ij\in IJ$, we have that $(i)j\in I$ and $i(j)\in J\implies ij\in I\cap J\implies IJ\subseteq I\cap J$.
    
    $(\supseteq)$ $I+J=\langle 1\rangle\implies \hat{i}+\hat{j}=1$ for some $\hat{i}\in I,\hat{j}\in J$. Then for all $a\in I\cap J$, $a=a(1)=(1)a=a(\hat{i}+\hat{j})=a\hat{i}+a\hat{j}\in IJ+JI=IJ,$ since commutativity in $R$ implies $IJ=JI$. Therefore, $I\cap J\subseteq IJ$.

    Thus, 
      $$IJ=I\cap J$$.

\end{proof}
%%%%%%%%%%%%%%%%%%%%%%%%%%%%%%%%%%%%%%%%%%%%%%%%% 3 %%%%%%%%%%%%%%%%%%%%%%%%%%%%%%%%%%%%%%%%
\newpage
\begin{prob}
  Let $\varphi:R\to S$ be a ring homomorphism, and let $J$ be an ideal of $S$. Prove that $I=\varphi^{-1}(J)=\{i\in R\mid \varphi(i)\in J\}$ is an ideal of $R$.
\end{prob}

\begin{proof}
  $\forall a,b\in \varphi^{-1}(J),\;\varphi(a),\varphi(b)\in J\text{ and so }\varphi(a-b)=\varphi(a)-\varphi(b)\in J.$ So then $a-b\in \varphi^{-1}(J)$ and $\varphi^{-1}(J)\leq_{+} R$.

  Next, for any $r\in R$ and any $i\in \varphi^{-1}(J),\; \varphi(ri)=\varphi(r)\varphi(i)\in J$ and $\varphi(ir)=\varphi(i)\varphi(r)\in J$, since $J$ is an ideal of $S$. So then $ri,ir\in \varphi^{-1}(J).$

  Thus, 
  \begin{center}
    If $\varphi:R\to S$ is a ring homomorphism and $J$ is an ideal of $S$, then $\varphi^{-1}(J)$ is an ideal of $R$.
  \end{center}
\end{proof}
%%%%%%%%%%%%%%%%%%%%%%%%%%%%%%%%%%%%%%%%%%%%%%%%% 4 %%%%%%%%%%%%%%%%%%%%%%%%%%%%%%%%%%%%%%%%
\newpage
\setcounter{thm}{3} % next prob is 4
\begin{prob}
  Let $\varphi:R\to S$ be a ring homomorphism, and let $J$ be an ideal of $R$.
  \begin{enumerate}[label=(\alph*)]
    \item Prove that $\varphi(J)=\{\varphi(j)\mid j\in J\}$ need not be an ideal of $S$.
    \item Prove that if $\varphi$ is surjective, then $\varphi(J)$ is an ideal of $S$.
    \item Prove that if $\varphi$ is surjective, and $I=\ker\varphi$, then $S\cong R/I$ and if we let $\overline{J}\subseteq R/I$ be the image of $\varphi(J)$ under this isomorphism, then
    $$
    (R/I)/\overline{J}\cong R/(I+J).
    $$
  \end{enumerate}
\end{prob}

\begin{proof}
  \textbf{(a)} Consider the inclusion map $\iota:\ZZ\hookrightarrow \RR$ defined by $\iota(n)=n\in R,\;\forall n\in \ZZ.$ This is a ring embedding by \textbf{Lemma 2}. Now look at $2\in 2\ZZ$, which is a well-known ideal of $\ZZ$. Well, $\pi\in \RR$ but $2\pi=\pi 2\not\in \iota(2\ZZ)=2\ZZ$. So then $\iota(2\ZZ)$ is not an ideal and we see that the ring homomorphic image of an ideal need not be an ideal of the codomain.

  \textbf{(b)} For all $\varphi(a),\varphi(b)\in \varphi(J)$ with preimages $a,b\in J$, $\varphi(a)-\varphi(b)=\varphi(a-b)\in \varphi(J)$ since $a-b\in J$. Therefore $\varphi(J)\leq_{+} S$. Next, since $\varphi$ is surjective, for any $s\in S$, $\exists r\in R$ such that $s=\varphi(r)$. So then $s\varphi(a)=\varphi(r)\varphi(a)=\varphi(ra)$ and $\varphi(a)s=\varphi(a)\varphi(r)=\varphi(ar)$ which must both belong to $\varphi(J)$ since $J\subseteq R$ is an ideal $\implies ar,rs\in J$. Therefore, $\varphi(J)$ is an ideal of $S$. That is, a surjective ring homomorphic image of an ideal is in fact an ideal of the codomain.

  \textbf{(c)} $\varphi$ is surjective, so by the First Isomorphism Theorem $R/\ker \varphi = R/I\cong \varphi(R)=S$ and so $S\cong R/\ker\varphi=R/I$. Let $\psi:S\to R/I$ be this pullback isomorphism. That is,
  \begin{align}
      &\overbrace{r}^{R}\overset{\varphi}{\mapsto}\overbrace{\varphi(r)=s}^{S}\overset{\psi}{\longleftrightarrow}\overbrace{\class[I]{r}=\class[\ker \varphi]{r}}^{R/I=R/\ker \varphi}\\
      \text{So: }&\psi(s)=\psi(\varphi(r))=\class[I]{r}\in R/I,\;\text{ for each }s=\varphi(r)\in \varphi(R)=S\\
      \implies &\psi(\varphi(j))=\class[I]{j}\in R/I\;\text{ for each }j\in J\\
      \text{Now let: }&\overline{J}=\psi(\varphi(J))=\{\psi(\varphi(j))=\class[I]{j}\mid j\in J\}\subseteq R/I
  \end{align}
  Recall the Third Isomorphism Theorem. For ideals $A,B$ of a ring $R$ where $A\subseteq B$ is a subset:
    $$\textcircled{\footnotesize 1}\; B/A\text{ is an ideal of }R/A\text{ and }\textcircled{\footnotesize 2}\;\frac{(R/A)}{(B/A)}\cong \frac{R}{B}.$$
  \textit{* Note that in normal rings without unity required, $B/A$ is actually a quotient ring since ideals are always rings. That is: $A$ is an ideal of $B$ if every ideal is a ring. That need not be true if you require rings to have unity and then $B/A$ need not be a quotient ring, but just a set of cosets. However, the resulting quotient is always a quotient ring since the set of cosets $(B/A)$ is an ideal of the quotient ring $(R/A)$. Yet another reason why (a) this coset notation sucks, and (b) this ring definition sucks.*} 

  By \textbf{Lemma 1}, $I+J$ is an ideal of $R$ since $I,J$ are ideals of $R$. Also, $I\subseteq I+J$ is a subset. Well, we can simply compute that $(I+J)/I=\{(i+j)+I\mid (i+j)\in I+J\}=\{j+I=\class[I]{j}\mid j\in J\}=\overline{J}$ by $(4)$. 

  Therefore, since $I,(I+J)$ are ideals of $R$ with $I\subseteq (I+J)$, by the Third Isomorphism Theorem we have that
    $$\textcircled{\footnotesize 1}\; \overline{J}=(I+J)/I\text{ is an ideal of $(R/I)$ and }\textcircled{\footnotesize 2}\;\frac{(R/I)}{\overline{J}}=\frac{(R/I)}{(I+J)/I}\cong \frac{R}{(I+J)}.$$
\end{proof}

%%%%%%%%%%%%%%%%%%%%%%%%%%%%%%%%%%%%%%%%%%%%%%%%% 5 %%%%%%%%%%%%%%%%%%%%%%%%%%%%%%%%%%%%%%%%

\begin{prob}
Let $R$ be a commutative ring, $a\in R$, and let $f_1(x),\ldots,f_r(x)\in R[x]$.
\begin{enumerate}[label=(\alph*)]
\item Prove that $R[x]/(x-a)\cong R$.
\item Prove the equality of ideals
$$
(f_1(x),\ldots,f_n(x),x-a)=(f_1(a),\ldots,f_n(a),x-a).
$$
\item Prove the useful substitution trick
$$
R[x]/(f_1(x),\ldots,f_n(x),x-a)\cong R/(f_1(a),\ldots,f_n(a)).
$$
\tbf{Hint:} Use part (c) of the previous problem.
\end{enumerate}
\end{prob}
\begin{proof}
\textbf{(a)} Consider $\epsilon_{a}:R[x]\to R$ defined by $\epsilon_{a}(f(x))=f(a)$, an evaluation which was proven to be a \textit {surjective homomorphism} in \textbf{Lemma 3}. We show that $\ker \epsilon_{a}=\langle x-a\rangle=\{f(x)(x-a)\mid f(x)\in R[x]\}.$ 

$(\subseteq)$ For any $B(x)=\sum_{k=0}^{n}b_{k}x^{k}\in \ker \varepsilon_{a},\; B(a)=0$. Well, $B(x)-B(a)=(\sum_{k=0}^{n}b_{k}x^{k})-(\sum_{k=0}^{n}b_{k}a^{k})=\sum_{k=0}^{n}b_{k}\newline(x^{k}-a^{k})=\sum_{k=1}^{n}b_{k}(x^{k}-a^{k})$. Then by $\textbf{Lemma 4},\;(x^{k}-a^{k})=(x-a)\sum_{i+j=k-1}x^{i}y^{j}$ for each $1\leq k\leq n$. Therefore,
$$B(x)-B(a)=\sum_{k=1}^{n}b_{k}(x^{k}-a^{k})=\sum_{k=1}^{n}b_{k}((x-a)\sum_{i+j=k-1}x^{i}a^{j})=(x-a)\sum_{k=1}^{n}b_{k}(\sum_{i+j=k-1}x^{i}y^{j})$$
and since in fact $B(a)=0$, we have that $B(x)=(x-a)\sum\limits_{k=1}^{n}b_{k}(\sum\limits_{i+j=k-1}x^{i}y^{j})\in \langle x-a\rangle$ since $\sum\limits_{k=1}^{n}b_{k}(\sum\limits_{i+j=k-1}x^{i}y^{j})$ belongs to $R[x]$. So then $\ker \epsilon_{a}\subseteq \langle x-a\rangle.$. 

$(\subseteq)\;\forall C(x)=f(x)(x-a)\in \langle x-a\rangle$ obviously $C(a)=f(a)(a-a)=0\implies C(x)\in \ker \epsilon_{a}$. So then $\langle x-a\rangle\subseteq \ker \epsilon_{a}.$. Thus, by the \textbf{First Isomorphism Theorem},
  $$R[x]/\ker \epsilon_{a}=R[x]/\langle \langle x-a\rangle\cong \epsilon_{a}(R[x])=R.$$

\textbf{(b)} For each $1\leq m\leq n,$ there exist $b_{m,0},\hdots,b_{m,d_{m}}\in R$ and $d_{m}\in \NN$ such that $f_{m}(x)=\sum_{k=0}^{d_{m}}b_{m,k}x^{k}$. Now we use a result obtained in \textbf{(a)}:
  $$f_{k}(x)-f_{k}(a)=(x-a)\sum_{i=1}^{d_{k}}b_{k,i}(\sum_{a+b=i-1}x^{a}y^{a})\implies f_{k}(x)=f_{k}(a)+(x-a)\sum_{i=1}^{d_{k}}b_{k,i}(\sum_{a+b=i-1}x^{a}y^{a})$$
So then 
$$f_{k}(x)\in \langle f_{k}(a),x-a\rangle \subseteq \langle f_1(a),\ldots,f_n(a),x-a\rangle,\;\forall 1\leq k \leq n$$
and
$$f_{k}(a)\in \langle f_{k}(x),x-a\rangle \subseteq \langle f_1(x),\ldots,f_n(x),x-a\rangle,\;\forall 1\leq k \leq n$$
Therefore, all generators of $\langle f_1(x),\ldots,f_n(x),x-a\rangle$ belong to $\langle f_1(a),\ldots,f_n(a),x-a\rangle$ and vice versa. So the ideals are equal.

\end{proof}
\newpage
(c)$$
R[x]/(f_1(x),\ldots,f_n(x),x-a)\cong R/(f_1(a),\ldots,f_n(a)).
$$
\begin{proof}
    \textbf{(c)} In $\textbf{(a)}$ we used the surjective homomorphism $\epsilon_{a}:R[x]\to R$ defined by $\epsilon_{a}(h(x))=h(a)$ to show that $R[x]/\langle x-a\rangle=R[x]/I\cong R$ by the \textbf{First Isomorphism Theorem} where $I=\ker \epsilon_{a}=\langle x-a \rangle$. Much like in $\textbf{Problem 4}$ let $\psi:R\to R[x]/I$ be the pullback isomorphism and let $J=\langle f_1(x),\ldots,f_n(x),x-a\rangle\subseteq R[x]$. Then:
    \setcounter{equation}{0}

      \begin{align}
      &\overbrace{h(x)}^{R[x]}\overset{\epsilon_{a}}{\mapsto}\overbrace{\epsilon_{a}(h(x))=h(a)}^{R}\overset{\psi}{\longleftrightarrow}\overbrace{\class[I]{h(x)}=\class[\ker \epsilon_{a}]{h(x)}}^{R[x]/I =R[x]/\ker \epsilon_{a}}\\
      \text{So: }&\psi(r)=\psi(\epsilon_{a}(h(x)))=\class[I]{h(x)}\in R[x]/I,\;\text{ for each }r=\epsilon_{a}(h(x))\in \epsilon_{a}(R[x])=R\\
      \implies &\psi(\epsilon_{a}(j(x)))=\class[I]{j(x)}\in R[x]/I\;\text{ for each }j(x)\in J\\
      \text{Now let: }&\overline{J}=\psi(\epsilon_{a}(J))=\{\psi(\epsilon_{a}(j))=\class[I]{j(x)}\mid j(x)\in J\}=J/I\subseteq R[x]/I \textit{ *an ideal of $R[x]/I$*}
  \end{align}
  Well, $(I+J)/I=(\langle x-a \rangle +\langle f_1(x),\ldots,f_n(x),x-a\rangle) / \langle x-a \rangle = \langle f_1(x),\ldots,f_n(x),x-a\rangle / \langle x-a \rangle=J/I=\overline{J}$. 
  Since $I,(I+J)$ are ideals of $R[x]$ with $I\subseteq (I+J)$, by the Third Isomorphism Theorem we have that
    $$\textcircled{\footnotesize 1}\; \overline{J}=(I+J)/I\text{ is an ideal of $(R[x]/I)$ and }\textcircled{\footnotesize 2}\;\frac{(R[x]/I)}{\overline{J}}=\frac{(R[x]/I)}{(I+J)/I}\cong \frac{R[x]}{I+J}=R[x]/J.$$
  Now, we pass elements of $\frac{R[x]/I}{\overline{J}}$ through $\Phi:\frac{R[x]/I}{\overline{J}}\to \frac{R[x]}{\langle f_{1}(a),\hdots,f_{n}(a)\rangle }$ defined via $$\Phi(\class[I]{h(x)} + \overline{J})=\psi^{-1}(\class[I]{h(x)})+\psi^{-1}(\overline{J})=\psi^{-1}(\class[I]{h(x)})+\epsilon_{a}(J)$$
  Obviously this is well defined since $\psi^{-1}$ is an isomorphism, and then $\forall \class[\overline{J}]{\class[I]{A(x)}}, \class[\overline{J}]{\class[I]{B(x)}}\in \frac{R[x]/I}{\overline{J}}:$
  \begin{align*}
  &\Phi{\class[\overline{J}]{\class[I]{f(x)}}+\class[\overline{J}]{\class[I]{g(x)}}}=\psi^{-1}(\class[I]{f(x)})+ \epsilon_{a}(J)+\psi^{-1}(g(x))+\epsilon_{a}(J)=\phi^{-1}(f(x)+g(x))+\epsilon_{a}(J)=\\
  \end{align*}

\end{proof}
%%%%%%%%%%%%%%%%%%%%%%%%%%%%%%%%%%%%%%%%%%%%%%%%% 6 %%%%%%%%%%%%%%%%%%%%%%%%%%%%%%%%%%%%%%%%

\begin{prob}
If $\kkk$ is an algebraic closed field, then the only maximal ideals of $\kkk[x]$ are of the form $(x-a)$ where $a\in \kkk$. In this problem, we’ll see that this is not true when $\kkk$ is not algebraically closed.
\begin{enumerate}[label=(\alph*)]
\item Use the first isomorphism theorem to show that $\RR[x]/(x^2+1)\cong \CC$.
\item Prove that $(x^2+1)$ is a maximal ideal of $\RR[x]$.
\end{enumerate}
\end{prob}

%%%%%%%%%%%%%%%%%%%%%%%%%%%%%%%%%%%%%%%%%%%%%%%%% 7 %%%%%%%%%%%%%%%%%%%%%%%%%%%%%%%%%%%%%%%%

\begin{prob}
Let $R$ be a commutative ring with $0\neq 1$. In this problem, we will prove that every proper ideal of $R$ is contained in some maximal ideal.
\begin{enumerate}[label=(\alph*)]
\item Look up Zorn’s Lemma and record it here.
\item Define $S=\{J\mid J\text{ is a proper ideal of }R\text{ and }J\supseteq I\}$. Explain why $S$ is a partially ordered set (what is the ordering?).
\item Given a chain $C$ in $S$, prove that $\bigcap_{J\in C}J$ is an ideal of $R$ (you will use that $C$ is totally ordered), and further that this ideal is in the set $S$.
\item Conclude using Zorn’s Lemma that $S$ has a maximal element.
\end{enumerate}
\end{prob}

%%%%%%%%%%%%%%%%%%%%%%%%%%%%%%%%%%%%%%%%%%%%%%%%% 8 %%%%%%%%%%%%%%%%%%%%%%%%%%%%%%%%%%%%%%%%

\begin{prob}
Let $\kkk$ be a field. In this problem, we will prove that the only maximal ideal of $\kkk[[x]]$ is $(x)$, which makes $\kkk[[x]]$ a local ring.
\begin{enumerate}[label=(\alph*)]
\item Explain why $(x)=\{f\in \kkk[[x]]\mid f\text{ has no constant term}\}$.
\item Compute $\kkk[[x]]/(x)$, and then explain why $(x)$ is a maximal ideal.
\item You may freely use the following result from the optional hint last week: $f\in \kkk[[x]]$ is a unit if and only $f$ has a nonzero constant term. Use Proposition 1.41 to show that the only maximal ideal of $\kkk[[x]]$ is $(x)$.
\end{enumerate}
\end{prob}

%%%%%%%%%%%%%%%%%%%%%%%%%%%%%%%%%%%%%%%%%%%%%%%%% 9 %%%%%%%%%%%%%%%%%%%%%%%%%%%%%%%%%%%%%%%%

\begin{prob}
Let $d$ be an integer which is not the square of an integer, and consider
$$
\QQ(\sqrt{d})=\{a+b\sqrt{d}\mid a,b\in \QQ\}.
$$
\begin{enumerate}[label=(\alph*)]
\item Prove that $\QQ(\sqrt{d})$ is a subring of $\CC$.
\item Define a function $N:\QQ(\sqrt{d})\to \QQ$ by $N(a+b\sqrt{d})=a^2+b^2d$. Prove that $N(zw)=N(z)N(w)$ and that $N(z)\neq 0$ if $z\neq 0$. This function is often called the norm.
\item Prove that $\QQ(\sqrt{d})$ is a field and is the smallest subfield of $\CC$ containing both $\QQ$ and $\sqrt{d}$ (use $N$).
\item Prove that $\QQ(\sqrt{d})\cong \QQ[t]/(t^2-d)$.
\end{enumerate}
\end{prob}

%%%%%%%%%%%%%%%%%%%%%%%%%%%%%%%%%%%%%%%%%%%%%%%%%%%%%%%%%%%%%%%%%%%%%%%%%%%%%%%%%%%%%%%%%%%%%%%%%%%%%

\end{document}