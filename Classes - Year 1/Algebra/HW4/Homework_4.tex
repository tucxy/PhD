\documentclass[addpoints,10pt]{exam}

\usepackage{amsmath,amsthm,enumitem,wrapfig,amsfonts,mathtools}
\usepackage[mathscr]{euscript}
\usepackage[super]{nth}

\usepackage{geometry}
\usepackage[T1]{fontenc} % Use 8-bit encoding that has 256 glyphs
\renewcommand{\rmdefault}{ptm} %Change the Front Family from the default(cmr) to ptm(Times)
\usepackage{amsmath,amsfonts,amsthm,amssymb} % Math packages
\usepackage{bm}
\usepackage{mathptmx}
\usepackage{graphicx}
\usepackage{sectsty} % Allows customizing section commands
% \allsectionsfont{\centering} % Make all sections centered, the default font and small caps
\usepackage{pgfplots}
\pgfplotsset{compat=1.18}
\usetikzlibrary{arrows.meta}
\usepackage{xcolor}
\definecolor{darkpastelgreen}{rgb}{0.01, 0.75, 0.24}
\definecolor{blue-violet}{rgb}{0.54, 0.17, 0.89}
%\usepackage{polylongdiv} %polynomial long division
% Custom problem environment
\usepackage{polynom}
\newcounter{cprob}
\newenvironment{cprob}[1]{%
    \setcounter{cprob}{#1}%
    \noindent\textbf{Problem \thecprob.}%
}{%
    \par\bigskip%
}

\theoremstyle{plain}
\newtheorem{thm}{\protect\theoremname}
  \theoremstyle{definition}
  \newtheorem{prob}[thm]{Problem}
  \newtheorem*{problem*}{Open Problem}
  \theoremstyle{plain}
  \newtheorem{conjecture}[thm]{Conjecture}
  \theoremstyle{plain}
  \newtheorem{lem}[thm]{Lemma}
  \newtheorem*{lem*}{Lemma}
  \newtheorem{obs}[thm]{Observation}
  \newtheorem{cor}[thm]{Corollary}
  \theoremstyle{definition}
\newtheorem{definition}[thm]{Definition}


% Patch prob environment to be single spaced
\let\oldprob\prob
\let\endoldprob\endprob
\renewenvironment{prob}
  {\begin{singlespace}\oldprob}
  {\endoldprob\end{singlespace}}

% start problem one line below like for enumerated problems with multiple parts
\newcommand{\belowtitle}{\leavevmode\newline}
%\Observe command
\newcommand{\Observe}{\text{Observe.}}
%(=>)
\newcommand{\IF}{\mathbf{(\Rightarrow)}}
%(<=)
\newcommand{\FI}{\mathbf{(\Leftarrow)}}
%equivalence classes; \class[S]{ *content in square brackets* }
\newcommand{\class}[2][]{\ensuremath{\left[\,#2\,\right]_{#1}}}

\newcommand{\horrule}[1]{\rule{\linewidth}{#1}}
\newcommand{\kk}{\ensuremath{\Bbbk}} 
\newcommand{\CC}{\ensuremath{\mathbb{C}}}
\newcommand{\FF}{\ensuremath{\mathbb{F}}}
\newcommand{\KK}{\ensuremath{\mathbb{K}}}
\newcommand{\NN}{\ensuremath{\mathbb{N}}}
\newcommand{\QQ}{\ensuremath{\mathbb{Q}}} 
\newcommand{\RR}{\ensuremath{\mathbb{R}}} 
\newcommand{\ZZ}{\ensuremath{\mathbb{Z}}}
\newcommand{\MM}{\ensuremath{\mathcal{M}}}
\newcommand{\TT}{\ensuremath{\mathcal{T}}}
\newcommand{\BB}{\ensuremath{\mathcal{B}}}
\newcommand{\VV}{\ensuremath{\mathcal{V}}}
\newcommand{\WW}{\ensuremath{\mathcal{W}}}
\newcommand{\UU}{\ensuremath{\mathcal{U}}}
\newcommand{\PP}{\ensuremath{\mathcal{P}}}
\newcommand{\LL}{\ensuremath{\mathcal{L}}}

\newcommand{\sm}{\char`\\}
%vector stuff
\DeclarePairedDelimiter{\ip}{\langle}{\rangle} %inner product/generate
\DeclarePairedDelimiter{\norm}{\lVert}{\rVert} %norm
\DeclarePairedDelimiter{\sqb}{\lbrack}{\rbrack} %corrd
\newcommand{\floor}[1]{\left\lfloor #1 \right\rfloor}
\newcommand{\ceil}[1]{\left\lceil #1 \right\rceil}
\newcommand{\mbf}[1]{\ensuremath{\mathbf{#1}}}
\newcommand{\tbf}[1]{\textbf{ #1 }}
\newcommand{\Span}{\ensuremath{\mathrm{Span}}}


\makeatletter
\renewcommand*\env@matrix[1][*\c@MaxMatrixCols c]{%
  \hskip -\arraycolsep
  \let\@ifnextchar\new@ifnextchar
  \array{#1}}
\makeatother

\def\env@matrix{\hskip -\arraycolsep
  \let\@ifnextchar\new@ifnextchar
  \array{*\c@MaxMatrixCols c}}

  \newcommand{\proj}[2]{\text{proj}_{#1}(#2)}
  

%%% Formatting: Page Header
\newcommand{\StudentName}{Danny Banegas}
\newcommand{\AssignmentName}{Homework 3}
\newcommand{\CourseName}{MATH 721 - Algebra II}


\pagestyle{headandfoot}
\runningheadrule
\firstpageheadrule
\firstpageheader{\CourseName}{\StudentName}{\AssignmentName}
\runningheader{\CourseName}{\StudentName}{\AssignmentName}
\firstpagefooter{}{\thepage}{}
\runningfooter{}{\thepage}{}

\printanswers

\DeclareMathAlphabet{\mathcal}{OMS}{cmsy}{m}{n}

\usepackage{parskip}
\usepackage{setspace}
\doublespacing
% % % % % % % % % % % % % % % % % % % % % % % % % % % % % % % % % % % % % % % % % % % % % % % % % % % % % % % % % % % % % % 
\begin{document}
Turn in problems 54, 55, 57, 58, 59, 60, 61, 62, 63, 64, 65.
%%%%%%%%%%%%%%%%%%%%%%%%%%%%%%%%%%% 40 %%%%%%%%%%%%%%%%%%%%%%%%%%%%%%%%%%%%%%

%%%%%%%%%%%%%%%%%%%%%%%%%%%%%%%%%%% 54–65 %%%%%%%%%%%%%%%%%%%%%%%%%%%%%%%%%%%%%%
\setcounter{thm}{53} % next prob is 54

\begin{prob}
If $f\in \KK[X]$ (with $\KK$ field) has degree $n$ and $\FF$ is a splitting field of $f$ over $\KK$, prove that $[\FF:\KK]\mid n!$.
\end{prob}

\begin{proof}
$f$ has at most $n$ distinct zeros. If $n=0$, then $f$ is constant and splits in $\KK=\FF$ so $[\FF:\KK]=1\mid n!$ (arguably). Suppose $n\geq 1$. Let $(a_{i})$ be the $k$-tuple consisting of all $1\leq k\leq n$ distinct zeros of $f$ in (strictly) ascending order; $a_{1}< \cdots < a_{k}$. Then $\FF\cong \KK(a_{1},\hdots,a_{k})$. Now consider $[\FF:\KK]$.


\end{proof}

\begin{prob}
If $K\subseteq F$ is a field extension, $F$ is algebraically closed, and $E$ is the set of all elements of $F$ that are algebraic over $K$, prove that $E$ is an algebraic closure of $K$.
\end{prob}

\setcounter{thm}{56} % skip 56

\begin{prob}
If $[F:K]=2$, then $K\subseteq F$ is a normal extension.
\end{prob}

\begin{prob}
If $d$ is a nonnegative rational number, then $\operatorname{Aut}_{\mathbb{Q}}\!\big(\mathbb{Q}(\sqrt{d})\big)$ is the identity or is isomorphic to $\mathbb{Z}_{2}$.
\end{prob}

\begin{prob}
What is the Galois group of $\mathbb{Q}\!\left(\sqrt{2},\sqrt{3},\sqrt{5}\right)$ over $\mathbb{Q}$?
\end{prob}

\begin{prob}
Assume $K$ is a field of characteristic $0$. Let $G$ be the subgroup of $\operatorname{Aut}_{K}\!\big(K(X)\big)$ generated by the $K$-automorphism induced by $X\mapsto X+1$. Prove that $G$ is an infinite cyclic group. What is the fixed field $E$ of $G$? What is $[K(X):E]$?
\end{prob}

\begin{prob}
Let $k$ be a finite field of characteristic $p>0$.
\begin{enumerate}[label=(\alph*)]
\item Prove that for every $n>0$ there exists an irreducible polynomial $f\in k[X]$ of degree $n$.
\item Prove that for every irreducible polynomial $P\in k[X]$ there exists $n\ge 0$ such that $P$ divides $X^{p^{\,n}}-X$.
\end{enumerate}
\end{prob}

\begin{prob}
Let $p$ be a prime and $\mathbb{F}_{q}$ (with $q=p^{s}$) be the finite field with $q$ elements. Let $f\in \mathbb{F}_{q}[X]$ be an irreducible polynomial. Prove that $f$ is irreducible in $\mathbb{F}_{q^{m}}[X]$ if and only if $m$ and $\deg f$ are relatively prime.
\end{prob}

\begin{prob}
Prove that $E=\mathbb{F}_{2}[X]/(X^{4}+X^{3}+1)$ is a field with $16$ elements. What are the roots of $X^{4}+X^{3}+1$ in $E$?
\end{prob}

\begin{prob}
Prove that an algebraic extension of a perfect field is a perfect field.
\end{prob}

\begin{prob}
Show that the extension $\mathbb{Q}\subseteq \mathbb{Q}\!\left(\sqrt[4]{2},\, i\right)$ is Galois. Find its Galois group.
\end{prob}


\end{document}