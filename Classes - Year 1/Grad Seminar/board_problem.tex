\documentclass[10pt,fleqn]{exam} % fleqn option for left alignment
\usepackage{amsthm}
\usepackage{libertine}
\usepackage[utf8]{inputenc}
\usepackage[margin=1in]{geometry}
\usepackage{amsmath,amssymb}
\usepackage{multicol}
\usepackage[shortlabels]{enumitem}
\usepackage{siunitx}
\usepackage{cancel}
\usepackage{graphicx}
\usepackage{pgfplots}
\usepackage{listings}
\usepackage{tikz}
\usepackage{tcolorbox}
\usepackage{varwidth}

\newtcolorbox{substitution}[1][]{colback=blue!5!white, colframe=blue!75!black, title=Substitution, #1}

\newtcolorbox{subinline}[1][]{
    colback=gray!10!white, 
    colframe=gray!50!black, 
    boxrule=0.5pt, 
    boxsep=1pt, 
    left=1pt, 
    right=1pt, 
    top=1pt, 
    bottom=1pt, 
    sharp corners, 
    #1
}

\pgfplotsset{width=10cm,compat=1.9}
\usepgfplotslibrary{external}
\tikzexternalize

\newcommand{\class}{MATH 790} % This is the name of the course
\newcommand{\examnum}{Reflections Week 2} % This is the name of the assignment
\newcommand{\examdate}{Danny Banegas} % This is the due date
\newcommand{\timelimit}{Graduate Seminar}

\begin{document}
\pagestyle{plain}
\thispagestyle{empty}

\noindent
\begin{tabular*}{\textwidth}{l @{\extracolsep{\fill}} r @{\extracolsep{6pt}} l}
    \textbf{\class} && \\
    \textbf{\timelimit} &&\\
    \textbf{\examdate} &&\\
    \textbf{\examnum} &&
\end{tabular*}

\vspace{0.5cm} % Adjust space after the tabular
\hrule
\vspace{0.5cm}
\noindent
\begin{center}
    \textbf{Reflections:}
\end{center}
$\newline\mathbf{[I]}$ \textbf{Write up a solution via LaTeX of the problem you presented. Send me both the pdf
and the LaTeX code.}

\setlength{\mathindent}{0pt}
$$
    \textbf{(3)} \int \frac{1}{x\sqrt{x^{2}-9}} \, dx\;\;\;
    \boxed{
        \begin{aligned}
            \textbf{Substitution}&:x=a\sec(\theta)\\
            a&=3\\
            x&=3\sec(\theta)\\
            \implies dx&=3\sec(\theta)\tan(\theta)\,d\theta\\
            \implies \theta &=\mathrm{arcsec}(\frac{x}{3})\\
            \sqrt{x^{2}-9}&=3\tan(\theta)
        \end{aligned}
    }
    =\int\frac{1}{\underbrace{(3\sec(\theta))}_{x}\underbrace{(3\tan(\theta))}_{\sqrt{x^{2}-9}}}\overbrace{(3\sec(\theta)\tan(\theta)\,d\theta)}^{dx}
$$
$=\int \frac{1}{3}\,d\theta = \frac{1}{3}\theta + c = \frac{1}{3}\overbrace{\mathrm{arcsec}(\frac{x}{3})}^{\theta}+c\newline$

$\newline\mathbf{[II]}$ \textbf{What do you think you succeed at most when presenting?} \\

I think I generally present things in a simple way; I don't overcomplicate things. I was a bit disorganized on the board this time but usually I am very consistent and organized in how I do problems. Also I come prepared and I am not nervous.

$\newline\mathbf{[III]}$ \textbf{What do you need to improve at?} \\ 

My main problem is talking at the board/to myself rather to the students. I talk and work though problems too fast. I also am not exciting or excited to do math in general, I'm not a \textit{fun} teacher, it's just not really who I am. I need to work on making class more enjoyable for students, and find a better vibe between super casual and strictly business-like or whatever, the latter is all I know how to do.

In general I don't try to innovate, and when I do problems on the board I don't interact with the class at all. Essentially the only engagement I have with students and that the students have with me is when I go around as they work in groups. This is something I would like to change. 

\end{document}